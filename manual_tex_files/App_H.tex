%
\documentclass[10pt]{report}
\usepackage{geometry} 
\geometry{letterpaper}
%
%
%   margins and inter-paragraph spacing
%
%---------------------------------------------
\setlength{\textheight}{630pt}
\setlength{\textwidth}{450pt}
\setlength{\oddsidemargin}{14pt}
\setlength{\parskip}{1ex plus 0.5ex minus 0.2ex}


%----------------------------------------
\usepackage{amsmath}
\usepackage{layout}
\usepackage{color}
%\usepackage{hyphenat}
%\usepackage{listings}

%----------------------------------------------
%
%          --- header and footer contents ---
%
\usepackage{fancyhdr} \pagestyle{fancy}
\setlength\headheight{15pt}
\lhead{\small{User's Guide - \ti{WARP3D}}}
\rhead{\small{\ti{Supported Computer Platforms}}}
\fancyfoot[L] {\small{\ti{Appendix H}\ \   (Updated: 4-8-2014)}}
\fancyfoot[C] {\small{H.\thepage}}
\fancyfoot[R] {\small{\ti{Supported Computer Platforms}}}



%---------------------------------------------------
\usepackage{graphicx}
\usepackage[labelformat=empty]{caption}
\numberwithin{equation}{section}

%---------------------------------------------
%
%     --- make section headers in helvetica ---
%
\frenchspacing
\usepackage{sectsty} 
\usepackage{xspace}
\allsectionsfont{\sffamily} 
\sectionfont{\large}
\usepackage[small,compact]{titlesec} % reduce white space around sections
%
%
%   which fonts system for text and equations. with all commented,
%   the default LaTex CM fonts are used
%
%
%\usepackage{pxfonts}  % Palatino text 
%\usepackage{mathpazo} % Palatino text
%\usepackage{txfonts}

%----------------------------------------------

%   ---  local commands ---

\newcommand{\bmf } {\boldsymbol }
\newcommand{\bsf } [1]{\textrm{\ti{#1}}\xspace}
\newcommand{\HRule}{\rule{\linewidth}{0.5mm}}
\newcommand{\patwarp}{\ti{patwarp\xspace}}
\newcommand{\eg}{\ti{e.g.},\xspace}
\newcommand{\ie}{\ti{i.e.},\xspace}
\newcommand{\ul} {\underline}
\newcommand{\hv} {\mathsf}   %helvetica text inside an equation
\newcommand{\ti}{\emph}

%
%        optional definition for bullet lists which
%        reduces white space.
%
\newcommand{\squishlist}{
 \begin{list}{$\bullet$}
  { \setlength{\itemsep}{0pt}
     \setlength{\parsep}{3pt}
     \setlength{\topsep}{3pt}
     \setlength{\partopsep}{0pt}
     \setlength{\leftmargin}{1.5em}
     \setlength{\labelwidth}{1em}
     \setlength{\labelsep}{0.5em} } }

\newcommand{\squishlisttwo}{
 \begin{list}{$\bullet$}
  { \setlength{\itemsep}{0pt}
     \setlength{\parsep}{0pt}
    \setlength{\topsep}{0pt}
    \setlength{\partopsep}{0pt}
    \setlength{\leftmargin}{2em}
    \setlength{\labelwidth}{1.5em}
    \setlength{\labelsep}{0.5em} } }

\newcommand{\squishend}{
  \end{list}  }
%
%
%     --- set chapter number or appendix letter ---
%
%-------------------------------------
\def\thechapter  {\Alph{chapter}}
\setcounter{chapter}{8} % means 8th alphabet letter
\newcounter{sectrefs}
%
%
%
%              start document 
%              ==========
%
%
\begin{document}
\LARGE
\hfill
\textbf{Appendix H}
\rule[0.15in]{450pt}{0.5mm}
\LARGE
\begin{flushright}
 \textbf{
{\fontfamily{phv}\selectfont Supported Computer Platforms}}
\end{flushright}
\normalsize
%
%---------------------------------------------------------------------------
%        section
%---------------------------------------------------------------------------
%
Support for various CPU architectures and computer platforms are 
added and removed to WARP3D as technology advances. 

Pre-compiled executables are distributed with the code for each 
supported platform. The executables for Windows and 
OS X provide parallel execution via threads and shared memory.
They also use static linking of all required libraries. The Intel compiler 
system is thus \ti{not} required on your systems to use these executables.

The pre-compiled executables on Linux do link with some shared 
libraries from the Intel compiler and MPI systems. We include 
these shared libraries in the WARP3D distribution (see directory 
linux\_pack-ages). Intel grants permission for software developers to re-distribute 
these libraries. Consequently, your Linux systems are not required to have the 
Intel software installed to run the �threads only� executable included in the 
distribution. However, the Intel MPI system must be installed to use the 
executable included in our distribution. The Intel MPI software creates the 
MPI runtime environment at execution startup and provides libraries 
needed to re-build the executable.  


%
%---------------------------------------------------------------------------
%        section
%---------------------------------------------------------------------------
%
\section{Windows}

The distribution includes a pre-compiled executable for 64-bit Windows (built on
Windows 7 SP1).
To re-build the Windows executable to include any modifications to the source code,
the Intel Fortran Composer XE system must be installed on your computer. We use Release
2013  of 
the Intel software at this time. Future versions of the Intel compiler system should remain
compatible with the WARP3D source code.


%
%---------------------------------------------------------------------------
%        section
%---------------------------------------------------------------------------
%
\section{Linux}

The pre-compiled executables run on 64-bit Linux systems. Our tests include
RedHat and Ubuntu distributions.

To re-build the threads-only Linux executable to include modifications to the source code,
the Intel Fortran Composer XE 2013 for Linux system must be installed 
on your computer.  Future versions of the Intel compiler system should remain
compatible with the WARP3D source code. To build the MPI + threads executable you will
also need the Intel MPI 4 system installed. We use Update 4 at present. The Intel MPI software is
also required to run the pre-compiled MPI + threads executable.


%
%---------------------------------------------------------------------------
%        section
%---------------------------------------------------------------------------
%
\section{Mac OS X}

The pre-compiled executable runs on Mac OS X computers under Mavericks
(10.9.x).
The Mac hardware must be 
Intel Core 2 Duo or later. WARP3D runs parallel using only threads at this time.

To re-build the executable to include modifications to the source code,
the Intel Fortran Composer XE 2013  for Mac OS X system must be installed 
on your computer. We use release 14.0.1 of 
the Intel software at this time. Future versions of the Intel compiler system 
should remain
compatible with the WARP3D source code
\end{document}



%
\documentclass[10pt]{report}
\usepackage{geometry} 
\geometry{letterpaper}
%
%
%   margins and inter-paragraph spacing
%
%---------------------------------------------
\setlength{\textheight}{630pt}
\setlength{\textwidth}{450pt}
\setlength{\oddsidemargin}{14pt}
\setlength{\parskip}{1ex plus 0.5ex minus 0.2ex}


%----------------------------------------
\usepackage{amsmath}
\usepackage{layout}
\usepackage{color}
%\usepackage{hyphenat}
%\usepackage{listings}

%----------------------------------------------
%
%          --- header and footer contents ---
%
\usepackage{fancyhdr} \pagestyle{fancy}
\setlength\headheight{15pt}
\lhead{\small{User's Guide - \ti{WARP3D}}}
\rhead{\small{\ti{Flat result Files}}}
\fancyfoot[L] {\small{\ti{Appendix K}\ \   (Updated: 9/25/2014)}}
\fancyfoot[C] {\small{K.\thepage}}
\fancyfoot[R] {\small{\ti{Flat Result Files}}}



%---------------------------------------------------
\usepackage{graphicx}
\usepackage[labelformat=empty]{caption}
\numberwithin{equation}{section}

%---------------------------------------------
%
%     --- make section headers in helvetica ---
%
\frenchspacing
\usepackage{sectsty} 
\usepackage{xspace}
\allsectionsfont{\sffamily} 
\sectionfont{\large}
\usepackage[small,compact]{titlesec} % reduce white space around sections
%
%
%   which fonts system for text and equations. with all commented,
%   the default LaTex CM fonts are used
%
%
%\usepackage{pxfonts}  % Palatino text 
%\usepackage{mathpazo} % Palatino text
%\usepackage{txfonts}

%----------------------------------------------

%   ---  local commands ---

\newcommand{\bmf } {\boldsymbol }
\newcommand{\bsf } [1]{\textrm{\ti{#1}}\xspace}
\newcommand{\HRule}{\rule{\linewidth}{0.5mm}}
\newcommand{\patwarp}{\ti{patwarp\xspace}}
\newcommand{\eg}{\ti{e.g.},\xspace}
\newcommand{\ie}{\ti{i.e.},\xspace}
\newcommand{\ul} {\underline}
\newcommand{\hv} {\mathsf}   %helvetica text inside an equation
\newcommand{\ti}{\emph}

%
%        optional definition for bullet lists which
%        reduces white space.
%
\newcommand{\squishlist}{
 \begin{list}{$\bullet$}
  { \setlength{\itemsep}{0pt}
     \setlength{\parsep}{3pt}
     \setlength{\topsep}{3pt}
     \setlength{\partopsep}{0pt}
     \setlength{\leftmargin}{1.5em}
     \setlength{\labelwidth}{1em}
     \setlength{\labelsep}{0.5em} } }

\newcommand{\squishlisttwo}{
 \begin{list}{$\bullet$}
  { \setlength{\itemsep}{0pt}
     \setlength{\parsep}{0pt}
    \setlength{\topsep}{0pt}
    \setlength{\partopsep}{0pt}
    \setlength{\leftmargin}{2em}
    \setlength{\labelwidth}{1.5em}
    \setlength{\labelsep}{0.5em} } }

\newcommand{\squishend}{
  \end{list}  }
%
%
%     --- set chapter number or appendix letter ---
%
%-------------------------------------
\def\thechapter  {\Alph{chapter}}
\setcounter{chapter}{11} % means 6th alphabet letter
\newcounter{sectrefs}
%
%
%
%              start document 
%              ==========
%
%
\begin{document}
\LARGE
\hfill
\textbf{Appendix K}
\rule[0.15in]{450pt}{0.5mm}
\LARGE
\begin{flushright}
 \textbf{
{\fontfamily{phv}\selectfont Flat Model and Result Files for Python, Excel, Matlab \& Others}}
\end{flushright}
\normalsize
%
%---------------------------------------------------------------------------
%        section
%---------------------------------------------------------------------------
%
Section 2.12.6 describes the command to generate a \ti{flat} model description
file and the format of that file.  The simple format makes possible efficient
reading of the file in Python, C, C++, and other scripting/programming
languages. An example Python script here illustrates reading of the
model into simple arrays.


Section 2.12.4 outlines the concepts of  \ti{flat} result files in
\ti{text} and \ti{stream} (binary) forms to simplify further the access to WARP3D
results using Python and other software
including Excel, Matlab, Mathematica, and standard languages C, C++, Fortran.

The flat files have no additional information in a readable format about the
model, for example, the number of nodes and elements.
The concept is to have a simple 2D array of data on the file,
where each row contains results for a model node or an element (without
node numbers or element numbers being stored).\footnote{See later section for handling of 
files produced during MPI-based executions.}

The flat text file produced by WARP3D has comment lines at the head of the
file which start with a \# consistent
with the capabilities of the \ti{loadtxt} function in Python. Before importing into
Excel, these comments lines may need to be deleted.
The stream file contains only double precision numbers (\ie float64).

The processing time to load large text result files with Python can be considerable.
A 500,000  node result file may take 15+ secs to load (wall time) on a fast machine
using  \ti{loadtxt}. 
The equivalent stream file may require $<$ 0.05 secs (wall time) using
\ti{fromfile}.

The averaging process to create nodal values and element centroidal
values and the data stored in each column of a row  are
described in Section 2.12.4. The remainder of this short appendix provides
example Python code fragments to read the text and stream files
into 2D array types (\ti{numpy}).
%
%---------------------------------------------------------------------------
%        section
%---------------------------------------------------------------------------
%
\section{Model Description File }

The format of the file is described in Section 2.12.6.
Here is a Python code segment.  This code reads a model 
description file for a model with 50K nodes and 50K
elements in about 1 second of wall time.
\small
\begin{verbatim}
from numpy import *
import os
import time
#
#         Example program to read the flat model description file 
#         into simple numpy arrays. design here is for speed.
#
print( "\n... Example to read a flat model descritpion file\n")
t0 = time.time()
#
directory = "./"
text_fname = directory + "model_flat.text"
file = open( text_fname, 'r' )
#
#         skip the 7 comment lines 
#
for i in xrange(7):
  line = file.readline()
#
#         number of nodes, elements
#
num_nodes = fromfile(file, dtype=int32, sep=" ", count=1)
num_elements = fromfile(file, dtype=int32, sep=" ", count=1)
print( "... Number of nodes, elements: %7d  %7d" 
       % ( num_nodes, num_elements ) )
#
#         read x, y, z coordinates into a [num_nodes,3] array
#         print a few values
#
print( "... Reading coordinates..." )
coords = fromfile( file, dtype=float64, sep=" ", 
                   count=3*num_nodes).reshape(num_nodes,3)
print( "\tNode 1 coords:      %20.9e  %20.9e  %20.9e" % 
       ( coords[0,0], coords[0,1], coords[0,2] ) )
print( "\tNode 2 coords:      %20.9e  %20.9e  %20.9e" % 
       ( coords[1,0], coords[1,1], coords[1,2] ) )
last = num_nodes - 1
print( "\tNode last coords:   %20.9e  %20.9e  %20.9e" % 
       ( coords[last,0], coords[last,1], coords[last,2] ) )
#
#         read element data into a num_elements x 29 array.
#         print values for last element
#
print( "... Reading element data..." )
edata = fromfile( file, dtype=int32, sep=" ", 
                  count=29*num_elements).reshape(num_elements,29)
last = num_elements -1
print( "\tLast element values:" )
print( "\t  etype, material number: %5d  %5d" % 
      ( edata[last,0], edata[last,1] ) )
print( "\t  incidences:")
loc = 2
for i in xrange(3):
   if i > 0 :
     print( "\n" ),  # trailing , suppresses \n
   print( "\t\t" ),
   for j in xrange(9):
     print( "%7d " % (edata[last,loc]) ),
     loc += 1
print( "\n" )
     
file.close()
print("... Walltime used (secs): %8.2f" % ( time.time() - t0 ) )
print("... All done")
exit(0)
\end{verbatim}
\normalsize

\noindent

%
%---------------------------------------------------------------------------
%        section
%---------------------------------------------------------------------------
%

\section{Text Result Files }

Text files of nodal results contain one row of data for each node of the model -- each value
is written in E format.
Element result files contain one row of data (E format) for each element (centroidal values).
The following segment of Python code illustrates reading of a flat text file of nodal stresses. 
The entire array
is read using \ti{loadtxt}, then columns of data are broken out into arrays and vectors with
more convenient names for subsequent processing, plotting, etc. \ti{loadtxt} allows comment 
lines anywhere in the file
beginning with \# in column 1. The number of model nodes is not defined in the flat file and is here
set directly in the Python code.

The first few lines of the data file will have the appearance below -- in this example load step 152 with
WARP3D compressing the text file into .gz format.  WARP3D writes the comments lines
into the file. \ti{loadtxt} automatically uncompresses the the text file.
\small
\begin{verbatim}
#
#   WARP3D nodal results in flat text file: stresses
#   Structure name:  seb
#   Model nodes, elements: 23473  29645
#   Date: 4/7/2014
#
#   Load(time) step: 00152
#
 0.49752e+02  0.65232e+01  ....  26 columns of data ..... 0.43063e+00  0.43058e+01 
          .
          .
          1 row for each node
          .
          . 
\end{verbatim}
\normalsize

Here is a Python code segment. 
\small
\begin{verbatim}
from numpy import *
import os
import time
#
directory = './'
text_fname = directory + "wns00152_text.gz"
#
num_nodes = 23473
num_values = 26  # number of columns
#
start_time = time.time()
data = zeros( [num_nodes,26], float64 ) # pre-allocate to correct size
data = loadtxt(text_fname, dtype=float64, comments='#')
end_time = time.time()
print "\t\t Time for file load with loadtxt: %10.3f (secs)"% ( 
     end_time-start_time ) 
#
#    extract columns of results table as needed with 
#    convenient names
# 
sigma_xx = data[:,0] 
sigma_yy = data[:,1]  
sigma_zz = data[:,2]  
tau_xy      = data[:,3]  
tau_yz      = data[:,4]  
tau_xz      = data[:,5]
mises       = data[:,7]
material_c3 = data[:,10]
....
....     code to process results, print results, make plots, etc.
....

exit(0)
\end{verbatim}
\normalsize

\noindent
Loops may be readily added to process all nodal stress result files to extract results,
plot results, etc.

%
%---------------------------------------------------------------------------
%        section
%---------------------------------------------------------------------------
%
\section{Stream Result Files }

Stream (binary) files contain double precision (float64) values one after the other with no
additional identifying of header  information on the file.  The Python code segment to read the
above results stored in stream format is shown here. Note the \ti{reshape} operator as part of
\ti{fromfile} to map the data into a 2D array with the specified number of rows and columns.


\small
\begin{verbatim}
from numpy import *
import os
import time
#
directory = './'
stream_fname = directory + "wns00152_stream"
#
num_nodes = 23473
num_values = 26  # number of columns
#
start_time = time.time()
fileobj = open(stream_fname, mode='rb')
num_vals_to_read = num_nodes*num_values
data = fromfile(file=fileobj,count=num_vals_to_read, 
                   dtype=float64).reshape(num_nodes, num_values)
end_time = time.time() 
print "\t\t Time for read, arrange stream file: %10.3f (secs)"% ( 
     end_time-start_time ) 
#
#    extract columns of results table as needed with 
#    convenient names
# 
sigma_xx = data[:,0] 
sigma_yy = data[:,1]  
sigma_zz = data[:,2]  
tau_xy      = data[:,3]  
tau_yz      = data[:,4]  
tau_xz      = data[:,5]
mises       = data[:,7]
material_c3 = data[:,10]
....
....     code to process results, print results, make plots, etc.
....

exit(0)
\end{verbatim}
\normalsize


%
%---------------------------------------------------------------------------
%        section
%---------------------------------------------------------------------------
%
\section{Result Files from MPI-based execution}

MPI-based executions distribute the model definition 
and results across the MPI ranks.
The handling of distributed results in response to an
\ti{output flat ...}
command is described below. The processing issues here are identical to those for
generation, and subsequent combination, of Patran result files during an MPI-based execution.

\subsection{Nodal values of vector quantities}
WARP3D maintains the values of key (vector type) nodal values for all model nodes on rank 0.
These include:  displacements, velocities, accelerations, reactions and temperatures.
Rank 0 may then generate flat files that have the same content and formats 
as those produced in analyses without MPI-based
execution. Example: a model has 40 domains and thus runs on 40 MPI ranks. An
\ti{output flat text displacements} command generates a single file of
results for a load step. 

\subsection{Nodal values of averaged strains and stresses}
Each rank has strain-stress values only for the elements it owns. In each 
rank, element results are extrapolated to the
nodes and values from elements are summed at the nodes 
(but not averaged) -- nodes internal to the domain and nodes on the domain
boundary that are shared with other domains.
A flat file is then written by each domain (rank) with the name that contains the rank number as a suffix.
Example: a model has 40 domains and runs on 40 MPI ranks.  Results are requested for load step 50 with a
command of the form \ti{output flat text compressed nodal stresses}. Forty files with names of the form
wns00050\_text0023.gz  are generated (here 23 is the specific rank number and text files 
have been compressed with gzip by WARP3D).\footnote{Automatic compression of text files 
by WARP3D is not available in Windows.}

The flat file generated by a rank contains results only for the internal and boundary nodes of the domain.
In each of these files of {partial} nodal results for the model, 2 additional data columns are added before
strain-stress values for each node: (1) the node number and (2) the count of the number of elements in the
domain that added extrapolated values
to the node. The remainder of the nodal values (summed element contributions)
are those as described in tables included in Chapter 2 (output).
The additional two columns of data are required to build a complete, single
set of averaged nodal values for the model once all partial flat fields are completed.

Programs external to WARP3D are provided to perform the combination, averaging and generation of
a single, flat file of averaged strain-stress results for the entire model at the load step.

To maintain the homogeneous structure of the flat file concept, the 2 additional data columns are 
written in double precision (float64) for \ti{stream} files and E-format for \ti{text} files.

\subsection{Element values of strains and stresses}
Each rank has results only for the elements it owns and writes a partial, flat file of centroidal strain-stress 
values. Example: a model has 40 domains and runs on 40 MPI ranks.  Results are requested for load step 50 with a
command of the form \ti{output flat text compressed element stresses}. Forty files with names of the form
wes00050\_text0023.gz  are generated (here 23 is the specific rank number and text files 
have been compressed with gzip).The file contains results only for the elements owned by domain 23. 

The flat files for element results have the element number included before the usual strain-stress
values as described in tables included in Chapter 2 (output). 

Programs external to WARP3D are provided to combine all the partial, domain flat files of
element results into a single file that contain values for all elements at a load step.

To maintain the homogeneous structure of the flat file concept, the 1 
additional data column (element number) is 
written in double precision (float64) for \ti{stream} files and E-format for \ti{text} files.


\end{document}

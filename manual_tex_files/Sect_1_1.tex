%!TEX TS-program=pdflatexmk
\documentclass[11pt]{report}
\usepackage{geometry} 
\geometry{letterpaper}

%---------------------------------------------
\setlength{\textheight}{630pt}
\setlength{\textwidth}{450pt}
\setlength{\oddsidemargin}{14pt}
\setlength{\parskip}{1ex plus 0.5ex minus 0.2ex}


%----------------------------------------
\usepackage{amsmath}
\usepackage{layout}
\usepackage{color}
\usepackage{array}

%----------------------------------------------
\usepackage{fancyhdr} \pagestyle{fancy}
\setlength\headheight{15pt}
\lhead{\small{User's Guide - \ti{WARP3D}}}
\rhead{\small{\ti{Overview}}}
\fancyfoot[L] {\small{\ti{Chapter {\thechapter}}\ \   (Updated: 7-6-2014)}}
\fancyfoot[C] {\small{\thesection-\thepage}}
\fancyfoot[R] {\small{\ti{Overview}}}

%---------------------------------------------------
\usepackage{graphicx} 
\usepackage[labelformat=empty]{caption}
\numberwithin{equation}{section}

%---------------------------------------------
%     --- make section headers in helvetica ---
%
\usepackage{sectsty} 
\usepackage{xspace}
\allsectionsfont{\sffamily} 
\sectionfont{\large}
\usepackage[small,compact]{titlesec} % reduce white space around sections
%---------------------------------------------->
%
%
%   which fonts system for text and equations. with all commented,
%   the default LaTex CM fonts are used
%
%
\frenchspacing
%\usepackage{pxfonts}  % Palatino text 
%\usepackage{mathpazo} % Palatino text
%\usepackage{txfonts}


%---------  local commands ---------------------


\newcommand{\bmf } {\boldsymbol }  %bold math symbol
\newcommand{\bsf } [1]{\textrm{\ti{#1}}\xspace}
\newcommand{\ul} {\underline}
\newcommand{\hv} {\mathsf}   %helvetica text inside an equation
\newcommand{\eg}{\emph{e.g.},\xspace}
\newcommand{\ie}{\emph{i.e.},\xspace}
\newcommand{\vs}{\emph{vs.}\xspace}
\newcommand{\ti}{\emph}
\newcommand{\veps}{\varepsilon}
\newcommand{\ol}{\overline}
\newcommand{\nid}{\noindent}

\newenvironment{offsetpar}[1]
{\begin{list}{}%
         {\setlength{\leftmargin}{#1}}%
         \item[]%
}
{\end{list}}

%
%
%        optional definition for bullet lists which
%        reduces white space.
%
\newcommand{\squishlist}{
 \begin{list}{$\bullet$}
  { \setlength{\itemsep}{0pt}
     \setlength{\parsep}{3pt}
     \setlength{\topsep}{3pt}
     \setlength{\partopsep}{0pt}
     \setlength{\leftmargin}{1.5em}
     \setlength{\labelwidth}{1em}
     \setlength{\labelsep}{0.5em} } }

\newcommand{\squishlisttwo}{
 \begin{list}{$\bullet$}
  { \setlength{\itemsep}{0pt}
     \setlength{\parsep}{0pt}
    \setlength{\topsep}{0pt}
    \setlength{\partopsep}{0pt}
    \setlength{\leftmargin}{2em}
    \setlength{\labelwidth}{1.5em}
    \setlength{\labelsep}{0.5em} } }

\newcommand{\squishend}{
  \end{list}  }
%


%-------------------------------------
\newcounter{sectrefs}
\setcounter{sectrefs}{0}
\setcounter{figure}{0}
\setcounter{chapter}{1}
\setcounter{section}{0}
\renewcommand{\thefigure}{\thesection.\arabic{figure}}

%
%--------------------------------------
%
%
%
%              start document 
%              ==========
%
%

\begin{document}


\LARGE
\hfill
\textbf{Chapter \thechapter}
\rule[0.15in]{450pt}{0.5mm}
\LARGE
\begin{flushright}
 \textbf{
{\fontfamily{phv}\selectfont Introduction}}
\end{flushright}
\normalsize

%
%------------------------------------------------------------------------------
\section{Overview}
%------------------------------------------------------------------------------
This manual describes the commands, workflows and the theoretical 
background material 
necessary to use the WARP3D finite element code. WARP3D 
is under continuing development as a research and production
code for the solution of large-scale, 3-D \ti{solid} models 
subjected to static and dynamic loads.  Many specific features in the code 
are oriented toward the investigation of ductile fatigue and
fracture in metals.

WARP3D is a complete, ready-to-run program to solve 3D finite element models
for mechanical and thermal loading. The code is extensible directly through
selected \ti{user} routines. The entire source code and build scripts are provided in
the distribution -- knowledgeable users have also extended the program by modifying the
source code. However, the majority of WARP3D users apply the as-distributed executables
for Windows, Linux and OS X to analyze models.

An \ti{implicit} framework is employed to solve the global nonlinear
equations of nodal equilibrium with an incremental-iterative 
approach. By default, every analysis is WARP3D is nonlinear and time dependent.
Linear analyses are accomplished by (1) making the material properties
linear-elastic (\eg a large yield stress), (2) ignoring large displacement
effects and (3) defining no contact surfaces. Quasi-static analyses are accomplished
by setting values of material mass density to zero and/or setting the time step
to a large number.  Time-dependent, quasi-static analyses, \eg creep simulations,
are achieved by setting the material mass density to zero and using time step sizes suitable
for the creep analysis.

The WARP3D architecture is designed throughout to execute in parallel
using shared-memory with threads (via OMP) and/or distributed
memory via the Message Passing Interface (MPI, Linux only). The code executes
equally well on laptop, multi-core computers running Windows, Mac
OS X and Linux through deskside and workgroup compute servers to
large supercomputers.

WARP3D is an \ti{analysis engine} -- the code depends on other programs
for mesh construction and graphical post-processing of computed results.
These include a wide variety of commercial, open-source and codes developed/distributed
by the national laboratories. WARP3D input (text) files have a remarkably
simple, easy-to-understand structure and are readily generated from model descriptions
produced by mesh generators. The widely used Patran 2.5 \ti{neutral} format is
supported by many mesh generation programs. The \ti{patwarp} program included in the
WARP3D distribution reads such neutral files and writes ready-to-run WARP3D 
input files. Similarly, WARP3D writes computed results (displacements, strains,
stresses, etc.) in a number of formats for use by post-processing programs,
included user-developed, specialized codes written in Python, C++, etc.

\nid A simplified workflow would contain these tasks:
\small
\squishlist
\item	Construct the finite element model using a mesh building program. Essential features
are nodal coordinates, element connectivities to nodes (\ti{incidences} in WARP3D
terminology) and possibly nodal loads/temperatures, element loads
(\eg surface pressures), displacement boundary conditions (\ti{constraints}
in WARP3D terminology), and other features. Examples: CUBIT, MSC-Patran,
FEACRACK, among many others.
\item	 Convert the output produced by the mesh building program into a very basic
WARP3D input file (a text file). In many cases the nodal coordinates and incidences
are located in separate files simply for convenience.
\item Use a text editor to complete all information in the WARP3D input file. For example,
definitions of the nonlinear materials, element properties (\eg integration order),
simple constraints (\eg nodes on a symmetry plane), construction of the nonlinear loading
(time) steps, nonlinear solution parameters (\eg convergence tests/tolerances), compute,
output and save restart commands.
\item Run the WARP3D code. Most often this is as simple as the following
command given in a window running a Bash shell (OS X, Linux) or the 
command prompt (Windows)
\small \begin{verbatim}
   %  warp3d 4 < in > out
\end{verbatim}
\nid where \% is the shell prompt, 4 is the number of threads for WARP3D to use, 
\ti{in} is the text file of WARP3D input and \ti{out} is the generated text output
file. Other types of output files may be produced during the course of execution.
\item Run post-processing software to display/examine the computed results.
Examples: ParaView, MSC-Patran, FEACRACK, among many others. Use a text editor
to examine contents of the file \ti{out} for error/warning messages and conventional
printed values of results.
\squishend \normalsize

The next section describes a few example analyses with model of
increasing complexity to aid in getting started with WARP3D.

\nid \textbf{\ti{Physical Units}}

WARP3D is unaware of physical units for quantities defined in the model
definition.  Users are responsible to adopt a consistent set of 
units for various quantities.
\end{document}



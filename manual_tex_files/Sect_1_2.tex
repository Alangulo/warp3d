

\documentclass[11pt]{report}
\usepackage{geometry} 
\geometry{letterpaper}

%---------------------------------------------
\setlength{\textheight}{630pt}
\setlength{\textwidth}{450pt}
\setlength{\oddsidemargin}{14pt}
\setlength{\parskip}{1ex plus 0.5ex minus 0.2ex} 


%----------------------------------------
\usepackage{amsmath} 
\usepackage{layout}
\usepackage{color}
\usepackage{array}
\usepackage{rotating}

%----------------------------------------------
\usepackage{fancyhdr} \pagestyle{fancy}
\setlength\headheight{15pt}
\lhead{\small{User's Guide - \textit{WARP3D}}}
\rhead{\small{Illustrative Problems}}
\fancyfoot[L] {\small{\textit{Chapter {\thechapter}}\ \   (Updated: 9-5-2014)}}
\fancyfoot[C] {\small{\thesection-\thepage}}
\fancyfoot[R] {\small{\textit{Illustrative Problems}}}

%---------------------------------------------------
\usepackage{graphicx}
\usepackage[labelformat=empty]{caption}
\numberwithin{equation}{section}
\usepackage{bm}

%---------------------------------------------
%     --- make section headers in helvetica ---
%
\usepackage{sectsty} 
\usepackage{xspace}
\allsectionsfont{\sffamily} 
\sectionfont{\large}
\usepackage[small,compact]{titlesec} % reduce white space around sections
%---------------------------------------------->
%
%
%   which fonts system for text and equations. with all commented,
%   the default LaTex CM fonts are used
%
%
\frenchspacing
%\usepackage{pxfonts}  % Palatino text 
%\usepackage{mathpazo} % Palatino text
%\usepackage{txfonts}


%---------  local commands ---------------------


\newcommand{\bmf } {\boldsymbol }  %bold math symbol
\newcommand{\degree } {\mathrm{o} }  %bold math symbol
\newcommand{\ttt} {\texttt}  %typewriter text
\newcommand{\bsf } [1]{\textrm{\textit{#1}}\xspace}
\mathchardef\mhyphen="2D
\newcommand{\ul} {\underline}
\newcommand{\hv} {\mathsf}   %helvetica text inside an equation
\newcommand{\eg}{\emph{e.g.},\xspace}
\newcommand{\ie}{\emph{i.e.},\xspace}
\newcommand{\ti}{\emph}
\newcommand{\vepsilon}{\varepsilon}
\newcommand{\etal}{\ti{et al.}\xspace}
\newcommand{\Fig}{{Fig.}\xspace}
\newcommand{\nid}{\noindent}
\newcommand{\noi}{\noindent}
\newcommand{\vareps}{\varepsilon}
\newcommand{\cauchyu}{\boldsymbol{\sigma}_{u}}
\newcommand{\cauchy}{\boldsymbol{\sigma}}

\newenvironment{offsetpar}[1]%
{\begin{list}{}%
         {\setlength{\leftmargin}{#1}}%
         \item[]%
}
{\end{list}}

%
%
%        optional definition for bullet lists which
%        reduces white space.
%
\newcommand{\squishlist}{
 \begin{list}{$\bullet$}
  { \setlength{\itemsep}{0pt}
     \setlength{\parsep}{3pt}
     \setlength{\topsep}{3pt}
     \setlength{\partopsep}{0pt}
     \setlength{\leftmargin}{1.5em}
     \setlength{\labelwidth}{1em}
     \setlength{\labelsep}{0.5em} } }

\newcommand{\squishlisttwo}{
 \begin{list}{$\bullet$}
  { \setlength{\itemsep}{0pt}
     \setlength{\parsep}{0pt}
    \setlength{\topsep}{0pt}
    \setlength{\partopsep}{0pt}
    \setlength{\leftmargin}{2em}
    \setlength{\labelwidth}{1.5em}
    \setlength{\labelsep}{0.5em} } }

\newcommand{\squishend}{
  \end{list}  }
%
\newcounter{Lcount}
\newcommand{\squishnum}{
\begin{list}{\arabic{Lcount}. }
{ \usecounter{Lcount}
\setlength{\itemsep}{0pt}
\setlength{\parsep}{3pt}
\setlength{\topsep}{3pt}
\setlength{\partopsep}{0pt}
\setlength{\leftmargin}{1in}
\setlength{\labelwidth}{1em}
\setlength{\labelsep}{0.5em} } }

\makeatletter
\renewcommand*\env@matrix[1][\arraystretch]{%
  \edef\arraystretch{#1}%
  \hskip -\arraycolsep
  \let\@ifnextchar\new@ifnextchar
  \array{*\c@MaxMatrixCols c}}
\makeatother


%-------------------------------------
\newcounter{sectrefs}
\setcounter{sectrefs}{0}
\setcounter{chapter}{1}
\setcounter{section}{1}
\setcounter{figure}{0}
\renewcommand{\thefigure}{\thesection.\arabic{figure}}
%
%--------------------------------------




\begin{document}

\section{Illustrative Problems}
%\layout
This section describes a number of example problems to illustrate the
use of WARP3D. Each example incorporates new modeling
capabilities, features and input commands. Brief explanations are included as 
needed -- most of 
the input commands are self-explanatory. Many example analyses with input
files ready-to-run are included in the WARP3D distribution.

\ul{\ti{The best way to learn about WARP3D is to use it ! Let's get started.}}

\subsection{Two element model}
Figure {\ref{fig:2elementinput}  shows a very simple 2 element mesh to 
illustrate the overall concept of
defining a model to WARP3D, requesting a solution followed by
output requests. Many features and 
capabilities of WARP3D are not shown for simplicity in this first
example.

%
\begin{figure}
\begin{center}
\includegraphics[trim=0.0in 0.3in 0.5in 0.7in, clip=true,scale=0.8,angle=0]{figures_example_1/input.pdf} 
\caption{{\small Fig. \thefigure\ 2 element example problem to illustrate key concepts in
model definition/solution with WARP3D. All input shown. The nearly column aligned
data is only for clarity in reading.}
\label{fig:2elementinput}}
%
\end{center}
\end{figure}
%

Two 8-node isoparametric
elements are stacked vertically. Nodes on the $y=0$ plane are fixed. The model loading
comprises three components to be each applied incrementally and sequentially over
three \ti{steps}: (1) a uniform pressure over the
top surface of element 2, (2) forces applied to nodes on the top surface,
and (3) a linear temperature gradient  from the top surface to the base. The 
material is linear-elastic and isotropic. The linear strain-displacement relations
are used and the analysis is quasi-static. The
material mass density is set to zero. The time step may then be set to any
convenient value since there are no inertia effects and the material 
behavior is time invariant. The entire input file is listed on the figure. 

\nid The essential input commands and the usual ordering are:
\small
\squishlist
\item	structure name -- used on headers for output and to include in the default
names of various files  generated during execution.
\item definition of  \ti{materials} for association with elements in the model. 
Materials provide linear-elastic properties, material density, 
nonlinear properties and a \ti{type} of constitutive model.
\item number of nodes and elements in model -- nodes and elements
must be numbered sequentially starting at 1.
\item	$X\mbox{-}Y\mbox{-}Z$ coordinates for all nodes in the model 
global coordinate system.
\item	connectivity of elements nodes to structure nodes -- termed 
\ti{incidences}.
\item assignment of  elements to \ti{blocks} for analysis. The \ttt{automatic} option is
 sufficient for all analyses run in parallel using only threads.
\item constraints are the displacement boundary conditions.
\item three  loading \ti{patterns} for the model. Loading patterns consist of nodal forces; 
element body forces, face tractions, face pressures which are converted to 
equivalent nodal forces; nodal and element temperature changes relative (here)
to a zero reference state.
\item a nonlinear/dynamic loading which defines the increment of load to 
be applied during each load(time) step. Loading increments for a step are 
defined using the loading patterns with multipliers.
\item parameters to control the nonlinear/dynamic solution process, \eg 
the time increment, the type of equation solver 
(direct, iterative), maximum number of Newton iterations, 
convergence tolerances, etc. (the minimal options are shown here)
\item a request to compute displacements for a list of load steps (listing step 3 here
means first solve steps 1 and 2). At the end of step 3, the total loading
applied to the model 
comprises $10.0\ \times$ \ttt{top\_pressure} $\ -\ 2.0\ \times$ \ttt{top\_torque} 
$ +\ 500.0\ \times $\ttt{temp\_grad}.
\item requests to output computed node and element results. Results 
for use by humans are directed (here) to the current output device with 
appropriate pagination, headers, labels, etc.
\item save command to create a file enabling a restart analysis with the
solution for step 3 as the stating point. 
\item	 stop terminates program execution.
\squishend
\normalsize

WARP3D commands are format free and may begin anywhere within the first 20 spaces 
on the line. Only the first 80 characters on a line are processed.
One or more blanks separate data items. A \ti{C}, \ti{c}, \# or ! in column 1
followed by a space denotes 
a comment line and is ignored by the input translator.  A \$ may be used to terminate a line
which makes the remainder of a line a comment.
A comma (,) at the end of a line indicates that the input for that command continues 
on the next line. Blank lines are ignored by the input translator. 
Decimal points are optional and may be omitted if not needed to specify the fractional part of a number. 
A \ttt{<list>} sometimes termed \ttt{<integer\ list>}  provides a shorthand method to specify 
integer values (usually node or element numbers). Examples are 1, 3, 6, 10, 12 and 1-10 by 2 
(which implies 1, 3, 5, 7, 9) and combinations, \eg 10-1000 by 2 2000 2002 2004 6000-9000 by 10.

The apparent column alignment of input commands listed in \Fig \ref{fig:2elementinput}
is simply for readability.

The input commands comprise keywords and data values. The ordering of commands is quite
flexible, \eg \ttt{coordinates} may be given before the \ttt{element} definitions but not before the
\ttt{number} command; \ttt{constraints} may appear anywhere after the \ttt{number} command; the three
loading patterns (\ttt{top\_pressure}, \ttt{top\_torque}, \ttt{temp\_grad})  may be given 
in any order; the list of \ttt{nonlinear analysis parameters} may be given 
in any order, etc. In the \ttt{constraints} command for example, the \ttt{u  v w}
values may be listed in any order.
Use of the underscore (\_) character often improves readability and is treated as an ordinary
character.

Textual data values in this example: \ttt{bar}, \ttt{metal}, \ttt{top\_pressure}, \ttt{test}, etc. are termed a
\ttt{<label>} in the WARP3D input system. Labels must begin with an alphabetic character and not contain special
characters (. ; ! \$ \# \* \{ \} etc.) The \ti{save} command in this example illustrates use of 
a \ttt{<string>} data type to include special characters where desired for convenience (strings are
enclosed in single or double quotes).

To continue with this example, let the input definition shown in \Fig \ref{fig:2elementinput}
be defined in the file name \ttt{warp3d.inp} -- there is \ti{no} 
significance attached to the \ttt{.inp} extension of the file name; it is simply 
for convenience when examining lists
of files in a directory.

The analysis is executed in a display window (\eg Bash shell on Linux and OS X, command prompt shell in
Windows or a Bash shell in Windows under the Cygwin system)
%
\begin{verbatim}
   % warp3d 1 < warp3d.inp  > out
\end{verbatim}  
%
\nid where \% is the shell prompt character. \ttt{warp3d} is the name of a 
shell script that checks/set environment variables,
sets the location of the WARP3D executable program, etc. Example versions of the \ttt{warp3d}
shell script are included in the WARP3D distribution. Here, 1 thread is used for execution. Usual
I/O re-direction sets standard input and output. With this command, WARP3D runs
interactively; the shell prompt returns once WARP3D completes execution. Append an
\& at the end of the line to have the job run in background independent of the display window.

The contents of file \ttt{out} are shown in \Fig \ref{fig:2elementoutput1} and 
continued in \Fig\ref{fig:2elementoutput2}. Input commands are mixed with messages
from WARP3D together with the requested output values. Some WARP3D messages are 
deleted here to reduce space requirements in the figures. The information for the solution of
each load (time) step is shown as the analysis progresses. Here the solution is linear-elastic such
that no (Newton) equilibrium iterations are necessary  -- the residual nodal forces after iteration 1
of each step are zero.

%
\begin{sidewaysfigure}
\begin{center}
\includegraphics[trim=0.0in 0.3in 0.5in 0.7in, clip=true,scale=0.8,angle=0]{figures_example_1/output_pg1.pdf} 
\caption{{\small Fig. \thefigure\ Output for 2 element example problem. 
Some non-essential lines are deleted and
replaced with ... for clarity}
\label{fig:2elementoutput1}}
%
\end{center}
\end{sidewaysfigure}
%
%
\begin{sidewaysfigure}
\begin{center}
\includegraphics[trim=0.0in 0.3in 0.5in 0.7in, clip=true,scale=0.8,angle=0]{figures_example_1/output_pg2.pdf} 
\caption{{\small Fig. \thefigure\ Continuation of output
 for 2 element example problem. Some non-essential lines are deleted and
replaced with ... for clarity}
\label{fig:2elementoutput2}}
%
\end{center}
\end{sidewaysfigure}
%


\subsubsection{Additional discussion of input}


\nid The directory \ttt{manual\_examples\_chpt\_1/example\_1} in the WARP3D
distribution contains the input file for this example (warp3d.inp).

\nid \ti{\ul{Materials} }

\nid  Materials are defined to be associated with finite elements in the model. A material definition
comprises the name of the material model and values of properties for that model. Multiple
materials may be necessary for a complex model representing an object
fabricated from different physical materials. In this first example, the material model \ttt{bilinear} available in
WARP3D is employed. This model uses mises plasticity with mixed isotropic-kinematic
 hardening (see Section 3.5 for details).  Properties for the \ttt{bilinear} model include
 \ttt{e, nu, yld\_pt, alpha, rho} specified here. Properties may be given in any order.
 For this first example, the yield stress is set to
 a large value to enforce linear-elastic behavior.
 
WARP3D has a number of built-in material models like \ttt{bilinear} that may be associated with any
of the solid elements. Several \ttt{cohesive} materials may be associated with \ttt{interface}
elements as one approach to model progressive fracturing processes. 

Plasticity type models that require the user to define uniaxial stress-strain curves utilize the
\ttt{stress-strain curve} command to define points along a linear, segmental curve to include temperature
or strain-rate dependence as needed. Input commands for the material then reference a defined \ttt{stress-strain}
curve.

For materials to be associated with solid elements WARP3D supports a \ti{user-defined} material model or
\ti{umat} (see Appendix I).

\nid \ti{\ul{Elements} } 

\nid Each finite element in the model must be one of the available
element types as described in Chapter 3. These include hex
elements with a variable number of nodes and names \eg \ttt{l3disop, q3disop, ts14isop, ...} and tet elements
(\ttt{tet4, tet10}). Interface elements are named \ttt{inter\_8, trint6, trint12, ...}.

Following the element type in the input is the keyword \ttt{linear} or \ttt{nonlinear}, which specifies the
linear or nonlinear strain-displacement formulation to be used. In WARP3D, \ttt{nonlinear} is analogous to 
NLGEOM in Abaqus.

The strain-displacement formulation is followed by the keyword \ttt{material} then the name
of the previously defined material. 

The element definition is completed with keywords for the element properties and the values.
Here the element properties used are: \ttt{order}, \ttt{bbar} and \ttt{short}. The quantities \ttt{bbar} and
\ttt{short} are \ti{logical} element properties -- appearance of  the keyword makes the values ``true". 
\ttt{bbar} invokes the so-called $\overline B$ element formulation to reduce volumetric locking
behavior (\ttt{bbar} has a default value of ``true"). \ttt{no\_bbar} disables this formulation. 
\ttt{short} requests a reduced set of output values for strains-stresses. See Section 3.1.2
for a full description of properties for the \ttt{l3disop} element. The element property

Element input must appear before \ttt{incidences}.

\nid \ti{\ul{Coordinates} }

\nid The default ordering is \ttt{ x,  y,  z}. Preface a coordinate value with labels \ttt{x y} or \ttt{z} to override
the default ordering. Nodes may appear in any order. Most often coordinates are placed into
a separate file (\eg \ttt{coordinates.inp}) for convenience and then referenced with a 
command

\small\begin{verbatim}
  *input from file "coordinates.inp"
\end{verbatim}\normalsize
\nid
where the typical contents of a coordinates input file has the form
\small\begin{verbatim}
    *echo off
    coordinates
      1    ..   .. ...
      2  .. ..  
      ...
      ...
    *echo on
\end{verbatim}\normalsize
\nid This is the first example use of the special * (star) commands in WARP3D. A number of such commands
are available to invoke useful services (see Section 2.15). In this case, the input translators switch
to read/process commands from the file \ttt{coordinates.inp} until an end-of-file condition is reached when
the translators simply return to reading the previous file. Such \ttt{*input from ... } commands can be
nested as proves convenient. The \ttt{*echo off/on} commands suppress the writing of large
numbers of coordinate definition lines to the output file.

\nid \ti{\ul{Incidences} }

\nid Chapter 3 describes the ordering of nodes on each available element to be defined in the 
incidences data. Elements maybe entered in any order. Use a comma to continue long
lines of node numbers over multiple physical lines. Incidence data, like coordinates data,
are most often placed in a separate file (\eg \ttt{incidences.inp}) 
for convenience and referenced with a command  \ttt{ *input from file "incidences.inp"}

\nid \ti{\ul{Blocking} }

\nid To improve parallel performance, WARP3D processes elements in \ti{blocks} usually with 64, 128, or
256 elements per block. Elements in a block are numbered sequentially, must be the same type
(\eg \ti{l3disop} as in this example), use the same material (\ti{metal} here), use the same
numerical integration order, same small or large displacement formulation, etc.
Section 2.6 provides details; Chapter 7 further explains concepts of parallel
execution in WARP3D and the usefulness of blocking. For most analyses, the automatic blocking
command used in this example suffices. The block size may be specified with the 
option \ttt{size = <value>} appended after the keyword \ttt{automatic}. The assignment of elements to 
blocks does not renumber the elements -- block sizes are adjusted to accommodate changes in element types, 
different materials, etc.

For parallel execution using MPI+threads, meshes are first partitioned into domains and then domains into blocks.
The \ttt{blocking} command is then expanded to include the domain numbers.

\nid \ti{\ul{Constraints} }

\nid The displacement boundary conditions here enforce simple zero changes at the prescribed nodes.
Non-zero displacements are imposed with equal simplicity.  Section 2.7
describes the \ttt{plane} command to set displacement values over all nodes that lie  on a plane, \eg to impose
symmetry conditions. Constraints may also be imposed in non-global coordinate systems. Multi-point
constraints may be defined to enforce a linear relationship between displacement values at multiple nodes. 
Transitions from coarse to refined mesh densities and between different element types may be accomplished
with the \ti{mesh tying} capabilities of the constraints system. Consistent with the incremental-iterative, implicit
solution in WARP3D, the defined constraints are treated as \ti{incremental} values imposed over each
load(time) step. The notion of \ti{incremental} constraints is a key concept in WARP3D. Example: the \ttt{constraints}
input has \ttt{5 u 0.1} -- by default after 10 load steps, node 5 will have a \ttt{u} displacement = 1.0. Whenever the \ttt{constraints}
command is given, all previously defined absolute and multipoint constraints are deleted. New input values
define the incremental displacement values for the next load(time) step. 

\nid \ti{\ul{Loading patterns} }

\nid The notion of loading patterns evolves from the structural engineering background of the WARP3D
developers where structural models are often analyzed for combinations of simple (pattern) loadings
such as dead, live, wind, snow, earthquake, thermal, etc.  A loading pattern consists of some
mixture of applied nodal forces, applied nodal temperatures and distributed loads applied to element
surfaces, and body forces from elements. Any number of such loading patterns may be defined. Note that
non-zero displacement loading of the model takes place through the constraints. Most often, pattern loads
are defined such that the maximum intensity in the definition has a unit value -- unit pressure on element surfaces,
temperature fields with maximum temperature value of unity, etc.

\nid \ti{\ul{Nonlinear/dynamic loading} }

\nid There exists one loading condition declared with the type \ttt{nonlinear} or \ttt{dynamic} -- these are synonyms.
The increment of loading imposed on the model over each load(time) step is then defined by a combination of
loading patterns and the constraints. The constraints are \ti{automatically} applied with a 1.0 multiplier
if not mentioned in the incremental definition for a load(time) step.  A more complex command than the simple one
used in this example is: \ttt{steps 20-2000 constraints 1.5 axial\_tension -3.2 unit\_pressure 10.5}. Non-zero 
entries in the constraints 
definitions at the point in the solution for a load(time) step are multiplied by 1.5 for application at the next
step. Load steps may be redefined any time up to the point when that step is analyzed.

\nid \ti{\ul{Nonlinear solution parameters} }
This example shows only the minimum needed options for the simplest solutions. Section 2.10 provides 
descriptions for all available solution parameters. Several of the most important include: choice of 
linear equation solver, adaptive solutions that subdivide a load step automatically into smaller increments
should the global Newton iterations fail, convergence criterion for the Newton iterations, levels of messages
output by WARP3D to track the nonlinear solutions, etc. As a research code, WARP3D provides ready access to
alter options controlling the incremental-iterative solution process. Default values often suffice
for new users.

\nid \ti{\ul{Compute commands} }

\nid The solution drivers in WARP3D automatically fill-in the solution of load (time)
steps as needed to satisfy the list of steps included in the command. Example: \ttt{compute displacements
loading test steps 500-1000}. Suppose the solution has advanced through step 295. The solution
for load steps 296-499 will be computed automatically in response to this \ttt{compute} 
command.

Compute commands are also used to request computation of domain 
and interaction  integral values (Chapter 5).

\nid \ti{\ul{Output commands} }

\nid Computed results most often are output to: (1) the text output file specified in the
shell command via standard out to invoke WARP3D, (2) another text file
defined via an \ttt{*output to file ...} command just prior to output commands, (3) flat files of nodal 
and element results in text or stream format, (4) formatted and/or unformatted files of
element and nodal results readable directly by Patran and other programs
that can read Patran-style result files (text and binary formats), (5) unformatted files of result
\ti{packets} to be processed by a user-written program. The \ti{flat} files have a very simple 2D array structure
with all values for a node or element contained on one line of the file. The text version of the
flat files can be imported directly into Excel. The text and stream versions are readily imported
into Python arrays. The \ti{packets} files are binary files (can be converted to text if requested)
with computed information stored in separate packets where a large number of packet types (\eg stresses for
an element, reactions for a node, a $J$-integral value, etc.)
are described in Appendix F.  Use of packet files does require investment by the user to tailor for their specific
purpose an example packet processing program included in the WARP3D distribution. This effort pays-off
in large parameter studies, for example, to extract results and to build input files for plotting programs.

Once WARP3D computes the solution for a load(time) step, output of all desired quantities
must be done before computations begin for the next step -- only results for the current load(time) step are
maintained. For analyses with a large number of load(time) steps and with frequent 
need for output, the large number of commands to compute/output becomes cumbersome. 
The feature \ttt{output commands
use file ...} resolves this issue. A separate file containing only output commands is created 
by the user, \eg \ttt{get\_results.inp}. Then the command
%

\small\begin{verbatim}
  output commands use file "get_results.inp" after steps 10-1000 by 10
  compute displacements loading test step 1000
\end{verbatim}  \normalsize
%
\nid drives the nonlinear solution through step 1000 with output generated following
steps 10, 20, 30, ...1000 using all the \ttt{output} commands contained in file
\ttt{get\_results.inp}.

\nid \ti{\ul{Analysis restart} }

\nid A file to enable restarting  the analysis may be written at completion of the
solution for a load(time) step. This is a binary (unformatted) sequential file that fully restores
the solution state to enable analysis continuation. Restart files are not transportable across computer
architectures (\eg a Windows restart file will not work on Linux).  This example illustrates use
of the \ttt{save} command to create a file for restart after load(time) step 3. These commands 
will restart this example and request some additional output, define new load steps 4-10,
request solution for steps 4-10 and create a new restart file

\small\begin{verbatim}
  restart from file "simple.save"
  output flat text displacements
  loading test 
    nonlinear
       steps 4-10 top_pressure 1.2
  compute displacements for loading test step 10
  save to file "simple_after_step_10.save"
  stop
\end{verbatim}  \normalsize


\nid \ti{\ul{Other commands not yet discussed} }

\nid These include stress-strain curves, contact surfaces, crack growth procedures, 
computation of domain and interaction integral values (\eg $J$)
at crack fronts, initial conditions for the model (displacements, velocities, temperatures), user-lists that
can greatly shorten input and additional * commands. 

\nid \ti{\ul{Notes for users familiar with Abaqus} }

\nid For implicit solutions, Abaqus employs the concept of \ti{steps} to define 
complex loading histories on
a model. A \ti{step} has a start and end (model) time, start and end 
loading/displacement/temperature etc.
levels.  A step is subdivided
into equal or variably sized \ti{increments} to accomplish the 
incremental-iterative nonlinear solution.
Each \ti{increment} has some number of global Newton \ti{iterations} to 
advance the solution from one increment to
the next. A key feature of the \ti{step} concept in Abaqus includes the 
ability to change analysis type, for example, static to
dynamic, static for pre-loading followed by a buckling or frequency 
analysis, etc.

WARP3D is a far less comprehensive program and employs only \ti{steps} 
and \ti{iterations}. A step in WARP3D combines
the role of steps and increments in Abaqus. Each step in WARP3D defines an increment 
of loading to be applied on the model with iterations to achieve the nonlinear solution for the step.

\subsection{Upsetting of a cylindrical disk}  

\nid The cylindrical metal disk shown in \Fig \ref{fig:upset_1} is crushed (upset) 
in the vertical direction for a significant
portion of the initial height through the prescribed 
downward motion of a rigid contact surface (\ie a punch). The axisymmetric geometry
is modeled here with a $1/8^{th}$ symmetric finite element mesh of 8-node solids. The top surface of the
disk is fixed to the punch. This problem has become a commonly used example to exercise finite-strain
plasticity implementations, element handling of incompressible deformations and simple rigid
body contact [\ref{R:NV1984}, \ref{R:SH1998}]. The element mesh, symmetry boundary conditions,  and
uniaxial (true) stress - (log) strain curve are shown in the figures. The yield stress is taken as 1.0
for simplicity with $E/\sigma_{ys}=1000$. The rigid contact plane
is assigned a velocity in the global $-y$ direction to crush the disk. At large deformations, outside elements
on the top row fold over and contact the rigid surface.  The disk height is reduced by 3
units in the symmetric
model (6 of the total 12 units of height for the full disk). 
The upsetting can continue for an additional
amount before several of the top row elements become inverted.
%
\begin{sidewaysfigure}
\begin{center}
\includegraphics[trim=1.5in 1.0in 1.0in 0.8in, clip=true,scale=0.9,angle=0]{figures_example_2/fig_disk_upset_1.pdf} 
\caption{{\small Fig. \thefigure\ Disk upsetting analysis. (a) overall disk geometry, (b) $1/8^{th}$ symmetric FE
model with 8-node brick elements, (c) uniaxial true stress -- log strain curve for material,
(d) imposed punch \ti{vs} time through rigid surface}
\label{fig:upset_1}}
%
\end{center}
\end{sidewaysfigure}
%

The WARP3D input file is shown in \Fig \ref{fig:upset2}. Nodal coordinates and incidences are located in separate files
(not shown). The \ttt{*input from file ..} command is used to have the input translators
read the those files.
%
\begin{sidewaysfigure}
\begin{center}
\includegraphics[trim=1.0in 1.0in 1.0in 0.8in, clip=true,scale=0.9,angle=0]{figures_example_2/fig_disk_upset_2.pdf} 
\caption{{\small Fig. \thefigure\ Disk upsetting analysis. (a) input file \ttt{warp3d.inp}
and (b) file of output commands \ttt{get\_output.inp} }
\label{fig:upset2}}
%
\end{center}
\end{sidewaysfigure}
%

\nid The directory \ttt{manual\_examples\_chpt\_1/example\_2} in the WARP3D
distribution contains the files for this example. 
For convenience, the input data is separated into files to define the mesh
and output commands. The main input file is named \ttt{warp3d.inp}. Files 
may have any desired/descriptive names; the \ttt{.inp} suffix is not required. Other files are 
included during execution using
\ttt{*input from file "<file name>"} commands in \ttt{warp3d.inp}. The \ttt{*input ...} may be nested
as convenient. Several Python scripts (*.py) are included as 
examples to extract results and make the plots.

The model definition begins with input of a stress-strain curve assigned the numerical id of 1 -- up to 
20 such curves may be defined (Section 2.2).  Strain-stress pairs are given with the first pair defining
the yield point. These are \ti{total} log strain and true (Cauchy) stress values. The last point provides 
a very small, positive slope through strain levels never reached. The \ttt{metal} material uses the
built-in \ttt{mises} plasticity model (Section 3.6). Note that the yield stress and modulus must
define a point consistent with the first point defined on the stress-strain curve.

Element definitions specify the \ttt{nonlinear} option for 
finite strains and finite rotations of material points.

The definition of key constraints (excluding the punch) is facilitated using the
\ttt{plane} command as shown.

The nonlinear loading is named \ttt{upset} and is defined to have a maximum of 100 load steps (increments)
with only \ttt{constraints} imposed. The actual loading will be enforced by
contact between the punch and the top row of elements. With all the defined, incremental constraints equal to
zero, the \ttt{1.0} multiplier here has no actual role. 

The \ttt{nonlinear solution parameters} contain additional entries for this severe loading.
Newton iteration parameters: the maximum iteration count is set at 8 given the rather large deformations
at each step and the convergence tolerance for the Euclidean norm of the  residual nodal forces is set to 0.01
(\ie 0.0001\%) -- tighter tolerances here lead to negligible changes in the computed results. The \ttt{adaptive}
solution strategy is turned on in anticipation of possible convergence issues for the large deformations and during
contact (element folding) at the punch.  Adaptive sub-incrementing is triggered if the residuals increase over 3
consecutive iterations or the maximum number of iterations is exceeded.

The \ttt{consistent q-matrix on} enhances significantly the global convergence
rate in this solution. The matrix $\mathbf{Q}$ impacts characteristics of 
the tangent stiffness for the material in finite strain
computations -- usually this option in set to \ttt{on} for finite strain 
simulations (see Section 1.9). With \ttt{off} this
solution requires twice as many global Newton iterations and adaptive 
reductions of some load step increments.

\ttt{trace solution on} causes WARP3D to emit messages to the current 
output device (file) indicating the solution 
status. The \ttt{batch} \ttt{messages on} option causes the file \ttt{disk.}\ttt{batch\_messages} 
to be created and updated during 
solution, where \ttt{disk} is the model name here. This file is opened, 
written to and closed each time a 
solution status message is output. This simplifies monitoring for long executions.

The input next defines a rigid contact surface.
Chapter 6 describes the rigid body modeling of contact available in WARP3D. Here a \ttt{plane} contact
surface (\ie the punch)
is defined with initial position matching the undeformed top surface of the disk.  The size of the plane
is sufficiently large to block penetration of nodes near the top of the disk as they undergo large displacements.
A velocity vector in \ti{global}
coordinates with unit magnitude
translates the rigid plane in the $-y$ direction to upset the disk. Note the coordinates of 3 points
to define the rectangular plane set the $+y$ direction of the plane point positively in the upsetting
direction. This type of orientation is required by the contact algorithms. A stiffness value defines the penalty number
used to prevent penetration through the rigid surface. Several values may be tried to understand
affects on the solution -- too small a value allows penetration while a much too large value
degrades conditioning of the assembled tangent stiffness. In this example, 
a multiple of $E A/ L\approx 80$ provides a 
starting point for values of the penalty stiffness with $E=1000$,
$A\approx 1$ and $L=12$.

Two user-defined lists (\ie sets) are next specified to simplify output requests. A variety of methods
are available to construct such lists as described in Section 2.16.

Output of results is desired at the completion of each load step. The \ttt{output commands use file ..}
feature makes this quite simple. Here the separate file \ttt{get\_ouput.inp} contains usual
\ttt{output} commands which will be executed at completion of each load (time) step (the \ttt{steps
all} option used here).

The remaining commands adjust the amount ($\Delta v$) that the rigid plane (punch) moves
downward  in each
load(time) step. One objective in this example is to demonstrate the robustness of the WARP3D finite-strain
plasticity computations -- thus a relatively few steps (40) are defined to upset the disk by approximately
50\% of the original height ($v_{total} =-3$ units in the symmetric model). The controlling
quantities are the plane velocity (-1.0) and the time increment for the step. Here the plane velocity remains
fixed at -1.0 (for convenience) with adjustments made to the time increment as the solution
progresses.
The first load step requires a comparatively small downward movement of the punch for
global convergence, $\Delta v =  10^{-5}$, to establish
contact with the disk and to initially load the disk in compression. The time increment 
is then adjusted to $\Delta t = 0.05$
($\Delta v =  -0.05$ units) for steps 2-5. The change is initiated by giving
the \ttt{nonlinear analysis parameters} command followed by the new
time increment -- all other analysis parameters remain unchanged.
Thereafter values of $\Delta t = 0.08$
($\Delta v =  -0.08$) are used for a total of 40 time steps at which the punch has moved
downward by 3.0 units and the model simulation time is 3.0 as well (from the selection of
a unit velocity of the contact surface). 

The selection of 40 steps satisfies two requirements: (1) the comparison solution 
[\ref{R:NV1984}] used 44 steps to reach 
$\Delta v_{total}=-2.64$ in the FE model, and (2) a plot showing details of the punch 
force \ti{vs} punch displacement is also
desired for comparison. The WARP3D solution algorithms will converge using
fewer than 40 load(time) steps but physical features of the
response maybe lost in the truncated loading history.

The total number of Newton iterations is 151 for an average 3.8 iterations per step. 
Decreasing the convergence tolerance from 0.01 to 0.001 increases the number
of Newton iterations to 243 and triggers the adaptive sub-stepping at load(time) step 24.
The punch force at step 40 decreases by 0.004\%.

Figure \ref{fig:upset3} shows the deformed mesh to true scale and a plot of the punch force
\ti{vs}
punch displacement. The results agree closely with those of the reference solution.

%
\begin{sidewaysfigure}
\begin{center}
\includegraphics[trim=0.0in 0.5in 1.0in 0.7in, clip=true,scale=0.9,angle=0]{figures_example_2/fig_disk_upset_3.pdf} 
\caption{{\small Fig. \thefigure\ Disk upsetting analysis. (a) comparison of predicted
punch force with punch displacement; WARP3D and reference solution, (b) deformed mesh
at step 40 to true scale and distribution of mises equivalent stress }
\label{fig:upset3}}
%
\end{center}
\end{sidewaysfigure}
%

\subsection{Linear-Elastic Fracture: Crack-Tip Parameters}  

\nid Figure \ref{fig:rpvlinear1}a shows a cylindrical vessel with a bolted header/dome
of the type found in pressurized water
reactors for commercial power generation. The thick wall is fabricated
from a ferritic steel with a thin, inner cladding layer of stainless steel to reduce
corrosion. Figure  \ref{fig:rpvlinear1}b shows a simplified, half-symmetric 3D FE model of the 
vessel (w/o the header) often used as a starting point for the analyses of axial and circumferential cracks, both 
internal and surface breaking. 

To simplify further for purposes of illustrating various features for fracture
analysis in WARP3D, the model contains a shallow, completely circumferential crack of uniform depth
that penetrates through the cladding and into the ferritic base material. Further, the crack is located 
remote from any penetrations of the vessel and the
joint with the header. With 
(1) internal pressure loading and (2) temperature gradients that vary only with position
($r_g$) everywhere through the wall, the FE model becomes axisymmetric and could be modeled with
2D, axisymmetric elements. 

Here we construct a 3D model using three layers of identical 8-node elements in the global hoop
($\theta_g$) direction to illustrate various computational capabilities. A model with one such layer
of elements would suffice equally as well for these idealized conditions. The top half of the model is defined as
shown in Fig. \ref{fig:rpvlinear2}a with $Y_g=0$ as a symmetry plane. Similarly, $Z_g=0$ (\ie $\theta_g=0$) 
is a symmetry 
plane and non-global constraints must be defined to enforce zero displacement normal to the 
$\theta_g=3^{\degree}$ plane of nodes. This definition of positive $\theta_g$ is taken for convenience.
The mesh has
a small hole at the top of the dome with radius 193 mm ($\phi_g=5^{\degree}$) to 
remove collapsed 3D elements  on the
generator axis. This reduces the axial force from pressure by less than 0.5\%. The cylinder height at the juncture
with the  (hemispherical)
header $=9.2 \times t_w$ such that the bending moment generated by the 
header-cylinder radius mismatch decays to a vanishingly small value on the $Y_g=0$
plane. The choice of 9.2 is one of convenience in meshing operations; a factor $> 7-8$ should
be sufficient (see [\ref{R:F1973}]). 
 The use of tied contact
(Section 2.7.6) simplifies transition to a coarser mesh as indicated in Fig. \ref{fig:rpvlinear2}b.
Figure \ref{fig:rpvlinear3} shows the crack front modeled with a region of wedge 
shaped, focused elements having a small 
radius to support finite-strain nonlinear analyses if needed. 


%
\begin{sidewaysfigure}
\begin{center}
\includegraphics[trim=0.0in 2.0in 1.5in 0.1in, clip=true,scale=0.9,angle=0]
{figures_example_3/figure_1_concept} 
\caption{{\small Fig. \thefigure\ Reactor pressure vessel. (a) much simplified 
illustration, (b) half-symmetric, 3D model without header/dome 
often a starting point for more detailed analyses. Current example
considers a circumferential crack around the complete inside at a location in the so-called beltline
region remote from any geometric discontinuities.}
\label{fig:rpvlinear1}}
%
\end{center}
\end{sidewaysfigure}
%



%
\begin{sidewaysfigure}
\begin{center}
\includegraphics[trim=0.0in 0.5in 0.4in 0.0in, clip=true,scale=0.8,angle=0]
{figures_example_3/figure_2_mesh} 
\caption{{\small Fig. \thefigure\ (a) 3D finite element model of $3^\degree$ segment
for axisymmetric analysis. (b) closer view
near crack front showing mesh transition using tied contact. (c) near front 
region on $\theta_g = 3^\degree$ plane showing internal pressure acting on crack
face. Total crack depth 10 mm ($a/t_w=0.044$). }
\label{fig:rpvlinear2}}
%
\end{center}
\end{sidewaysfigure}
%

The axisymmetric, displacement boundary conditions require $w=0$ at all nodes
on the $Z_g=0$ plane and a zero displacement
normal to the $\theta_g=3^\degree$ plane. To enforce the second, non-global constraint,  
a local coordinate system ($X_\ell, Y_\ell, Z_\ell$)
is defined at all these nodes such that $Z_\ell$ is
normal to the plane, $X_\ell$ is aligned with $r_g$ and $Y_\ell = Y_g$, \ie 
$Z_\ell= X_\ell \otimes Y_\ell$.
The needed axisymmetric constraint on these nodes is then $w=0$. Printed output at all model
nodes is in the global coordinates.

%
\begin{sidewaysfigure}
\begin{center}
\includegraphics[trim=0.0in 1.5in 2.5in 0.1in, clip=true,scale=0.8,angle=0]
{figures_example_3/figure_3_near_tip_mesh} 
\caption{{\small Fig. \thefigure\ View very near the crack front on $\theta_g=3^\degree$ plane. 
Small radius of crack front is visible.  The $v$ displacements at node A indicated on 
$\theta_g=0, 3^\degree$ planes can be used to check domain integral derived values of $K_I$. Element/node
stresses, $\sigma_{xx}$, along the crack face on  $\theta_g=0^\degree$ provide an estimate for the
$T$-stress (\ie $T_{11}$). Nodes marked with yellow dots are 
used to define the first domain at the crack front. Elements incident on these nodes
are in the 1st domain. Domains 2, 3, ... are generated automatically using just topology
information. See also Section 4.3.4 for definition of crack-tip nodes when
a non-zero initial radius is used.  }
\label{fig:rpvlinear3}}
%
\end{center}
\end{sidewaysfigure}
%

Two linear-elastic materials (clad and base) are defined for the analysis. These 
materials have different elastic modulus and thermal expansion coefficients. The subsequent
nonlinear analyses include temperature-dependent properties for these materials.

The vessel operates at steady conditions of 288C (550F)
uniform temperature over the wall and 15.2 MPa internal pressure (2.2 ksi) including pressure acting on
the crack faces. Of primary interest here 
is the situation of very rapid cool down of the inner wall. To approximate
such severe conditions, a quadratic temperature variation is defined over the wall --
the outside surface (insulated) remains at 288C with the inner wall dropping to 100C while the internal
pressure remains unchanged. \ti{Actual} pressure loading and wall 
temperatures are strongly time dependent for various normal and abnormal
operating scenarios, 
and involve complex heat exchange processes on the cladding surface. None of these
complexities are included here in this simplified, illustrative problem. The 
stress-intensity factor generated by the
thermal gradient is several times larger than caused by the
internal pressure acting alone at the steady-state temperature. 
Contraction of the clad and base material at the inner surface from the
imposed temperature gradient exert significant tensile, axial stresses on the
shallow crack front. The larger CTE of the clad material increases this effect
a small amount in the example material properties adopted here.
The linear analysis  for such a severe temperature gradient predicts stresses
larger than the yield level of the lower-strength clad material. Once yielded, the
clad material becomes limited in the intensity of axial
stresses exerted on the crack front. 

$J$-integral values are computed with the domain integration 
procedures/commands described in Chapter 4. Domains are defined at some 
distance from the front to insure path independence -- although this is not
necessary for the linear analyses where $J$-values are unchanged from domain (ring) 2 
at the front outwards with increasing $r_c$ -- see Fig. \ref{fig:rpvlinear3}}.
Stress-intensity ($K_I$) values are obtained
from these $J$-values using the plane-strain conversion $K_I=\sqrt{E J/(1-\nu^2)}$.


Table 1.3.1 summarizes key geometry, material and FE mesh quantities of interest for the model.


Physical units employed in the model definition are: mm, MPa, $^\degree \mathrm{C}$. Computed 
displacements are thus in mm, stresses in MPa and reaction forces at nodes in N. Computed/printed
$J$-integral values thus have units of  $\mathrm{kJ/m^2}$, where
$\mathrm{175\, kJ/m^2=1\, kip\mhyphen in/in^2}$. Computed/printed $K_I$ values $\div\,\sqrt{10^3}$
have customary units $\mathrm{MPa\sqrt{m}}$, where $1\, \mathrm{MPa\sqrt{m}}= 
\mathrm{0.91\, ksi\sqrt{in}}$.

For convenience, the large volume of input data is separated into files to define the mesh, 
constraints, loading and domain definitions. The directory \ttt{manual\_examples\_chpt\_1} in the WARP3D
distribution contains all files for this linear analysis and the subsequent nonlinear analyses of this
model. Subdirectory \ttt{example\_3\_linear} contains the
files for this linear analysis. The main input file is named \ttt{warp3d.inp}. Files 
may have any desired/descriptive names; the \ttt{.inp} suffix is not required. Other files are 
included during execution using
\ttt{*input from file "<file name>"} commands in \ttt{warp3d.inp}. The \ttt{*input ...} may be nested
as convenient.
The following lists the files created for the analysis.

\squishlist
\item \ti{mesh\_coords.inp}. ($X_g, Y_g, Z_g$) coordinates for all nodes.
\item \ti{mesh\_incid.inp}. Connectivities of elements to nodes.
\item \ti{tied\_contact.inp}. Definition of two \ti{master} surfaces and 
two corresponding \ti{slave} surfaces for this model. Master 
surfaces are on the coarse side of mesh transitions; slave surfaces on the more refined side. Surfaces
consist of element numbers and faces of those elements. Section 3.1 provides face information
for hex elements. A
\ti{tie mesh} command then lists each master surface and the connected slave surface; here there
are two entries in the lists.
\item \ti{unit\_internal\_pressure.inp}. Define a unit pressure over the inside
surface of the model. List has element numbers and loaded faces. Net axial force for the
$\theta_g=3^\degree$ model is 0.12804 MN (average axial stress 4.66 MPa).
\item \ti{unit\_crack\_face\_pressure.inp}. Define a unit pressure on each element over the crack
face. The crack-front length in the model is $s_c= 2 (R_i + a/2) \pi /120 = 116.5\ \mathrm{mm}$,
for a total crack face area $= 10 \times 116.5=1165\ \mathrm{mm}^2$, vertical force for 
unit pressure = 0.001165 MN.
\item \ti{unit\_temp\_gradient.inp}. Define a temperature at every node of the model representing a change,
$\Delta T(r_g)$ from the uniform, steady temperature of 288C. 
The largest change is -1C at all nodes on the inner surface and 
0C at all nodes on the outer surface. See equation in Table 1.3.1. 
A \ti{user\_routine} is often employed to
define complex spatial and temporal temperature variations  in these
type of models (see Section 2.8.2).
The temperature change acting alone is self-equilibrating and thus generates
no next axial force in the model.
\item \ti{domain\_define\_compute.inp}. Define the domains and request
$J$-integral and $I$-integral (interaction)  computation at the four corner
node locations along the crack front ($\theta=0, 1, 2, 3$-degrees). A list of nodes
at the finite-root radius  is specified to enable the automatic construction of domains 
at that location
and assignment of $q$-values at nodes (see Fig. \ref{fig:rpvlinear3}).
\squishend

Unit-value pressure and temperature loadings are defined simply for
convenience. Scaled values are specified in the load steps to
impose the actual loading intensity.

\ttt{warp3d.inp} makes extensive use of the user-define list command to define/store
long lists of nodes and elements (see Section 2.16).  For simplicity,
the solution is computed separately in three analyses for each of the three loadings.
%
%=================================================

%\begin{tabular*}{0.75\textwidth}{ | c | c | c | r | }



\begin{table}[htb]	
\centering
{
\setlength{\extrarowheight}{2.5pt}
\small
\begin{tabular}{ | p{4in} | p{2in} | }
\hline
\textbf{Quantity} & \textbf{Values} \\
\hline \hline
Inner, outer radius & $R_i=2220$ mm, $R_o=$2445 mm \\ \hline
$R_i/R_o$ & 0.91 \\ \hline
Total thickness, clad and base metal thickness & $t_w=225$, $t_c=6$, $t_b=219$ mm  \\ \hline
Height of cylinder part of model&2070 mm $= 9.2 \times t_w$  \\ \hline
Hole radius at top of dome  & $R_{hole}=193$ mm  \\ \hline
Crack depth & 10 mm; $a/t_w=0.044$ \\ \hline
Crack-front initial radius & $\rho_0 = 0.00375$ mm \\ \hline
Steady-state internal pressure, uniform wall temperature  &  15.2 MPa, 288C  \\ \hline
Clad properties &   \\ \hline
\hspace{2em}$E$& 157,200 MPa\\ \hline
\hspace{2em}$\nu$ & 0.3 \\ \hline
\hspace{2em}CTE & $17 \times 10^{-6}/^\degree$C \\ \hline
Base properties &   \\ \hline
\hspace{2em}$E$&  193,000 MPa\\ \hline
\hspace{2em}$\nu$ & 0.3 \\ \hline
\hspace{2em}CTE & $14 \times 10^{-6}/^\degree$C \\ \hline
Nominal axial stress (15.2 MPa pressure) & 70.8 MPa \\ \hline 
Nominal hoop stress (15.2 MPa pressure) @ $R_i, R_o$ & 158, 143 MPa \\ \hline 
Temperature gradient over $t_w$; $-1^\degree$C over $R_i$, 0 over $R_o$& $\Delta T(r_g) = -\left [ (r_g-R_o)/t_w\right ]^2$  \\ \hline 
Nodes on $R_i, Y_g=0$. $v$ must be identical
  for symmetry. $\theta_g=0, 1, 2, 3^\degree$. & 2609, 5647, 11183, 16719  \\ \hline
Nodes to compute $K_I$ from $v$ @ $r_c=0.0474\ \mathrm{mm}$ on $\theta_c=\pi$.
    $\theta_g=0, 1, 2, 3^\degree$.  & 3730, 9926, 15462, 20998 \\ \hline
\end{tabular}}
\normalsize
	
%
\caption{Table 1.3.1 
Quantities for the linear-elastic vessel analysis.}
\label{table:linear-vessel}
\end{table}

\nid \ti{\ul{Nominal stresses in cylinder} }

\nid In the absence of a crack, the nominal hoop stress is given by
%
\begin{equation}\label{E:nomstress}
\sigma_\theta (r_g) = \frac{R_i^2 (r_g^2+R_i^2 )\, p }{r_g^2 (R_o^2-R_i^2)}\ .
\end{equation}
%
\nid For $p=$15.2 MPa, $\sigma_\theta (r_g=R_i) = 158$ MPa, $\sigma_\theta (r_g=R_i+a) = 157$ MPa,
and $\sigma_\theta (r_g=R_o) = 143$ MPa. 

For the internal pressure, the average axial stress is given by $15.2 \times 4.66=70.8$ MPa (see bullet four
above for unit pressure values).

\nid \ti{\ul{Computed reaction forces} }

\nid Comparison of computed reactions at nodes on $Y_g=0$ for the $\theta_g=3^\degree$
model with known applied forces provides a simple statics check. Loading is 15.2 MPa internal pressure
and $\Delta T=188^o$ on inside surface -- no change at outside surface.
\squishlist
\item \ti{Crack-face pressure}. Net applied axial force: 0.01771 MN; computed reaction 0.017699 MN. 
\item \ti{Internal pressure}. Net applied axial force: 1.9462 MN; computed reaction 1.9462 MN. 
\item $\Delta T$. Net applied axial force: 0.0; computed reaction $0.49 \times 10^{-8}$ MN. 
\squishend

\nid \ti{\ul{Computed $K_I$-values} }

\nid Stress-intensity factors are computed separately for each of the three loadings using three 
approaches: (1) from $J$-values determined using the domain integral method 
(convert $J$ to $K_I$ assuming plane-strain conditions), (2) directly from domain interaction integrals
(pressure loadings only), and (3) by matching of the $v$ displacements at one or more nodes
behind the crack front on $\theta_c=\pi$ to values for the 1st term of the asymptotic field.
The current implementation of the
domain interaction integrals does not support thermal loadings.

Table 1.3.1 lists the nodes used to compute $K_I$-values from the $v$ displacements using this
expression
%
\begin{equation}\label{E:KIfromv}
K_I(r_c,\theta_c=\pi) = \frac{2 \,G \,v(r_c)}{(\kappa + 1) \sqrt{\dfrac{r_c}{2\pi}}}
\end{equation}
%
\nid where $\kappa=3-4\,\nu$ for plane-strain and $G=0.5 \,E / (1+\nu)$.

Table 1.3.2 summarizes the $K_I$-values and also lists values for the thermal gradient loading
(1) with the clad material replaced by the base material, \ie homogenous material properties
over the model; and (2) with CTE = 0 for the clad to understand the relative
contribution of the clad and base thermal contraction on the
crack front. For a matching $\mathrm{CTE_{clad}=CTE_{base}}$,  $K_I$ 
decreases by about  6\%. With  $\mathrm{CTE_{clad}=0}$, $K_I$ decreases by more than
60\% indicating the large contribution of thermal contraction in the clad towards loading
the crack front.

For the internal pressure loading, a comparison can also be made to a short edge crack in a long 
rectangular plate under remote tension loading,
%
\begin{equation}\label{E:edgecrack}
K_I = 1.12 \,\sigma_{axial}\sqrt{\pi a} = 1.12 \times 70.8 \sqrt{\pi \times 0.010} = 14.0\ \mathrm{MPa\sqrt{m}}
\end{equation}
%
\nid  As  expected for this
vessel geometry with $a/t=0.044$, the short edge crack solution for the SE(T) 
approximates the vessel conditions quite well.

\nid \ti{\ul{Computed $T$-stress values} }

\nid The domain interaction integral method as implemented presently in WARP3D computes the $T$-stress 
only for mechanical loadings. Table 1.3.2 shows the $T$-stress values from the interaction
integral for the crack face loading and the internal pressure loading.

$T$-stress values may also be estimated  from the FE 
displacements/stresses in near-front elements (see [\ref{R:APS1998}]). Use of the directly computed
$\sigma_r$ on $\theta_c=\pi$ as the $T$-stress value requires exceptionally refined mesh resolution to obtain
values comparable in accuracy to those from the interaction integral approach.

The $T$-stress is thus estimated using (radial) nodal displacements
from the expression
%
\begin{equation}\label{E:Tstressdispl}
T= \sigma_r ( r_c\rightarrow 0, \theta_c=\pi) = \vareps_r E + \nu \sigma_\theta
\end{equation}
%
\nid where $\vareps_r$ is computed by a central difference approximation
using displacements at nodes on $\theta_c=\pi$. In [\ref{R:APS1998}], $T$-stress
values become relatively constant once nodes at distances $r_c/a\approx 0.04$ 
and larger are employed
in the above expression. For the present (axisymmetric) model, the challenge becomes 
setting the value of hoop stress --
in plane-strain models we have the simpler condition $T=E^\prime \vareps_{xx}$. The
hoop stress in the uncracked cylinder for internal pressure (15.2 MPa) at the 
crack front location is 157 MPa (see section above). The FE hoop stresses in elements
along $\theta_c=\pi$ behind the crack show a nearly uniform value of 150 MPa at
distances in the vicinity of $r_c/a = 0.04$ (the traction-free crack faces
must relax the hoop stress values near $\theta_c=\pm \pi$.  Using the above approach with
$\sigma_\theta=157$, the estimated $T$-stress from radial displacements is -42 MPa
compared to -53 MPa from the domain integral method. Using $\sigma_\theta=150$
the estimated $T$-stress becomes  -43 MPa.  The comparison is 
not nearly as good for the face loading where the values are also much smaller.
These $T$-stress values are small compared
to the base metal yield strength. 

$T$-stress estimates for the $\Delta T$ loading lie
in the $-80 \rightarrow -90$ MPa range and vary some 
with the adopted hoop stress which changes
along $\theta_c=\pi$.

The negative $T$-stress values are as expected with the shallow crack geometry.

Table 1.3.2 summarizes the $T$-stress values computed for the three loadings using the
interaction integral and displacement approaches. The FE hoop stress on 
$\theta_c=\pi$ is used in Eq. (\ref{E:Tstressdispl}).

\begin{table}[htb]	
\centering
{
\setlength{\extrarowheight}{2.2pt}
\small
\begin{tabular}{ | p{3.2in} | p{2.5in} | }
\hline
\textbf{Quantity} & \textbf{Values} \\
\hline \hline
Vertical reaction @ $Y_g=0$:  &  \\ \hline
\hspace{2em}crack-face pressure & Exact: 0.017708 MN; FE: 0.017699 MN \\ \hline
\hspace{2em}internal pressure & Exact: 1.9462 MN; FE: 1.9462 MN \\ \hline
\hspace{2em}$\Delta ( T )$ & Exact: 0.0 MN; FE: $0.49 \times 10^{-8}$ MN \\ \hline
$K_I$ values: &  \\ \hline
\hspace{1em}15.2 MPa face pressure from $J$, $I^\ddag$, displacement & 3.02, 3.03, 3.14 MPa$\sqrt{\mathrm{m}}$ \\ \hline
\hspace{1em}15.2  MPa internal pressure from $J$, $I$, displacement & 13.5, 13.5, 13.6  MPa$\sqrt{\mathrm{m}}$ \\ \hline
\hspace{1em}$\Delta T^\dag$ from $J$, displacement & 95.1, 96.3 MPa$\sqrt{\mathrm{m}}$ \\ \hline
Material CTE mismatch effect $K_I$ values: &  \\ \hline
\hspace{2em}$\Delta T^\dag$ $\mathrm{CTE_{clad}>CTE_{base}}$  & 95.1 MPa$\sqrt{\mathrm{m}}$ \\ \hline
\hspace{2em}$\Delta T^\dag$ $\mathrm{CTE_{clad}=CTE_{base}}$ & 89.6 MPa$\sqrt{\mathrm{m}}$ \\ \hline
\hspace{2em}$\Delta T^\dag$ $\mathrm{CTE_{clad}=0}$ & 33.3, 33.6 MPa$\sqrt{\mathrm{m}}$ \\ \hline
$T$-stress values: &  \\ \hline
\hspace{1em}15.2 MPa face pressure from $I$, displacement & 8.5, -0.9 MPa \\ \hline
\hspace{1em}15.2 MPa internal pressure from $I$, displacement & -53, -43 MPa \\ \hline
\hspace{1em}$\Delta T^\dag$ from  displacement & -80 to -90  MPa \\ \hline
\end{tabular}}
\small
$^\dag$ -188C decrease on inner surface; 0C change at outer surface. Quadratic $r_g$ variation.	\\
$I^\ddag$ denotes domain interaction integral.
%
\normalsize
\caption{Table 1.3.2 
Computed quantities for the linear-elastic vessel analysis.}
\label{table:linear-vessel-results}
\end{table}


\nid \ti{\ul{Notes on input files}}

\nid Figure \ref{fig:rpvlinearinput1} lists the \ti{warp3d.inp} highest level file
to define the model, request solution and output results. Most all of the comment
lines are removed to save space. The \ti{list} command is used extensively to
define convenient names for long lists of node/element numbers. As indicated
by the marker A, the \ti{list} commands have features to construct lists of nodes that
satisfy certain geometric requirements; here the nodes that lie on the $\theta_g=3^\degree$
plane.  At marker B, three definitions are given for load step 1 of the analysis. Two of the three are
commented to run an analysis [recall that all solutions in WARP3D are assumed to be
nonlinear and dynamic].  The clad and base materials have a large yield stress to prevent
yielding and the time step is very large to eliminate inertia effects (note also that the material
mass densities are not specified and have default values = 0).


%
\begin{sidewaysfigure}
\begin{center}
\includegraphics[trim=0.0in 0in 0in 0.1in, clip=true,scale=0.8,angle=0]
{figures_example_3/figure_4_warp3d_inp} 
\caption{{\small Fig. \thefigure\ Highest level input file 
for linear analysis of the pressure vessel. }
\label{fig:rpvlinearinput1}}
%
\end{center}
\end{sidewaysfigure}
%

\nid Figure \ref{fig:rpvlinearinput2} lists the \ti{tied\_contact.inp}  file
and a sketch of the \ti{surfaces} of element faces tied together. 
Any convenient/descriptive names may be assigned to the surfaces --
here there are four surfaces. The \ti{master} surfaces most often
are assigned to the coarser side of the mesh transition.

\nid Figure \ref{fig:rpvlinearinput3} lists the \ti{domain\_define\_compute.inp}  file.
A \ti{domain} is set-up at each of the four corner nodes along the
crack front at $\theta_g=0,1,2,3^\degree$. In this file, the \ti{compute}
commands are commented for  $\theta_g=1,2^\degree$ -- these were
included in the mesh verification tests to insure identical $J$ and $I$ values
at each location along the front in this axisymmetric model. Other
comment lines are omitted in this listing to save space. 

The domain definitions each have a command \ti{crack face tractions ty 15.2}.
When the crack face loading is present, this allows a more accurate evaluation
of the $I$-integrals than is possible using the equivalent nodal forces
corresponding to the applied face pressure. This command should be
commented for other loadings.

The \ti{tangent vector} commands for domains at  $\theta_g=0,3^\degree$ 
are necessary to maintain path independence of $J,\, I$ values. The chordal
crack-front segments defined by the linear elements do not intersect the
$\theta_g=0,3^\degree$ planes at a right angle (see Chapter 4 for 
more details). Without this command,
the $J,\,I$ values become increasingly path dependent as domains expand outward
from the front. At the interior locations, $\theta_g=1,2^\degree$, the average of the
tangents to the adjacent two chords lies tangent to curved front.


%
\begin{sidewaysfigure}
\begin{center}
\includegraphics[trim=0.0in 0in 0in 0.1in, clip=true,scale=0.8,angle=0]
{figures_example_3/figure_5_tied_contact} 
\caption{{\small Fig. \thefigure\ Input file to define tied-contact
for coare-fine mesh transition. }
\label{fig:rpvlinearinput2}}
%
\end{center}
\end{sidewaysfigure}
%


%
\begin{sidewaysfigure}
\begin{center}
\includegraphics[trim=0.0in 0in 0in 0.1in, clip=true,scale=0.8,angle=0]
{figures_example_3/figure_6_domain_define} 
\caption{{\small Fig. \thefigure\ Input file to define/request $J,\, I$ at nodes
along the crack front. }
\label{fig:rpvlinearinput3}}
%
\end{center}
\end{sidewaysfigure}
%

\subsection{Nonlinear Fracture: Crack-Tip Parameters}  

\nid Analyses of the reactor pressure vessel described in the previous section are
extended here to include elastic-plastic material response and temperature-dependent
strain-stress curves. Representative strain-stress curves for the stainless steel clad and
ferritic steel base metal are shown in Fig. \ref{fig:rpvsigeps}. The analyses employ 
the simple \ti{bilinear} mises material
model which supports only a constant hardening rate; 
a more realistic, detailed representation of the strain-stress curves is supported by
the \ti{mises} model (additional data values to describe the shape of 
the stress \ti{vs.} plastic-strain curves). Modulus values $E$, vary with temperature as
well but the CTE values remain invariant of temperature.

%
\begin{sidewaysfigure}
\begin{center}
\includegraphics[trim=0.0in 2in 3.0in 0.0in, clip=true,scale=0.9,angle=0]
{figures_example_3/figure_7_sig_eps_curves} 
\caption{{\small Fig. \thefigure\ Temperature dependent, constant hardening 
flow properties for base and clad materials. }
\label{fig:rpvsigeps}}
%
\end{center}
\end{sidewaysfigure}
%


The loading sequence for nonlinear analysis becomes: (1) set initial, uniform temperature to 288C,
(2) increase the internal pressure to 15.2 MPa in five equal load steps, then (3) impose the $\Delta T$
field over the thickness (described in previous section) in 47 equal increments that lower
the surface of the vessel in increments of -4C
to 100C while maintaining 288C everywhere on the outside surface. A larger and smaller
number of load steps may be used for the solution. As the number of load steps decreases,
consideration must be given to represent adequately the changing flow stress with
temperature over the wall thickness. 


Three nonlinear analyses are executed: (1) with the temperature-dependent material properties of
Fig. \ref{fig:rpvsigeps}, (2) with the material properties held fixed at room 
temperature (20C), and (3) with the material properties held fixed 
at 250C (as an approximation for 288C). Global Newton iterations for each load step of each analysis 
converge readily as the model response 
remains effectively in small-scale yielding (SSY). $J$-values are extracted using data on rings 15-20 at the 
$\theta_g=0^\degree$ crack-front position. $J$-values are identical at each of the other crack-front positions
as verified in the prior LEFM solutions. $K_J$-values for plotting are computed using the plane-strain conversion 
with $E^\prime$ equal to the value at the temperature of the crack-front (base material) elements [the
domain computation routines perform this conversion as a convenience]. Several Python
scripts (*.py) are included with the input files to illustrate extraction and plotting of the
results.

WARP3D input to define the material properties is shown in Fig. \ref{fig:rpvsigepsloadinput}a. 
Multiple materials are defined, \eg \ti{clad\_rm\_temp, clad\_temp\_dependent, ...}
to simplify editing of input for each of the three cases described above. The loading steps are defined in 
Fig. \ref{fig:rpvsigepsloadinput}b.
A separate \ttt{commands.inp} file has a pair of \ti{compute displacements} and $J$-value 
computation commands for each load step (see the included .py script that makes 
this convenient file). For convenience, the large volume of input data is separated into files to define the mesh, 
constraints, loading and domain definitions. The directory \ttt{manual\_examples\_chpt\_1} in the WARP3D
distribution contains all files for this linear analysis and the subsequent nonlinear analyses of this
model. Subdirectory \ttt{example\_3\_nonlinear} contains the
files for this linear analysis. The main input file is named \ttt{warp3d.inp}. Files 
may have any desired/descriptive names; the \ttt{.inp} suffix is not required. Other files are 
included during execution using
\ttt{*input from file "<file name>"} commands in \ttt{warp3d.inp}. The \ttt{*input ...} may be nested
as convenient. Other support files are included which may be of interest (the model neutral
file, Python scripts to extract data/make plots, etc.)


%
\begin{sidewaysfigure}
\begin{center}
\includegraphics[trim=0.0in 1.3in 0.8in 0.1in, clip=true,scale=0.9,angle=0]
{figures_example_3/figure_8_sig_eps_load_input} 
\caption{{\small Fig. \thefigure\ (a) WARP3D input to define the 6 stress \ti{vs}
plastic-strain curves with temperature dependence and the corresponding
materials with temperature dependence, fixed room temperature
and fixed 250C properties. (b) Input to define the loading steps and the
list of expanded solution parameters for nonlinear analysis. }
\label{fig:rpvsigepsloadinput}}
%
\end{center}
\end{sidewaysfigure}
%

Figure \ref{fig:rpvKJtip}a shows the evolution of $K_J$-values as 
the temperature of the inside surface decreases from 288C
to 100C. The loading here represents an \ti{exceptionally strong thermal gradient} with
a fixed through-wall dependence of temperature on radius  -- taken only as
a simplification for the purposes of demonstrating the application of WARP3D.
The values are mildly sensitive to temperature dependence of 
the material properties -- the cladding yields for each set of material properties. 
$K_J$ reaches a maximum value of  99 $\mathrm{ MPa\sqrt{m}}$
for temperature-dependent properties compared to 
112 $\mathrm{ MPa\sqrt{m}}$ for the linear elastic solution -- about a
10\% reduction from plasticity.  The solution with CTE $=0$ for the clad material
shows the strong contribution from contraction in the clad to the $K_J$ values
at the crack-front. At
an inner surface temperature of 100C, Fig.  \ref{fig:rpvKJtip}b
shows a near view of the deformed crack-front region (at true scale) with fringes 
indicating the accumulated
plastic strain ($\overline \vareps_p$). The maximum extent of the plastic region is 
approximately 10-12 mm along the red line at about $\theta_c\doteq70^\degree$ extending
from the crack front as indicated on the figure. The simple estimate for plastic zone size
is $r_p\doteq 0.15 (K_J/\sigma_{flow})^2=9$ mm.
The CTOD at the 100C level is approximately 0.072 mm (72 $\mu$m, 2.8 mils) taken as $2\,\times$ the
opening displacement of nodes just behind the crack front. This leads to an
estimated $J$-value: $J\doteq1.5\,\sigma_{flow}\,\delta= 1.5\times420\times 
(72\times10^{-6})=45\,\mathrm{ kJ/m^2}$ compared to the domain-integral value of 
44.3 $\mathrm{ kJ/m^2}$. 
%
\begin{figure}
\begin{center}
\includegraphics[trim=0.5in 0.0in 0.8in 0.0in, clip=true,scale=1.0,angle=0]
{figures_example_3/figure_9_KJ_values_crack_tip} 
\caption{{\small Fig. \thefigure\ (a) $K_J$ values from nonlinear analyses using
four different sets of material properties. Internal pressure remains constant at
15.2 MPa during temperature reduction over wall thickness. (b) Near front
deformation (true scale) at end of temperature reduction and fringes of
accumulated plastic strain. For reference, nominal yield strain is
$\approx 0.0023$. }
\label{fig:rpvKJtip}}
%
\end{center}
\end{figure}
%

Figure \ref{fig:rpvcladyield} shows an overview of plastic strains ahead of the crack front and in the
cladding. The clad has non-zero plastic strain of magnitude $1\mhyphen2\,\times$ the nominal 
yield strain of $\approx 0.0018$. A thin layer of base metal immediately 
adjacent to the clad also appears to experience quite small amounts of yielding,


%
\begin{sidewaysfigure}
\begin{center}
\includegraphics[trim=0.5in 2.0in 2.8in 0.1in, clip=true,scale=0.9,angle=0]
{figures_example_3/figure_10_clad_yielding} 
\caption{{\small Fig. \thefigure\ Accumulated plastic strains 
at the end of loading with inside surface at 100C, outside surface 288C and
internal pressure of 15.2 MPa.  Nominal clad yield strain is 0.002.}
\label{fig:rpvcladyield}}
%
\end{center}
\end{sidewaysfigure}
%

%*****************************************************
\subsection {References}
%*****************************************************
\small 


\medskip
\noindent[\refstepcounter{sectrefs}\label {R:NV1984}\thesectrefs]~J.C. Nagtegaal, and F.E. Veldpaus.
On the Implementation of Finite Strain Plasticity Equations in a Numerical Model.
In Numerical Analysis of Forming Processes (edited by J.F. Pittman, O. C. Ziekiewicz, R. D. Wood 
and J. M. Alexander), p. 351. John Wiley and Sons, New York, 1984.


\medskip
\noindent[\refstepcounter{sectrefs}\label {R:SH1998}\thesectrefs]~J.C. Simo and T.J.R. Hughes.  
Computational Inelasticity (Interdisciplinary Applied Mathematics) (v. 7).  Springer-Verlag New York, Inc. 1998.
ISBN 0-387-97520-9.

\medskip
\noindent[\refstepcounter{sectrefs}\label {R:F1973}\thesectrefs]~W. Fl{\"u}gge.
Stress in shells. Springer-Verlag, Berlin. 1973.

\medskip
\noindent[\refstepcounter{sectrefs}\label {R:APS1998}\thesectrefs]~M.R. , M.J. Pavier, and 
D.J. Smith. Determination of $T$-stress from finite element analysis for mode 
I and mixed mode I/II loading\ti{International Journal of Fracture}, Vol. 91, No. 3, pp. 283-298, 1998.

\end{document}



\documentclass[11pt]{report}
\usepackage{geometry} 
\geometry{letterpaper}

%---------------------------------------------
\setlength{\textheight}{630pt}
\setlength{\textwidth}{450pt}
\setlength{\oddsidemargin}{14pt}
\setlength{\parskip}{1ex plus 0.5ex minus 0.2ex}


%----------------------------------------
\usepackage{amsmath}
\usepackage{layout}
\usepackage{color}
\usepackage{array}

%----------------------------------------------
\usepackage{fancyhdr} \pagestyle{fancy}
\setlength\headheight{15pt}
\lhead{\small{User's Guide - \textit{WARP3D}}}
\rhead{\small{Equations of Motion}}
\fancyfoot[L] {\small{\textit{Chapter {\thechapter}}\ \   (Updated: 5-20-2015)}}
\fancyfoot[C] {\small{\thesection-\thepage}}
\fancyfoot[R] {\small{\textit{Introduction}}}

%---------------------------------------------------
\usepackage{graphicx}
\usepackage[labelformat=empty]{caption}
\numberwithin{equation}{section}
\usepackage{bm}

%---------------------------------------------
%     --- make section headers in helvetica ---
%
\usepackage{sectsty} 
\usepackage{xspace}
\allsectionsfont{\sffamily} 
\sectionfont{\large}
\usepackage[small,compact]{titlesec} % reduce white space around sections
%---------------------------------------------->
%
%
%   which fonts system for text and equations. with all commented,
%   the default LaTex CM fonts are used
%
%
\frenchspacing
%\usepackage{pxfonts}  % Palatino text 
%\usepackage{mathpazo} % Palatino text
%\usepackage{txfonts}


%---------  local commands ---------------------


\newcommand{\bmf } {\boldsymbol }  %bold math symbol
\newcommand{\bsf } [1]{\textrm{\textit{#1}}\xspace}
\newcommand{\ul} {\underline}
\newcommand{\hv} {\mathsf}   %helvetica text inside an equation
\newcommand{\eg}{\emph{e.g.},\xspace}
\newcommand{\ti}{\emph}
\newcommand{\bardelta}{\bar \delta}
\newcommand{\vepsilon}{\varepsilon}
\newcommand{\etal}{\ti{et al.}\xspace}

\newenvironment{offsetpar}[1]%
{\begin{list}{}%
         {\setlength{\leftmargin}{#1}}%
         \item[]%
}
{\end{list}}

%
%
%        optional definition for bullet lists which
%        reduces white space.
%
\newcommand{\squishlist}{
 \begin{list}{$\bullet$}
  { \setlength{\itemsep}{0pt}
     \setlength{\parsep}{3pt}
     \setlength{\topsep}{3pt}
     \setlength{\partopsep}{0pt}
     \setlength{\leftmargin}{1.5em}
     \setlength{\labelwidth}{1em}
     \setlength{\labelsep}{0.5em} } }

\newcommand{\squishlisttwo}{
 \begin{list}{$\bullet$}
  { \setlength{\itemsep}{0pt}
     \setlength{\parsep}{0pt}
    \setlength{\topsep}{0pt}
    \setlength{\partopsep}{0pt}
    \setlength{\leftmargin}{2em}
    \setlength{\labelwidth}{1.5em}
    \setlength{\labelsep}{0.5em} } }

\newcommand{\squishend}{
  \end{list}  }
%


%-------------------------------------
\newcounter{sectrefs}
\setcounter{sectrefs}{0}
\setcounter{chapter}{1}
\setcounter{section}{3}
\setcounter{figure}{0}
\renewcommand{\thefigure}{\thesection.\arabic{figure}}
%
%--------------------------------------
%
%
%
%              start document 
%              ==========
%
%
\begin{document}

\section{Nonlinear Equations of Motion}
\noindent 
The model occupies the configuration $\mathcal{R}_0$  at time $t=0$ and  
evolves through time to the deformed configuration $\mathcal{R}$ at time $t$.  
For quasi-static simulations, time 
maybe replaced by a loading-like parameter.
In the $\mathcal{R}_0$ configuration, the model is undeformed and at rest. 
In reaching the deformed configuration, the model may accelerate and displace, 
including simple rigid body translation/rotation in addition to true deformation. 
This situation is illustrated in Fig. \ref{fig:motion}. The position vector $\bm{X}$ locates
a material point in 
the undeformed configuration and $\bm{x}$ denotes the position vector of the same point 
in the deformed (current) configuration. The displacement vector, $\bm{d}$, 
takes the point from the initial to the deformed configuration. The coordinates 
of the model in the reference configuration represent the  geometry interpolated 
from the parametric coordinates in the isoparametric formulation. The nonlinear 
implementation of the finite element method in WARP3D employs a continuously 
updated formulation naturally suited for solids with only translational dof at the nodes. 
The expression of virtual work defining equilibrium and the equations of motion are 
defined and solved on the current, $\mathcal{R}$, configuration. 

The remainder of this section summarizes the equations of motion. 
Methods for solution of the resulting nonlinear algebraic equations are described 
in subsequent sections and followed by descriptions of the specific finite element 
formulations and the adopted formulation to model finite strains and rotations.

%
\begin{figure}[h]
\begin{center}
\includegraphics[trim=0.0in 6.5in 3in 0.5in, clip=true,scale=1.0,angle=0]{fig_motion.pdf} 
\caption{{\small Fig. \thefigure\ Definition of initial and current (deformed) configurations.
Equations of motion are written on the deformed configuration.}
\label{fig:motion}}
%
\end{center}
\end{figure}
%



The weak formulation of momentum balance equations (virtual work) 
expressed in the current configuration is given by \footnote{No 
damping terms are included in the current implementation. May be added in the future.}	
%
\begin{equation}\label{E:weak_form_1}
\int_{V}\delta \bm{\varepsilon}^T \bm{\sigma}\, dV -\int_{V}\delta 
\bm{d}^T \bm{f} \, dV - \sum_{i=1}^m \delta \bm{d}_i^T \bm{p}_i = 0
\end{equation}
%
\noindent where $V$ denotes the current volume, $\delta \bm{\varepsilon}$ and 
$\bm{\sigma}$ are the virtual rate of deformation 
vector and the Cauchy stress vector, $\bm{f}$ is the known body force vector 
per unit volume in the deformed 
configuration, and each $\bm{p}_i$ is a $3 \times 1$ vector of known forces acting 
at $m$ discrete 
points, see Belytschko \etal  [\ref{R:BLME2014}]. 
We use $6 \times 1$ vector forms 
of the symmetric tensors for $\delta \bm{\varepsilon}$ and $\bm{\sigma}$. 
The operator $\delta$ denotes the usual, arbitrary  virtual 
variation compatible with the essential (geometric)
boundary conditions. The virtual rate of deformation tensor and the Cauchy stress tensor form 
a work-conjugate pair defined on the current configuration.
The nodal forces $\bm{p}_i$ may comprise, for example, (1) directly applied nodal 
forces, (2) work-equivalent 
nodal forces due to specified surface tractions applied on element 
faces and other body forces, \eg self-weight, (3) forces necessary to maintain specified
contact conditions, (4) forces
required to maintained specified relationships between displacements
(multi-point constraints), and (5) forces that arise from various crack growth processes
(node release, element removal). Inertial D'Alembert forces arising 
from accelerations are given by
%
\begin{equation}\label{E:f definition}
\bm{f} = - \rho\, \ddot{\bm{d}}
\end{equation}
%
\noindent where $\rho$ is the mass density in the deformed configuration. 
By including acceleration forces in $\bm{f}$ and equivalent nodal forces from
applied surface tractions and body forces in $\bm{p}_i$, Eq. (\ref{E:weak_form_1}) becomes	
%
\begin{equation}\label{E:weak_form_2}
\int_{V}\delta \bm{\varepsilon}^T \bm{\sigma}\, dV + \int_{V}\delta \bm{d}^T  
\rho \ddot{\bm{d}} \, dV - \sum_{i=1}^m \delta \bm{d}_i^T \bm{p}_i = 0\ .
\end{equation}
%
\noindent Following standard procedures (see Cook, \etal  [\ref{R:CMPW2001}], Hughes [\ref {R:H2000}]), 
Eq. (\ref{E:weak_form_2}) transforms from a continuum form to an 
(equivalent) finite element form as given below, beginning 
with integrations over each element to define the volume 
integral over the model
%
\begin{equation}\label{E:fe_form_1}
\sum_{j=1}^{\#elem}   \int_{V_e^j}\delta \bm{\varepsilon}^T \bm{\sigma}\, dV_e +
\sum_{j=1}^{\#elem} \int_{V_e^j}\delta \bm{d}^T  
\rho \ddot{\bm{d}} \, dV_e - \sum_{i=1}^m \delta \bm{d}_i^T \bm{p}_i = 0
\end{equation}
%
%
\begin{equation}\label{E:fe_form_2}
\sum_{j=1}^{\#elem}  \left ( \delta \bm{u}_e^T\bm{I}_e\right)_j +
\sum_{j=1}^{\#elem} \left ( \delta \bm{u}_e^T\mathbf{M}_e\ddot{\bm{u}}_e\right)_j   
 - \sum_{i=1}^m \delta \bm{d}_i^T \bm{p}_i = 0
\end{equation}
%
%
\begin{equation}\label{E:fe_form_3}
\delta \bm{u}^T    \left(       \sum \bm{I}_e +
\left ( \sum \mathbf{M}_e \ddot{\bm{u}}_e  \right )  
 - \bm{P} \right ) = 0
\end{equation}
%
\noindent 
where $\bm{u}$ is the global nodal displacement vector, $\bm{u}_e$ 
is an element nodal displacement vector, $\bm{I}_e$ is an element internal force vector, 
$\mathbf{M}_e$ is an element mass matrix,  and $\bm{P}$ is the global vector of known nodal forces. 
Subsequent sections outline procedures to compute the element internal force vector and 
the element mass matrix as well as the element tangent stiffness matrix. 
The summations in Eq. (\ref{E:fe_form_3}) imply the standard 
assembly process. Since the $\delta \bm{u}$ are arbitrary we have then
%
\begin{equation}\label{E:fe_form_4}
      \sum I_e +
\left ( \sum \mathbf{M}_e \right ) \ddot{\bm{u}}_e   
 - \bm{P}  = 0\ .
\end{equation}
%
After performing the assembly processes implied by the $\sum$ in Eq. (\ref{E:fe_form_4}), 
the global equation of motions become simply
%
\begin{equation}\label{E:fe_form_5}
 \bm{I} + \bm{M}  \ddot{\bm{u}} - \bm{P}  = 0\  .
\end{equation}
%
The vectors have size $3 \times nnode$, where $nnode$ denotes the number of 
structure nodes. 

Nonlinearity in $\bm{I}$ arises from the element internal force vectors 
(geometric and/or material effects) while $\bm{P}$ may become nonlinear  from contact conditions,
crack growth, and
when tractions applied to element faces have constant orientation relative 
to the deformed face (\eg pressure loads).

%*****************************************************
\subsection {References}
%*****************************************************
\small

\noindent[\refstepcounter{sectrefs}\label {R:BLME2014}\thesectrefs]~T. Belytschko, W.K. Liu, B. Moran, K. Elkhodary.
Nonlinear Finite Elements for Continua and Structures. 2nd Edition. John Wiley \& Sons, Inc., 2014.

\medskip
\noindent[\refstepcounter{sectrefs}\label {R:CMPW2001}\thesectrefs]~R. Cook, D. Malkus, M. Plesha, R. Witt.
Concepts and Applications of Finite Element Analysis, 4th Edition. John Wiley \& Sons, Inc., 2001.

\medskip
\noindent[\refstepcounter{sectrefs}\label {R:H2000}\thesectrefs]~T.J. Hughes.
The Finite Element Method: Linear Static and Dynamic Finite Element Analysis. Dover, Inc., 2000.

\end{document}

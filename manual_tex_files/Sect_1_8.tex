
\documentclass[11pt]{report}
\usepackage{geometry} 
\geometry{letterpaper}

%---------------------------------------------
\setlength{\textheight}{630pt}
\setlength{\textwidth}{450pt}
\setlength{\oddsidemargin}{14pt}
\setlength{\parskip}{1ex plus 0.5ex minus 0.2ex}


%----------------------------------------
\usepackage{amsmath}
\usepackage{layout}
\usepackage{color}
\usepackage{array}

%----------------------------------------------
\usepackage{fancyhdr} \pagestyle{fancy}
\setlength\headheight{15pt}
\lhead{\small{User's Guide - \textit{WARP3D}}}
\rhead{\small{Element Formulations}}
\fancyfoot[L] {\small{\textit{Chapter {\thechapter}}\ \   (Updated: 7-30-2014)}}
\fancyfoot[C] {\small{\thesection-\thepage}}
\fancyfoot[R] {\small{\textit{Introduction}}}

%---------------------------------------------------
\usepackage{graphicx}
\usepackage[labelformat=empty]{caption}
\numberwithin{equation}{section}
\usepackage{bm}

%---------------------------------------------
%     --- make section headers in helvetica ---
%
\usepackage{sectsty} 
\usepackage{xspace}
\allsectionsfont{\sffamily} 
\sectionfont{\large}
\usepackage[small,compact]{titlesec} % reduce white space around sections
%---------------------------------------------->
%
%
%   which fonts system for text and equations. with all commented,
%   the default LaTex CM fonts are used
%
%
\frenchspacing
%\usepackage{pxfonts}  % Palatino text 
%\usepackage{mathpazo} % Palatino text
%\usepackage{txfonts}


%---------  local commands ---------------------


\newcommand{\bmf } {\boldsymbol }  %bold math symbol
\newcommand{\bsf } [1]{\textrm{\textit{#1}}\xspace}
\newcommand{\ul} {\underline}
\newcommand{\hv} {\mathsf}   %helvetica text inside an equation
\newcommand{\eg}{\emph{e.g.},\xspace}
\newcommand{\ie}{\emph{i.e.},\xspace}
\newcommand{\ti}{\emph}
\newcommand{\vepsilon}{\varepsilon}
\newcommand{\etal}{\ti{et al.}\xspace}
\newcommand{\nid}{\noindent}
\newcommand{\vareps}{\varepsilon}

\newenvironment{offsetpar}[1]%
{\begin{list}{}%
         {\setlength{\leftmargin}{#1}}%
         \item[]%
}
{\end{list}}

%
%
%        optional definition for bullet lists which
%        reduces white space.
%
\newcommand{\squishlist}{
 \begin{list}{$\bullet$}
  { \setlength{\itemsep}{0pt}
     \setlength{\parsep}{3pt}
     \setlength{\topsep}{3pt}
     \setlength{\partopsep}{0pt}
     \setlength{\leftmargin}{1.5em}
     \setlength{\labelwidth}{1em}
     \setlength{\labelsep}{0.5em} } }

\newcommand{\squishlisttwo}{
 \begin{list}{$\bullet$}
  { \setlength{\itemsep}{0pt}
     \setlength{\parsep}{0pt}
    \setlength{\topsep}{0pt}
    \setlength{\partopsep}{0pt}
    \setlength{\leftmargin}{2em}
    \setlength{\labelwidth}{1.5em}
    \setlength{\labelsep}{0.5em} } }

\newcommand{\squishend}{
  \end{list}  }
%
\newcounter{Lcount}
\newcommand{\squishnum}{
\begin{list}{\arabic{Lcount}. }
{ \usecounter{Lcount}
\setlength{\itemsep}{0pt}
\setlength{\parsep}{3pt}
\setlength{\topsep}{3pt}
\setlength{\partopsep}{0pt}
\setlength{\leftmargin}{1in}
\setlength{\labelwidth}{1em}
\setlength{\labelsep}{0.5em} } }

\makeatletter
\renewcommand*\env@matrix[1][\arraystretch]{%
  \edef\arraystretch{#1}%
  \hskip -\arraycolsep
  \let\@ifnextchar\new@ifnextchar
  \array{*\c@MaxMatrixCols c}}
\makeatother


%-------------------------------------
\newcounter{sectrefs}
\setcounter{sectrefs}{0}
\setcounter{chapter}{1}
\setcounter{section}{7}
\setcounter{figure}{0}
\renewcommand{\thefigure}{\thesection.\arabic{figure}}
%
%--------------------------------------
%
%
%
%              start document 
%              ==========
%
%
\begin{document}

\section{Element Formulations}
\nid 
Development of the finite element formulation for three-dimensional solid elements
begins with interpolation of the element geometry and displacement fields. The description
here focuses on the kinematic (geometric) nonlinear formulation; simplifications to
obtain the linear-displacement formulation are straightforward.  The development employs
isoparametric concepts for hexahedral elements -- modifications for tetrahedral elements using
triangular parametric coordinates represent extensions and are described in the formulations
for those element in Chapter 3. Interface-cohesive elements 
use displacement
jumps across initially coincident surfaces and
corresponding work conjugate normal and shear tractions -- their formulation is
described in Chapter 3 with descriptions of the interface elements and the cohesive
constitutive models.

Let $\bmf{X}_{3\times 1}$ denote the Cartesian position vector for material points at
time $t=0$ (see Fig. 1.4).  Position vectors for material points at time $t$ are denoted 
$\bmf{x}_{3\times 1}$. In the Lagrangian approach, each node of an element corresponds
to the same material point throughout the deformation process. The displacements of material
points are thus given by $\bmf{u} = \bmf{x}-\bmf{X}$ and the material point
velocity vectors by $\dot{\bmf{u}}$ (later we will also use $\bmf{v}$ to denote material point
velocities). In static analyses, we associate a time-like parameter $t$ 
with a specified level of external loading imposed on the model to facilitate definition
of strain and stress rates when needed.

\subsection{Interpolating Functions}

\nid The velocity of a material point at time $t$ is interpolated from the nodal 
velocities using isoparametric functions expressed in matrix form as
%
\begin{equation}\label{E:IF1}
\dot{\bmf{d}} =
 \begin{Bmatrix}\dot u \\ \dot v\\ \dot w\end{Bmatrix}_{3\times 1} = 
 \hat{\mathbf{N}}_{3\times3ne} \begin{Bmatrix} \left (\dot{\bmf{u}}_e\right )_{ne \times 1} 
 \\ \left (\dot{\bmf{v}}_e\right )_{ne \times 1} \\ 
 \left (\dot{\bmf{w}}_e\right )_{ne \times 1}\end{Bmatrix}_{3ne\times 1}  =
 \hat{\mathbf{N}} \, \dot{\bmf{u}}_e
\end{equation}
%
\nid where $ne$ denotes the number of element nodes. Note the non-conventional 
ordering of nodal values in $\dot{\bmf{u}}_e$ which facilitates local vectorization
of element-level processing code with stride one innermost array accesses.
Cartesian coordinates of material points at time $t$ are interpolated from the
current nodal coordinates using the same functions
%
\begin{equation}\label{E:IF2}
\bmf{x} =
 \begin{Bmatrix} x \\ y\\ z\end{Bmatrix}_{3\times 1} = 
 \hat{\mathbf{N}}_{3\times3ne} \begin{Bmatrix} \left (\bmf{x}_e\right )_{ne \times 1} 
 \\ \left ( \bmf{y}_e\right )_{ne \times 1} \\ 
 \left (\bmf{z}_e\right )_{ne \times 1}\end{Bmatrix}_{3ne\times 1}  =
 \hat{\mathbf{N}} \, \bmf{c}_e
\end{equation}
%
\nid where $\bmf{c}_e = \bmf{c}_e(t=0) + \bmf{u}_e$. The element interpolating functions, one for each
node, are functions of the usual parametric coordinates $\xi,  \eta, \zeta$.
For convenience, the interpolating functions are grouped in the row
%
\begin{equation}\label{E:IF3}
\bmf{N} = \langle \ N_1\ N_2\  N_3\  \dots\  N_{ne} \rangle\ .
\end{equation}
%
Derivatives \ti{wrt} the parametric coordinates are  also represented
in row format
%
\begin{equation}\label{E:IF4}
\begin{aligned}
\bmf{N}_{,\xi} &= \langle \ N_{1,\xi}\ N_{2,\xi}\  N_{3,\xi}\  \dots\  N_{ne,\xi} \rangle  \\
\bmf{N}_{,\eta}& = \langle \ N_{1,\eta}\ N_{2,\eta}\  N_{3,\eta}\  \dots\  N_{ne,\eta} \rangle  \\
\bmf{N}_{,\zeta}& = \langle \ N_{1,\zeta}\ N_{2,\zeta}\  N_{3,\zeta}\  \dots\  N_{ne,\zeta} \rangle \ .
\end{aligned}
\end{equation}
%
\nid We define the interpolating matrix used in Eq. (\ref{E:IF1}, \ref{E:IF2}) as 
%
\begin{equation}\label{E:IF5}
\hat {\mathbf{N}} = \begin{bmatrix} \bmf{N} &\bmf{ 0} & \bmf{0} \\
 \bmf{0} & \bmf{N} & \bmf{0} \\ 
\bmf{0} & \bmf{0} & \bmf{N} \end{bmatrix}_{3\times 3ne} \ .
\end{equation}
%

\subsection{Cartesian Derivatives}

\nid The standard Jacobian matrix relates derivatives in parametric 
and Cartesian coordinates via the chain rule. Let $H(\xi, \eta, \zeta)$ be a smooth
function. For example,
%
\begin{equation}\label{E:CDa}
\frac{\partial H}{\partial \xi} = \frac{\partial H}{\partial x} \frac{\partial x}{\partial \xi} + 
\frac{\partial H}{\partial y} \frac{\partial y}{\partial \xi} +
 \frac{\partial H}{\partial z} \frac{\partial z}{\partial \xi} 
\end{equation}
%
\nid with similar forms for $\partial H /\partial \eta$ and $\partial H /\partial \zeta$. In matrix form we have
%
\begin{equation}\label{E:CDb}
 \begin{bmatrix}[2.0]
 \dfrac{\partial H}{\partial \xi} \\ \dfrac{\partial H}{\partial \eta} \\\dfrac{\partial H}{\partial \zeta} 
 \end{bmatrix}_{3\times 1} = \mathbf{J}
 \begin{bmatrix}[2.0]
 \dfrac{\partial H}{\partial x } \\ \dfrac{\partial H}{\partial y} \\\dfrac{\partial H}{\partial z} 
 \end{bmatrix}_{3\times 1} 
\end{equation}
%
\nid where
%
\begin{equation}\label{E:CDc}
\mathbf{J} = \begin{bmatrix}[2.0]
 \dfrac{\partial x}{\partial \xi }  &  \dfrac{\partial y}{\partial \xi } & \dfrac{\partial z}{\partial \xi}  \\ 
\dfrac{\partial x}{\partial \eta }  &  \dfrac{\partial y}{\partial \eta } & \dfrac{\partial z}{\partial \eta}  \\ 
 \dfrac{\partial x}{\partial \zeta }  &  \dfrac{\partial y}{\partial \zeta } & \dfrac{\partial z}{\partial \zeta}  
 \end{bmatrix}_{3\times 3} 
\end{equation}
%
\nid with the inverse of the Jacobian matrix denoted by
%
\begin{equation}\label{E:CDe}
\mathbf{\Gamma} = \mathbf{J}^{-1}\ .
\end{equation}
%
\nid The gradients of velocity \ti{wrt} the $\bmf{x}$ configuration are written in the form
%
\begin{equation}\label{E:CDf}
\dot{\bmf{\theta}} = \begin{Bmatrix}[1.2]
\dot{ \bmf{d}}_{,x}  \\ 
\dot{ \bmf{d}}_{,y}   \\ 
\dot{ \bmf{d}}_{,z} 
 \end{Bmatrix}_{9\times 1} =  \begin{Bmatrix}[1.2]
\dot{ \bmf{\theta}}_{,x}  \\ 
\dot{ \bmf{\theta}}_{,y}    \\ 
\dot{ \bmf{\theta}}_{,z}  
 \end{Bmatrix}_{9\times 1} = \left [ \dot{u}_{,x} \  \dot{v}_{,x}\ \dot{w}_{,x}   \ 
 \dot{u}_{,y} \   \dot{v}_{,y} \  \dot{w}_{,y} \ 
 \dot{u}_{,z} \   \dot{v}_{,z} \  \dot{w}_{,z} 
  \right    ]^T \ .
\end{equation}
%
\nid The velocity gradients in parametric space are written similarly
%
\begin{equation}\label{E:CDg}
\dot{\bmf{\phi}} = \begin{Bmatrix}[1.2]
\dot{ \bmf{d}}_{,\xi}  \\ 
\dot{ \bmf{d}}_{,\eta}   \\ 
\dot{ \bmf{d}}_{,\zeta} 
 \end{Bmatrix}_{9\times 1} =   \left [ \dot{u}_{,\xi} \   \dot{v}_{,\xi}\  \dot{w}_{,\xi}   \ 
 \dot{u}_{,\eta} \   \dot{v}_{,\eta} \  \dot{w}_{,\eta} \ 
 \dot{u}_{,\zeta} \   \dot{v}_{,\zeta} \ , \dot{w}_{,\zeta} 
  \right    ]^T\ .
\end{equation}
%
Cartesian derivatives are computed from parametric derivatives using
%
\begin{equation}\label{E:CDh}
\dot{\bmf{\theta}} = \hat {\mathbf{\Gamma}} \dot{ \bmf{\phi}}
\end{equation}
%
\nid where
%
\begin{equation}\label{E:CDi}
\hat{\mathbf{\Gamma} }= \begin{bmatrix}[1.2]
 \Gamma_{11}\mathbf{I}_3  & \Gamma_{12}\mathbf{I}_3  &\Gamma_{13}\mathbf{I}_3  \\ 
 \Gamma_{21}\mathbf{I}_3  & \Gamma_{22}\mathbf{I}_3  &\Gamma_{23}\mathbf{I}_3  \\ 
 \Gamma_{31}\mathbf{I}_3  & \Gamma_{32}\mathbf{I}_3  &\Gamma_{33}\mathbf{I}_3 
  \end{bmatrix}_{9\times 9} \ .
\end{equation}
%
\nid Here, $\mathbf{I}_3$  denotes an order 3 identity matrix. 

To introduce the finite element
interpolating functions, we write
%
\begin{equation}\label{ECDj}
\dot{\bmf{\phi}} = \mathbf{G} \dot{ \bmf{u}}_e
\end{equation}
%
\nid where
%
\begin{equation}\label{E:CDk}
\mathbf{G}= \begin{bmatrix}[1.0]
\hat{ \bmf{N}}_{,\xi}  \\\hat{ \bmf{N}}_{,\eta}  \\\hat{ \bmf{N}}_{,\zeta}   
   \end{bmatrix}=
 \begin{bmatrix}[1.0]
 \bmf{N}_{,\xi} & \bmf{0} & \bmf{0}  \\  \bmf{0} & \bmf{N}_{,\xi} &  \bmf{0}  \\ \bmf{0} & \bmf{0}  & \bmf{N}_{,\xi} \\ 
  \bmf{N}_{,\eta} & \bmf{0} & \bmf{0}  \\  \bmf{0} & \bmf{N}_{,\eta} &  \bmf{0}  \\ \bmf{0} & \bmf{0}  & \bmf{N}_{,\eta} \\ 
 \bmf{N}_{,\zeta} & \bmf{0} & \bmf{0}  \\  \bmf{0} & \bmf{N}_{,\zeta} &  \bmf{0}  \\ \bmf{0} & \bmf{0}  & \bmf{N}_{,\zeta}  
   \end{bmatrix}_{9\times3ne}
  \end{equation}
%
\nid and $\bmf{0}$ is a $1 \times ne$ row of zeroes.
\subsection{Strain-Displacement $\mathbf{B}$ Matrix}

\nid At time $t$ we impose a compatible virtual displacement field on the current 
(deformed) configuration $\bmf{x}$. The symmetric (virtual) deformation tensor
is defined in $6 \times 1$ vector form as 
%
\begin{equation}\label{E:Ba}
\delta \bmf{\vareps} = \begin{Bmatrix}[1.2]
\delta \vareps_x  \\  
\delta \vareps_y  \\  
\delta \vareps_z  \\  
\delta \gamma_{xy}  \\  
\delta \gamma_{yz}  \\  
\delta \gamma_{xz}  \\  
\end{Bmatrix}_{6\times 1} = 
\begin{Bmatrix}[1.2]
\delta u_{,x}  \\  
\delta v_{,y}  \\  
\delta w_{,z}  \\  
\delta u_{,y} + \delta v_{,x}  \\  
\delta v_{,z} + \delta w_{,y}  \\  
\delta w_{,x} + \delta u_{,z}  \\  
\end{Bmatrix}_{6\times 1}  
\end{equation}
%
\nid where it is understood, for example, that 
$\delta u_{,x} = \partial (\delta u)/\partial x$. In terms of virtual nodal
displacements, we write in conventional form
%
\begin{equation}\label{E:Bb}
\delta \bmf{\vareps} = \mathbf{B}_{6\times 3ne} \, \delta \bmf{u}_{e\,(3ne\times1)}
\end{equation}
%
\nid where the strain-displacement matrix $\mathbf{B}$ is constructed formally
as follows. Define the Boolean matrix
%
\begin{equation}\label{E:Bc}
\widetilde{\mathbf{B}}= 
 \begin{bmatrix}[1.0]
 1\  0\ 0\ 0\ 0\ 0\ 0\ 0\ 0\ \\
 0\ 0\ 0\ 0\ 1\  0\ 0\ 0\ 0\  \\
 0\ 0\ 0\ 0\ 0\ 0\ 0\ 0\ 1 \\
 0\ 1\  0\ 1\  0\ 0\ 0\ 0\ 0\ \\
 0\ 0\ 0\ 0\ 0\ 1\  0\ 1\  0\ \\
 0\ 0\ 1\  0\ 0\ 0\ 1\  0\ 0 
   \end{bmatrix}_{6\times9}
  \end{equation}
%
which supports formal expression of the strain-displacement 
$\mathbf{B}$ using
%
\begin{equation}\label{E:Bd}
\mathbf{B}_{6\times 3ne} = \widetilde \mathbf{B}_{6\times9} \, 
\hat{\mathbf {\Gamma}}_{9\times9}\,  \mathbf{G}_{9\times 3ne}\ .
\end{equation}
%

In a similar approach, we may compute real strain rates $\dot{\bmf{\vareps}}$ 
when required using
the same form of Eq. (\ref{E:Bb})
%
\begin{equation}\label{E:Be}
 \dot{\bmf{\vareps}} = \mathbf{B}_{6\times 3ne} \,  \dot{\bmf{u}}_{e\,(3ne\times1)}\ .
\end{equation}
%
Also in quasi-static analyses we compute increments of real strains using
%
\begin{equation}\label{E:Bf}
 \Delta \bmf{\vareps} = \mathbf{B}_{6\times 3ne} \,  \Delta{\bmf{u}}_{e\,(3ne\times1)}\ .
\end{equation}
%
For geometric nonlinear analyses, 
$\mathbf{B}$ may be computed at multiple, different times $t$ as the global nonlinear
procedures iteratively advance the solution from $t_n \rightarrow t_{n+1}$,
for example, $\mathbf{B}(t_{n+1/2})$ or $\mathbf{B}(t_{n+1})$. Here, $\dot{\bmf{\vareps}}$ 
and $\Delta \bmf{\vareps}$ are vector forms of the \ti{spatial rate of deformation} tensor, $\mathbf{D}$
and $\Delta\mathbf{D}$, as discussed further in Section 1.9. For geometrically linear analyses,
 $\dot{\bmf{\vareps}}$ and $\Delta \bmf{\vareps}$ have the  usual forms as derivatives above are \ti{wrt}
 the undeformed coordinates ($\bmf{x}=\bmf{X})$.

\subsection{Internal Nodal Force Vector}
\nid From the virtual work expressions in Section 1.4, the internal virtual work 
for an element at time $t$ is 
%
\begin{equation}\label{E:IFVa}
 \delta W_{int} = \int_{Ve} \delta \bmf{\vareps}^T \bmf{\sigma} \, dV_e\ .
\end{equation}
%
\nid Using Eq. (\ref{E:Bb}) to cast the expressions into finite element form
\begin{equation}\label{E:IFVb}
 \delta W_{int} = \int_{Ve} \delta \bmf{\vareps}^T \bmf{\sigma} \, dV_e = 
 \delta \bmf{u}_e^T\int_{Ve} \mathbf{B}^T \bmf{\sigma} \, dV_e =
  \delta \bmf{u}_e^T \, \bmf{I}_{e(3ne\times1)} \ .
\end{equation}

In the above, $V_e$ denotes the deformed volume of the element at time
$t$ and $\bmf{\sigma}$ is the $6\times 1$ form of the symmetric Cauchy stress
tensor also at time $t$. The ordering of stresses follows those for the strain
components shown in Eq. (\ref{E:Ba}). The $\mathbf{B}$ matrix is evaluated using coordinates of 
element nodes at $t$, $\bmf{c}_e= \bmf{c}_e(t=0) +\bmf{u}_e(t)$.

We will have occasion to use an alternative but equivalent form of the internal
virtual work expression with integration taken over the undeformed
configuration ($V_0$) at $t=0$. The current volume $dV_e$ and the corresponding initial
volume  ($dV_0$) are connected through the determinant of the
deformation gradient, $J=|\mathbf{F}|$ (see Section 1.9.1). Thus the above form for
$\bmf{I}_e$ may also be written as
%
\begin{equation}\label{E:IFVb-1}
\bmf{I}_e =
 \int_{V_0} \mathbf{B}^T\, \bmf{\sigma}\, J \, dV_0  =\int_{V_0} \mathbf{B}^T\, \bmf{\tau} \, dV_0
\end{equation}
%
\nid where $\bmf{\tau}= J \bmf{\sigma}$ is termed the (symmetric) Kirchhoff stress. Under 
incompressible plastic deformation, $J=1.0$, and thus quite small differences
arise between the Cauchy and Kirchhoff stresses even during the early stages of elastic-plastic
deformation in metals.

Standard integration over the deformed element volume (using Cauchy stress as in \ref{E:IFVb})
 re-cast into usual
isoparametric form leads to (for hexahedral elements)
%
\begin{equation}\label{E:IFVc}
 \bmf{I}_{e(3ne\times1)}= \int_{-1}^{1} \int_{-1}^{1} \int_{-1}^{1} 
 \mathbf{B}^T \bmf{\sigma}\, |\mathbf{J}| \, d\xi\, d\eta \,d\zeta 
\end{equation}
\nid where the parent element in parametric coordinates occupies the bi-unit
cube $-1 \le \xi, \, \eta,\, \zeta\le 1$.
%

\subsection{Strain Increments for Stress Updating}
\nid Newton's method advances the global solution from time step 
$t_n \rightarrow t_{n+1}$ through a series of
iterative improvements to the nodal displacements. Let $i$ denote the current Newton iteration
for the solution at $t_{n+1}$, $\bmf{u}_e^{(i)}$ the $i^{th}$ estimate for the element nodal 
displacements at $t_{n+1}$ and $\bmf{u}_e$ the converged solution for element 
nodal displacements at $t_n$. Using the mid-increment configuration, 
the $i^{th}$ estimate for the (mechanical) strain increment over the step is given by
%
\begin{equation}\label{E:SIa}
\Delta \bmf{\vareps}^{(i)}= \mathbf {B}_{n + 1/2}\, \Delta \bmf{u}_e^{(i)} - \Delta \bmf{\vareps}_{T}
\end{equation}
%
\nid 
where the $\mathbf{B}$ matrix is evaluated using nodal coordinates 
$\bmf{c}_e = \bmf{x}_{n+1/2}$ and $\Delta \bmf{u}_e^{(i)} =\bmf{u}_e^{(i)} - \bmf{u}_e$.
Here, $\Delta \bmf{\vareps}$ is the vector form of the symmetric 
tensor $\mathbf{D}_{n+1/2}\,\Delta t$ (shear terms in $\mathbf{D}$ are doubled).
The specified thermal strain increments 
over $t_n \rightarrow t_{n+1}$
are indicated by $\Delta \bmf{\vareps}_T$\footnote{The
implementation should be expanded to include
other (known) initial strains including the creep
strain increment.}. The strain increment $\Delta \bmf{\vareps}^{(i)}$
 is passed to the stress updating (constitutive) models 
after rotation effects are neutralized
as described in Section 1.9.

\subsection{Strains for Output}

For the small-strain formulation, output strain values are simply  the 
sum of converged increments, $\Delta \bmf{\vareps}$,
over all time increments. 

For the finite strain formulation, the output strains following convergence of load load(time)
step $n$  are given by (see Section 1.9 for definitions of unrotated strains $\Delta \mathbf{d}$
and further discussion)
%
\begin{equation}\label{E:SIab}
\overline{\mathbf{D}}_n= \mathbf {R}_n\, \left  ( \sum_{k=1}^n \Delta \mathbf{d}_k \right ) \mathbf {R}_n^T
\end{equation}
%
\nid 
where $\Delta \mathbf{d}_k$ is the converged increment of unrotated strain for load(time) step $k$.
$\mathbf{R}$ is the rotation matrix from the fixed-global axes to the axes that rotate with the
material point computed from the polar
decomposition of the deformation gradient $\mathbf{F}$.
The output strain values defined in Eq. (\ref{E:SIab}) correspond to those produced by Abaqus (Standard)
as E1, E2, etc. -- Abaqus rotates existing strain, $\bmf{\vareps}_n\rightarrow\bmf{\vareps}_{n+1}$
using an approximate $\Delta \mathbf {R}$ while adding $\Delta \bmf{\vareps}$. 
WARP3D accumulates strain increments on the coordinate
system that rotates with the material point, then rotates to the fixed-global system 
as needed for output using the above expression.

Key and Krieg [\ref{R:KK1982}] and Nagtegaal and Veldpaus [\ref{R:NV1984}] and others have demonstrated
that Eq. (\ref{E:SIa}) defines a constant rate of logarithmic strain over the step. 
In a one-dimensional 
setting, simple calculations illustrate that
integration of this strain rate to define a total strain measure using the 
mid-point rule above remains surprisingly accurate for 
large increments. 

In multi-dimensional settings, the interpretation of 
logarithmic strain holds if the principal directions of 
strain rotate to match the rigid-body rotation of the material point. This rarely 
occurs and thus accumulated increments of converged  strain values 
do not necessarily represent a conventional measure of total strain. The simple example
worked through in [\ref{R:BLME2014}] (Example 3.4)  illustrates the issue of integrating 
the rate-of-deformation for an initially unit square of material over a closed loop of shear $\rightarrow$ 
extension $\rightarrow$ reverse shear $\rightarrow$ compression that
leaves the material in the unit square. The material is undeformed 
yet  the ending value $\bmf{\vareps} = \int \dot {\bmf{\vareps}}\,dt \ne \bmf{0}$ -- the rate-of-deformation 
is not a loading path independent
measure of the material deformation (other strain measures, \eg the Green strain, do 
provide a path independent
measure).





\subsection{Tangent Stiffness Matrix}
\nid For the globally implicit solution procedure adopted in WARP3D, the
linearized form of the element nodal forces is required. Starting from Eq. (\ref{E:IFVb}), we can 
write (using the Kirchhoff stress form for $\bmf{I}_e$)
%
\begin{equation}\label{E:TSa}
 \dot{ \bmf{I}}_{e(3ne\times1)} = 
 \int_{V_0} \left [ \mathbf{B}^T \dot{\bmf{\tau}} J+\dot{\mathbf{B}}^T \bmf{\tau}J +
  \mathbf{B}^T \bmf{\tau}\dot J \right ]\, dV_0 =
 \left [ \mathbf{K}_T \right ]_e \dot{\bmf{u}}_e
\end{equation}
%
\nid 
where the $(\dot\,)$ derivative refers to real time for dynamic analyses or to a load-like
parameter for quasi-static analyses. The first term leads to the \ti{material} stiffness with the second term
defining the \ti{geometric} (or \ti{initial}) stiffness matrix (see [\ref{R:BLME2014}] Section 6.4 
and [\ref{R:C1997}] Chapter 12 for details of 
derivations of the geometric stiffness). The third term will be neglected under the assumption of
incompressibility of deformation for the material models of interest in
WARP3D; moreover, the $\dot J$ term leads to a non-symmetric tangent stiffness if retained.
 As a consequence, the  first two integrals above maybe evaluated over the 
current configuration ($dV_e=J dV_0$) using the Cauchy stress. The result  will be written in the form
%
\begin{equation}\label{E:TSb}
 \left [ \mathbf{K}_T \right ]_e =  \left [ \mathbf{K}_{mat} \right ]_e +  \left [ \mathbf{K}_{geo} \right ]_e\ .
\end{equation}
%

In Section 1.9, objective stress rates are introduced to provide kinematically correct descriptions of
$\dot{\bmf{\sigma}}$. A motion $\dot{\bmf{u}}_e\Delta t$ corresponding to rigid
rotation at a material point produces $\dot{\bmf{\sigma}}\ne \mathbf{0}$. Objective stress
rates introduce the product of rotation (rate) matrices with the Cauchy stress tensor leading to additional
stiffness terms that may be included in  $ \left [ \mathbf{K}_{mat} \right ]_e$
or $ \left [ \mathbf{K}_{geo} \right ]_e$. In WARP3D, we include the rotation terms from use of
an objective stress rate in $ \left [ \mathbf{K}_{mat} \right ]_e$ to retain the essentially standard form
for $ \left [ \mathbf{K}_{mat} \right ]_e$ discussed next.


The symmetric geometric stiffness has this form (recall the ordering of  terms in $\bmf{u}_e$ and $\bmf{\sigma}$
described previously)
%
\begin{equation}\label{E:TSc}
 \left [ \mathbf{K}_{geo} \right ]_e =  
 \int_{Ve} \mathbf{G}^T \,\hat{\bmf{\Gamma}}^T\,\mathbf{M}_{\sigma}\,\hat{\bmf{\Gamma}} \,\mathbf{G} \, dV_e
\end{equation}
%
\nid where 
%
\begin{equation}\label{E:TSd}
 \left [ \mathbf{M}_{\sigma} \right ] =  
  \begin{bmatrix}[1.0]
\sigma_1 \mathbf{I}_3\ \sigma_4 \mathbf{I}_3 \ \sigma_6 \mathbf{I}_3  \\
 \sigma_4 \mathbf{I}_3\ \sigma_2 \mathbf{I}_3\ \sigma_5 \mathbf{I}_3  \\
\sigma_6 \mathbf{I}_3\ \sigma_5 \mathbf{I}_3\ \sigma_3 \mathbf{I}_3  
   \end{bmatrix}_{9\times9}
  \end{equation}
%
\nid  and $\mathbf{I}_3$ is a $3 \times 3$ identity matrix.

The material stiffness has the familiar form
%
\begin{equation}\label{E:TSe}
 \left [ \mathbf{K}_{mat} \right ]_e =  
 \int_{Ve} \mathbf{B}^T\, \hat{\mathbf{E}}\, \mathbf{B} \, dV_e
\end{equation}
%
\nid where the $6 \times 6$ constitutive matrix $\hat{\mathbf{E}}$ relates rates of the spatial Cauchy
stress to rates of the spatial deformation, $\dot{\mathbf{D}}$, expressed in vector 
form as $\dot{\bmf{\varepsilon}}$,
%
\begin{equation}\label{E:TSf}
 \dot{\bmf{\sigma}} = \hat{\mathbf{E}}\,\dot{\bmf{\varepsilon}} = 
 \hat{\mathbf{E}}\, \mathbf{B}\, \dot{\bmf{u}}_e\ .
\end{equation}
%
In geometrically linear analyses, $\hat{\mathbf{E}}$ takes on the usual form reflecting linear-elasticity
$\hat{\mathbf{E}}= \mathbf{E}$ or 
(small-strain) material nonlinear response, $\hat{\mathbf{E}}= \mathbf{E}_T$.

For geometrical nonlinear analysis, the spatial rate of 
Cauchy stress does not necessarily vanish when $\dot{\bmf{\vareps}}\rightarrow \bmf{0}$,
\ie simple rigid rotation at a material point renders $\dot{\bmf{\sigma}}\ne\bmf{0}$.
Consequently, a rotation neutralized (objective) stress rate must be adopted in the
proper construction of $\hat{\mathbf{E}}$. A number of objective stress rates may be defined to 
accommodate rotation effects on $\dot{\bmf{\sigma}}$ -- different objective 
rates arising from assumptions to compute the rotation rate at a material point (\eg
Section 3.7 of [\ref{R:BLME2014}]  and the recent, extensive discussion [\ref{R:JWB2013}]).
WARP3D adopts the Green-Naghdi (objective) stress rate ([\ref{R:JB1984}]
and references therein) 
expressed using the concept of a corotational stress and a work-conjugate rate-of-deformation  
to formulate a symmetric $\hat{\mathbf{E}}$ (see Section 3.4.2 of [\ref{R:BLME2014}] and also the 
motivating work by Dienes [\ref{R:D1979}]).

Upon combining Eqs. (\ref{E:TSc}, \ref{E:TSe}), the element tangent stiffness matrix is computed using
standard numerical integration with
%
\begin{equation}\label{E:TSg}
 \left [ \mathbf{K}_T \right ]_e= \int_{-1}^{1} \int_{-1}^{1} \int_{-1}^{1} 
 \left [ \mathbf{B}^T \hat{\mathbf{E}} \mathbf{B} +
  \mathbf{G}^T \,\hat{\bmf{\Gamma}}^T\,\mathbf{M}_{\sigma}\,\hat{\bmf{\Gamma}} \,\mathbf{G} \right]
    \, |\mathbf{J}| \, d\xi\, d\eta \,d\zeta \ .
\end{equation}
%
\nid All deformation dependent quantities here refer to values for the current, $i^{th}$ iteration estimate of the 
nonlinear solution at time (load step) $n+1$: $\mathbf{B}$ is evaluated using the updated nodal coordinates
$\bmf{x}_{n+1}^i$; the Cauchy stresses appearing in $\mathbf{M}_{\sigma}$ are $\bmf{\sigma}_{n+1}^i$;
and $\hat{\mathbf{E}}$ is the material tangent modulus which advances the spatial rate of Cauchy stress
from $n\rightarrow n+1$ \ti{consistent} with the material stress updating procedure for the strain
increment given in Eq. (\ref{E:SIa}) and and which accommodates correctly the rigid-rotation rate at the material
point.

This formulation adopted in WARP3D and that used in Abaqus/Standard for globally 
implicit solutions differ in the choice of objective stress
rate: WARP3D uses the Green-Naghdi rate, Abaqus/Standard uses the Jaumann rate of Cauchy stress.
Abaqus/Explicit uses the Green-Naghdi
rate building on the prior implementation in PRONTO3D [\ref{R:TF1989}]. Hallquist [\ref{R:H1984}] 
appears to have been the first to adopt the 
Green-Naghdi rate for a globally implicit solution framework building on the work of Dienes [\ref{R:D1979}].

\subsection{Mass Matrix}
\nid The element consistent mass matrix follows from virtual work of the acceleration 
(inertial) force term in Eqs. (1.4.3, 1.4.4) given by
%
\begin{equation}\label{E:Ma}
\delta W_{accel}=\int_{V_e} \delta \bmf{d}^T \rho\, \ddot{\bmf{d}}\, d V_e
\end{equation}
%
\nid where integration in this form occurs over the deformed element
configuration $V_e$ at time $t$ and thus $\rho$ denotes the mass
density in the deformed configuration. Using Eq. (\ref{E:IF1}), this may be 
written as
%
\begin{equation}\label{E:Mb}
\int_{V_e} \delta \bmf{d}^T \rho\, \ddot{\bmf{d}}\, d V_e =
\delta \bmf{u}_e^T \left [ \int_{V_e}  \hat{\bmf{N}}^T \hat{\bmf{N}} 
\rho\, d V_e \right ] \ddot{\bmf{u}}_e\ .
\end{equation}
%
\nid The consistent mass matrix for the element is recognized as
%
\begin{equation}\label{E:Mc}
\mathbf{M}_e =  \int_{V_e}  \hat{\bmf{N}}^T \hat{\bmf{N}} 
\rho\, d V_e = \int_{-1}^1 \int_{-1}^1 \int_{-1}^1\hat{\bmf{N}}^T \hat{\bmf{N}} 
\rho\, |\mathbf{J}|\, d \xi\, d\eta\, d\zeta
\end{equation}
%
where $ |\mathbf{J}|$ is evaluated using nodal coordinates at $t$.
Considering the block diagonal structure of Eq. (\ref{E:IF5}), the element consistent 
mass matrix also exhibits the same form. Thus,
computation is needed for only 
the block diagonal mass matrix corresponding to one of the 
three nodal accelerations and to assign 
this matrix to the other two nodal accelerations.

The mass density $\rho$ appearing in the above equations corresponds to the 
current configuration as the inertial body force acts there. 
This may be expressed in terms of the mass density in the undeformed ($t=0$) 
configuration using
%
\begin{equation}\label{E:Md}
\rho_0 = \rho\, | \mathbf{F}|
\end{equation}
%
where $|\mathbf{F}|$ denotes the determinant of the deformation gradient,
$\mathbf{F} = \partial \bmf{x} / \partial  \bmf{X}$ (see Section 1.9). 
Using the relation $dV_e =|\mathbf{F}| d V_0$,
 the element consistent mass matrix may be expressed using quantities for the the $t=0$
 configuration
%
\begin{equation}\label{E:Me}
\mathbf{M}_e =   \int_{-1}^1 \int_{-1}^1 \int_{-1}^1\hat{\bmf{N}}^T \hat{\bmf{N}} 
\rho_0\, |\mathbf{J}_0|\, d \xi\, d\eta\, d\zeta
\end{equation}
%
where $|\mathbf{J}|$ is the usual determinant of the coordinate Jacobian at $t=0$. 
The element consistent mass matrix defined above becomes invariant of time.
Consequently, the matrix $\mathbf{M}_e$ is computed only once for an analysis during the initial setup 
operations for the first time step.

For computations in WARP3D, the consistent mass is converted to diagonal form ($3ne \times 1$), 
$\overline {\bmf{M}}_e$, through a simple procedure 
that conserves total mass of the element  (see  [\ref{R:HRZ1976},\ref{R:ZTZ2013}] for discussion of other alternatives).
The element mass is given by 
%
\begin{equation}\label{E:Me2}
m_e =   \int_{-1}^1 \int_{-1}^1 \int_{-1}^1\rho_0\, |\mathbf{J}_0|\, d \xi\, d\eta\, d\zeta~.
\end{equation}
%
The $ne \times 1$ diagonal terms, $\bmf{m}_e$,  are then computed with
%
\begin{equation}\label{E:Me3}
\left (\bmf{m}_e\right )_{i} =   \int_{-1}^1 \int_{-1}^1 \int_{-1}^1\rho_0\, 
N_i N_i \,|\mathbf{J}_0|\, d \xi\, d\eta\, d\zeta
\end{equation}
%
and the trace $\overline m_e= \sum \bmf{m}_e$.  The final form of $\overline {\bmf{M}}_e$ that
preserves total element mass is given by
%
\begin{equation}\label{E:Me3}
\overline {\bmf{M}}_e = \frac{m_e}{\overline m_e}
\begin{Bmatrix}[1.2]
\bmf{m}_e  \\  
\bmf{m}_e\\  
\bmf{m}_e\end{Bmatrix}_{3ne\times 1}~.
\end{equation}
The global  lumped mass matrix is found through the usual assembly of the element matrices.
%*****************************************************
\subsection {References}
%*****************************************************
\small

\noindent[\refstepcounter{sectrefs}\label {R:KK1982}\thesectrefs]~S.W. Key and R.D. Krieg.
On the Numerical Implementation of Inelastic Time Dependent and Time Independent, 
Finite Strain Constitutive Equations in Structural Mechanics. \ti{Computer Methods in Applied Mechanics and
Engineering}, Vol. 33, 1982, pp. 439-452.

\medskip
\noindent[\refstepcounter{sectrefs}\label {R:NV1984}\thesectrefs]~J.C. Nagtegaal, and F.E. Veldpaus.
On the Implementation of Finite Strain Plasticity Equations in a Numerical Model.
In Numerical Analysis of Forming Processes (edited by J.F. Pittman, O. C. Ziekiewicz, R. D. Wood 
and J. M. Alexander), p. 351. John Wiley and Sons, New York, 1984.

\medskip
\noindent[\refstepcounter{sectrefs}\label {R:HRZ1976}\thesectrefs]~E. Hinton, T. Rock, and O.C. Zienkiewicz.
A Note on Mass Lumping and Related Processes in the Finite Element Method. \ti{Earthquake Engineering and
Structural Dynamics}, Vol. 4, No. 3, 1976, pp 245-249.

\medskip
\noindent[\refstepcounter{sectrefs}\label {R:ZTZ2013}\thesectrefs]~O.C. Zienkiewicz, R.L. Taylor, 
and J.Z. Zhu. The Finite Element Method: Its Basis \& Fundamentals. 7th Ed. Butterworth-Heinemann, 
Waltham, MA, 2013.

\medskip
\noindent[\refstepcounter{sectrefs}\label {R:BLME2014}\thesectrefs]~T. Belytschko, W.K. Liu,
B. Moran, K.I. Elkhodary. Nonlinear Finite Elements for Continua and Structures. 2nd Ed. John Wiley \& Sons,
New York, 2014.

\medskip
\noindent[\refstepcounter{sectrefs}\label {R:C1997}\thesectrefs]~M. Crisfield. Nonlinear Finite Element Analysis of Solids and Structures.Volume 2: Advanced Topics. John Wiley \& Sons,
New York, 1997.

\medskip
\noindent[\refstepcounter{sectrefs}\label {R:D1979}\thesectrefs]~J.K. Dienes. On the Analysis of 
Rotation and Stress Rate in Deforming Bodies. \ti{Acta Mechanica}, Vol. 32, 1979, pp. 217-232.

\medskip
\noindent[\refstepcounter{sectrefs}\label {R:H1984}\thesectrefs]~J.O. Hallquist. NIKE 3D - A Vectorized, 
Implicit, Finite Deformation, Finite-Element Code for Analyzing the Static and 
Dynamic Response of 3-D Solids. Lawrence Livermore Laboratory Report UCID-18822, 1984.

\medskip
\noindent[\refstepcounter{sectrefs}\label {R:JB1984}\thesectrefs]~C.C. Johnson and D.J. Bammann.
A Discussion of Stress Rates in Finite Deformation Problems. \ti{International Journal for Solids and Structures},
 Vol. 20, 1984, pp. 725-737.

\medskip
\noindent[\refstepcounter{sectrefs}\label {R:TF1989}\thesectrefs]~L.M. Taylor and D.P. Flanagan.
PRONTO 3D: A three-dimensional transient solid dynamics program. SAND-87-1912 ON: DE89010517,
OSTI ID: 6212624, 1989.

\medskip
\noindent[\refstepcounter{sectrefs}\label {R:JWB2013}\thesectrefs]~W. Ji, A.M. Waas, and M.Z. Bazant. 
On the importance of work-conjugacy and objective stress rates in finite deformation incremental finite element analysis.
\ti{Journal of Applied Mechanics},  Vol. 80(4), 2013. 9pgs.

 
 \end{document}

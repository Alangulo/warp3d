 %!TEX TS-program = pdflatexmk
 %
\documentclass[11pt]{report}
\usepackage{geometry} 
\geometry{letterpaper}

%---------------------------------------------
\setlength{\textheight}{630pt}
\setlength{\textwidth}{450pt}
\setlength{\oddsidemargin}{14pt}
\setlength{\parskip}{1ex plus 0.5ex minus 0.2ex}


%----------------------------------------
\usepackage{amsmath}
\usepackage{layout}
\usepackage{color}
\usepackage{hyphenat}

%----------------------------------------------
\usepackage{fancyhdr} \pagestyle{fancy}
\setlength\headheight{15pt}
\lhead{User's Guide - \textit{WARP3D}}
\rhead{\textit{Compute}}
\fancyfoot[L] {\textit{Chapter {\thechapter}\ \   (Updated: 12-12-2012)        }}
\fancyfoot[C] {\thesection-\thepage}
\fancyfoot[R] {\textit{Model Definition}}

%---------------------------------------------------
\usepackage{graphicx}
\usepackage[labelformat=empty]{caption}
\numberwithin{equation}{section}

%---------------------------------------------
%     --- make section headers in helvetica ---
%
\frenchspacing
\usepackage{sectsty} 
\usepackage{xspace}
\allsectionsfont{\sffamily} 
\sectionfont{\large}
\usepackage[small,compact]{titlesec} % reduce white space around sections
%
%----------------------------------------------

%---------  local commands ---------------------

\newcommand{\nin} {\noindent}
\newcommand{\noi}{\noindent}
\newcommand{\HRule}{\rule{\linewidth}{0.5mm}}
\newcommand{\bmf } {\boldsymbol }
\newcommand{\bsf } [1]{\textrm{\textit{#1}}\xspace}
\newcommand{\ul} {\underline}
\newcommand{\hv} {\mathsf}   %helvetica text inside an equation
\newcommand{\eg}{\emph{e.g.},\xspace}
\newcommand{\ie}{\emph{i.e.},\xspace}
\newcommand{\vs}{\emph{vs.}\xspace}
\newcommand{\ti}{\emph}
\newcommand{\veps}{\varepsilon}
\newcommand{\ol}{\overline}
\newcommand{\mdash}{\mbox{-}}
\newcommand{\clrr}{\color{red}}
\newcommand{\clrb}{\color{black}}
\newcommand{\Fig}{Fig.\xspace}
\newcommand{\Figs}{Figs.\xspace}




%
%
%        optional definition for bullet lists which
%        reduces white space.
%
\newcommand{\squishlist}{
 \begin{list}{$\bullet$}
  { \setlength{\itemsep}{0pt}
     \setlength{\parsep}{3pt}
     \setlength{\topsep}{3pt}
     \setlength{\partopsep}{0pt}
     \setlength{\leftmargin}{1.5em}
     \setlength{\labelwidth}{1em}
     \setlength{\labelsep}{0.5em} } }

\newcommand{\squishlisttwo}{
 \begin{list}{$\bullet$}
  { \setlength{\itemsep}{0pt}
     \setlength{\parsep}{0pt}
    \setlength{\topsep}{0pt}
    \setlength{\partopsep}{0pt}
    \setlength{\leftmargin}{2em}
    \setlength{\labelwidth}{1.5em}
    \setlength{\labelsep}{0.5em} } }

\newcommand{\squishend}{
  \end{list}  }
%

%-------------------------------------
\newcounter{sectrefs}
\setcounter{sectrefs}{0}
\setcounter{chapter}{2}
\setcounter{section}{10}
\setcounter{figure}{0}
\renewcommand{\thefigure}{\thesection.\arabic{figure}}

%--------------------------------------
%--------------------------------------
%---------------------------------------

\begin{document}

\section{Compute Requests}

\noindent \bf{Solution For Load Steps}\rm

\nin The nonlinear (and dynamic) solution for a series of one or more load steps is requested with the command
\begin{align*}
& \hv{\ul{comp}ute\ \ul{displ}acements\ (\ul{for})\ \ul{load}ing<nonlinear\ load\ id: label>(\ul{for})\ \ul{step}s
<integer\ list>}
\end{align*}
\nin A comma may be used anywhere in the line for continuation. WARP3D 
compares the last step number solved against the list of steps provided. 
A list of steps for computation is generated from this process and computations 
initiated. For example, let steps 1-10 be analyzed in the first compute command. 
The second \ti{compute} command then requests computation for steps 20-25. 
WARP3D automatically inserts steps 11-19 into the list of steps for computation.

WARP3D verifies the data provided in this command for correctness, \eg the 
nonlinear load must exist and the steps requested must be defined in that 
load step. When errors are encountered, the command is ignored 
and a new input line read.

Once this command is accepted and computations begin, the user 
cannot intervene in the solution process until the analysis for 
all steps in the list is completed.

Examples of \ti{compute} commands are:
\small
\begin{verbatim}
   compute displacements load test steps 1-20
   compute displacements for loading crush for steps 15-30
\end{verbatim}
 \normalsize

\noindent \bf{Domain Integral ($J$) and Interaction Integrals ($I$I)}\rm

\nin Once the solution for a load step is available, a domain integral evaluation 
to compute the $J$-integral may be requested. For linear-elastic analyses, a 
family of interaction integrals are available to compute mixed-mode 
stress intensity factors and the non-singular $T$-stresses. The domain(s) 
for computation must be defined immediately prior to the compute request. 
Chapter 4 describes commands to define domains 
for $J$ computation. The compute command has the form
\begin{align*}
&\ \ \hv{\ul{compute}}\ \left [
\begin{Bmatrix}
\hv{  \ul{domain}} \\ \hv{\ul{interaction}} 
\end{Bmatrix}
\right ] \hv{(integral)}
\end{align*}

% ----------------- end of real text ----------------------------------
\end{document}

 


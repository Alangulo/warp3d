%!TEX TS-program = pdflatexmk
 %
%
\documentclass[11pt]{report}
\usepackage{geometry} 
\geometry{letterpaper}

%---------------------------------------------
\setlength{\textheight}{630pt}
\setlength{\textwidth}{450pt}
\setlength{\oddsidemargin}{14pt}
\setlength{\parskip}{1ex plus 0.5ex minus 0.2ex}


%----------------------------------------
\usepackage{amsmath}
\usepackage{layout}
\usepackage{color}
\usepackage{hyphenat}
\usepackage{array}
\usepackage{hyperref }
%----------------------------------------------
\usepackage{fancyhdr} \pagestyle{fancy}
\setlength\headheight{15pt}
\lhead{\small{User's Guide - \textit{WARP3D}}}
\rhead{\small{\textit{Output Requests}}}
\fancyfoot[L] {\textit{\small{Chapter {\thechapter}}\ \   (Updated: 2-3-2015)}}
\fancyfoot[C] {\small{\thesection-\thepage}}
\fancyfoot[R] {{\small{\textit{Model Definition}}}}

%---------------------------------------------------
\usepackage{graphicx}
\usepackage[labelformat=empty]{caption}
\numberwithin{equation}{section}

%---------------------------------------------
%     --- make section headers in helvetica ---
%
\frenchspacing
\usepackage{sectsty} 
\usepackage{xspace}
\allsectionsfont{\sffamily} 
\sectionfont{\large}
\usepackage[small,compact]{titlesec} % reduce white space around sections
%
%----------------------------------------------

%---------  local commands ---------------------
\newcommand{\tb} {\textbf}
\newcommand{\nf} {\normalfont}
\newcommand{\df} {\dotfill}
\newcommand{\nin} {\noindent}
\newcommand{\bmf } {\boldsymbol }  %bold math symbol
\newcommand{\bsf } [1]{\textrm{\textit{#1}}\xspace}
\newcommand{\ul} {\underline}
\newcommand{\hv} {\mathsf}   %helvetica text inside an equation
\newcommand{\eg}{\emph{e.g.},\xspace}
\newcommand{\ti}{\emph}
\newcommand{\bardelta}{\bar \delta}
\newcommand{\barDelta}{\bar \Delta}
\newcommand{\veps}{\varepsilon}
\newcommand{\nid}{\noindent}

\newcommand{\tl}{\textless\xspace}
\newcommand{\tg}{\textgreater\xspace}

\newcommand{\HRule}{\rule{\linewidth}{0.5mm}}
\newcommand{\ie}{\emph{i.e.},\xspace}
\newcommand{\vs}{\emph{vs.}\xspace}
\newcommand{\ol}{\overline}
\newcommand{\mdash}{\mbox{-}}
\newcommand{\clrr}{\color{red}}
\newcommand{\clrb}{\color{black}}
\newcommand{\Fig}{Fig.\xspace}
\newcommand{\Figs}{Figs.\xspace}

\newcommand{\q}{\qquad}


\newenvironment{offsetpar}[1]%
{\begin{list}{}%
         {\setlength{\leftmargin}{#1}}%
         \item[]%
}
{\end{list}}

%
%
%        optional definition for bullet lists which
%        reduces white space.
%
\newcommand{\squishlist}{
 \begin{list}{$\bullet$}
  { \setlength{\itemsep}{0pt}
     \setlength{\parsep}{3pt}
     \setlength{\topsep}{3pt}
     \setlength{\partopsep}{0pt}
     \setlength{\leftmargin}{1.5em}
     \setlength{\labelwidth}{1em}
     \setlength{\labelsep}{0.5em} } }

\newcommand{\squishlisttwo}{
 \begin{list}{$\bullet$}
  { \setlength{\itemsep}{0pt}
     \setlength{\parsep}{0pt}
    \setlength{\topsep}{0pt}
    \setlength{\partopsep}{0pt}
    \setlength{\leftmargin}{2em}
    \setlength{\labelwidth}{1.5em}
    \setlength{\labelsep}{0.5em} } }

\newcommand{\squishend}{
  \end{list}  }
%

%-------------------------------------
\newcounter{sectrefs}
\setcounter{sectrefs}{0}
\setcounter{chapter}{2}
\setcounter{section}{11}

%--------------------------------------
%--------------------------------------
%---------------------------------------

\begin{document}

\section{Output Requests}
%\layout

\nin The output command provides computational results and model information in eight forms:
\small
\squishlist
\item \ul{conventional printed output:} with optional page and column headers using a narrow
(80) or wide (132) column format. Exponential formatting of floating point
values is available as an option. Output is directed to the window executing WARP3D or to a
text file (ASCII).
\item  \ul{packets:} a sequential binary (unformatted)  file of computed 
results organized into structured packets. Written using Fortran I/O.
A very convenient output form  to support development of customized
post-processing programs that search through results, compute secondary
quantities, build input files for plotting software, etc.  
\item \ul{Patran compatible nodal result files:} of displacements, velocities, accelerations, temperatures,
 and averaged values of  strains, stresses in either binary (unformatted) or ASCII (text)
formats. These sequential files are readable without modification by 
Patran\footnote{MSC Software Corporation.\nolinkurl{www.mscsoftware.com}}
for post-processing and include the required header lines.
The unformatted files are written using Fortran I/O and have the usual, compiler-dependent
metadata defining records.
\item \ul{Patran compatible element result files:} of strains, stresses at the element center in either binary 
(unformatted) or ASCII (text)
formats. These sequential files are readable without modification by Patran for post-processing
 and include the required header lines.
The unformatted files are written using Fortran I/O and have the usual, compiler-dependent
metadata defining records.
\item \ul{Flat text files of results compatible with Python, Excel, and Python:} (1) nodal values
of displacements, velocities, accelerations, temperatures, and (averaged across elements) strains, stresses,
and (2)  element center values of  strains, stresses, and internal material model state variables
(\eg elastic strain components in a plasticity material model). Simple flat text files suitable for reading directly, for
example with \ti{loadtxt} in Python, importing into Excel and reading with standard file I/O 
operators in Matlab, Mathematica, C, C++, Fortran. Files may contain comment lines as desired. 
On Linux and OS X versions of WARP3D, the option exists to also write only the compressed (.gz)
results files (Python \ti{loadtxt} automatically uncompresses the .gz files).
\item \ul{Flat results files in stream (binary) form:} (1) nodal values
of displacements, velocities, accelerations, temperatures, and (averaged across elements)  strains, stresses,
and (2)  element center values of  strains, stresses, and internal material model state variables
(\eg elastic strain components in a plasticity material model). WARP3D writes these
files with Fortran 2003 code using \ti{access=stream}
which omits the usual record structuring information 
included in \ti{access=sequential}. The file is thus compatible with standard
binary read operations in C and C++ (and Fortran 2003). In Python the file may
be read directly using \ti{fromfile}. Files do not contain any header lines
or sizing information, only numerical results. Other programs can also readily read 
these flat stream files including
Matlab, Mathematica and others.
\item \ul{Patran neutral file (ASCII) for the model}: the standard, packet-based file that
includes model sizes, nodal coordinates, 
element types, associated material, incidences, and (absolute) constraints.
\item \ul{Flat file with model description}: includes model sizes, nodal coordinates, 
element types, input material number, and incidences. Similar in purpose to a Patran neutral file
but without the packet-based structure. Options are available for both ASCII \ti{text} 
and binary \ti{stream} files. The \ti{text} file may be compressed into \ti{.gz} form. The flat
model description file(s) are designed specifically for very efficient reading by Python programs
(see comments above about flat results files).
\item \ul{ASCII file:} summarizing various energy terms for each completed load
step.
\squishend
\normalsize

\nid Output commands may be specified using a combination
of two approaches: (1) 
commands given immediately after the \ti{compute} command for a 
load step, and/or (2) a separate file of \ti{output}
commands that will be executed
automatically after completion of each step in a specified list of load (time) 
steps.
This second form is especially useful for analyses with a large
number of steps during which the same output commands
are desired after all load steps or after load steps at some
regular frequency, \ie after every 10 steps.
The two output approaches may be mixed together to achieve
the desired result output, \eg use (2) for output following a pattern of 
load steps or every step and (1) to output at additional results 
after less frequent steps.

In the first approach, \ti{output} commands are given immediately 
after the \ti{compute} command for a 
load step -- WARP3D does not retain results for 
other than the current load step. For example,
\small
\begin{verbatim}
   .
  compute displacements loading test for steps 10-20
  output displacements nodes 100-20 by -2
  compute displacements loading test for steps 21-30
   .
 \end{verbatim}
 \normalsize
The \ti{output} command above prints results only for load step 20. No
results are printed for steps 10-19 using the above command sequence. However
\ti{packet} type 
output maybe generated during solution for steps 10-19 as described
subsequently.

In the second approach, a file containing \ti{output} commands (and comment lines)
is specified with a list of load steps, \eg
\small
\begin{verbatim}
   .
  output commands file "get_ouput" after steps 10-1000 by 10
  compute displacements loading test for steps 1000
   .
 \end{verbatim}
 \normalsize
\nid The \ti{output} commands contained in file \ti{get\_output} are
executed at completion of load steps 10, 20, 30, ... 1000. The commands file
may contain as many \ti{ouput} commands described in this section as
needed. The full description of the command is
\begin{align*}
& \hv{\ul{output}\ {\ul{comm}ands}\  (\ul{use})\ (\ul{file})\ \ <file name:string>
\ (\ul{after})\ \ul{step}s\  <integerlist>  }
\end{align*}
\nid Note that the keyword \ti{steps} is required before the list of step numbers.

\nid The \ti{output commands file ...} may be repeated as needed in the model definition -- only
the last specified file name and list of steps are retained
 and are available across an analysis restart. Additional (separate)  \ti{output} commands may be
 specified after \ti{compute} commands as needed.
\subsection{Printed Output}
\nin
The command to request printed results directed to the current
output device has the form
\begin{align*}
& \hv{\ul{output}\ \ [<options>]\ <quantity>\ (\ul{for})  }
\begin{Bmatrix}
\hv{\ul{node}s} \\ \hv{\ul{elem}ents} 
\end{Bmatrix} \ \hv{<integer\ list>}
\end{align*}

\nin
where \tl{quantity}\tg is \ti{one} of the following types of results: 
\ti{\ul{displ}acements},
\ti{\ul{velo}cities}, 
\ti{\ul{accel}erations}, \ti{\ul{temp}eratures}, \ti{\ul{strain}s}, 
\ti{\ul{stress}es}, \ti{\ul{react}ions}. The repeatable  \tl{options}\tg include
\ti{\ul{wide}}, \ti{\ul{eform}at}, \ti{\ul{prec}ision}, \ti{\ul{nohead}er}, 
\ti{\ul{total}s}\ (\ul{\ti{only}}).

The destination for printed results is the current output device specified by the
user. The output device is either a sequential, formatted (ASCII) file or 
the terminal/command window in which WARP3D began execution. An
output file is declared using: (1) standard output via I/O redirection
(\ie the \tl \ti{file name})
convention of Windows, Linux, OS X on the program invocation line 
in the shell window, or (2) through the \ti{*output to \tl{file}\tg}command available in 
WARP3D (refer to Section 2.15 for a description of all *
commands). 

\nin More notes on use of the \ti{output} command:
\small
\squishlist
\item By default, output routines  format numerical values and headers
to fit on an 8.5 in. $\times$ 11 in page oriented in portrait mode. The \ti{wide} option
permits extension of output up to 132 columns for printing in the
landscape orientation
\item Numerical results are printed with an f12.6 format. An e12.5 format is
requested with the \ti{eformat} option. These \ti{precision} option increases these fields
to f26.16 and e26.16.
\item The \ti{noheader} option suppresses the printing of all identifying labels.
This simplifies greatly the development of post-processing type programs that
read through the ASCII result files using Python, Perl, ...
\item When a list of \ti{elements} is specified for output of displacements,
velocities, accelerations, temperatures or reactions, results are printed for the nodes
of each element in the list (not the merged set of nodes for all elements in the
list). 
\item Only lists of elements are permitted for output of strains and stresses.
\item When the \tl{integer list}\tg of elements or nodes is omitted, results are
printed for all elements or nodes of the model.
\item \ti{Reactions} are external forces required for equilibrium at constrained
nodal dof. At unconstrained nodal dof they are the remaining force imbalance
due to nonlinear response and or linear equation solving. These forces include
the effects of inertia loading on the model. Separate algebraic sums of the $X, Y,
 Z$ components of these forces are printed following the nodal results to assist
in the checking of reactions. The \ti{totals only} option suppresses printing of
reaction values for individual nodes in the node list -- only the summed values
are printed. 
\item All strain-stress quantities refer to the global Cartesian coordinate
system for the model. 
\item All displacements, velocities, accelerations, reactions are in the global
$X, Y, Z$ coordinate system even when a local coordinate system is specified at
nodes.
\item The location/number of strain points for output is specified via the element
logical properties: \ti{gausspts}, \ti{nodpts} or \ti{center\_output} (see element
descriptions in Chapter 3). Node point values are
extrapolated from the interior integration point values. The center-point values are numerical
averages of integration point values. The default output location is \ti{gausspts}.
\item The number of strain-stress items printed for each element is specified in the element
properties. The element logical property \ti{long} requests an extended
set of strain-stress results at the output points. The additional quantities
include principal values, direction cosines for principal directions, state variables provided
by the material models, etc. The \ti{short} output option in the element properties
is the default. 
\squishend
\normalsize

\nin The following tables define the printed output values:
\small
\vspace{0.1in}
\hrule
\nin \ul{\bf{Strain output}}:

\nin \ti{short} option values

\nin $\veps_{x},\, \veps_{y},\, \veps_{z},\, \gamma_{xy},\, \gamma_{yz},\, \gamma_{xz}$  
 (where $\gamma_{ij} = 2\veps_{ij}$; note ordering of transverse strains)

\nin $\veps_{eff} =\frac{\sqrt{2}}{3} \sqrt{ (\veps_x - \veps_y)^2+(\veps_y - \veps_z)^2 +(\veps_x - \veps_z)^2
+ 1.5 ( \gamma_{xy}^2 +  \gamma_{yz}^2 + \gamma_{xz}^2)}$

\nin \ti{long} option includes these additional values

\nin $I_1 = \veps_x + \veps_y + \veps_z$ 

\nin $I_2 = -\veps_{xy}^2-\veps_{yz}^2-\veps_{xz}^2 + \veps_x\veps_y+
 \veps_y\veps_z+ \veps_x\veps_z$
 
 \nin $I_3 = \veps_x \left ( \veps_y\veps_z - \veps_{yz}^2 \right ) - 
       \veps_{xy} \left ( \veps_{xy}\veps_z - \veps_{yz}\veps_{xz} \right ) +
       \veps_{xz} \left ( \veps_{xy}\veps_{yz} - \veps_y\veps_{xz} \right ) $ 
       
\nin $\veps_1 \le \veps_2 \le \veps_3$ (principal values)

\nin $\ell_1, m_1, n_1$  (direction cosines for $\veps_1$)

\nin $\ell_2, m_2, n_2$ (direction cosines for $\veps_2$)

\nin $\ell_3, m_3, n_3$ (direction cosines for $\veps_3$)

\vspace{0.1in}
\hrule
\nin \ul{\bf{Stress output}}:

\nin \ti{short} option values

\nin $\sigma_{x},\, \sigma_{y},\, \sigma_{z},\, \sigma_{xy},\, \sigma_{yz},\, \sigma_{xz}$  
 (note ordering of transverse shear stresses)

\nin $\sigma_{vm} =\frac{1}{\sqrt{2}} \sqrt{ (\sigma_x - \sigma_y)^2+(\sigma_y - \sigma_z)^2 +(\sigma_x - \sigma_z)^2
+ 6 ( \sigma_{xy}^2 +  \sigma_{yz}^2 + \sigma_{xz}^2)}$

\nin $c_1, c_2, c_3$ (state/history variables provided by material model output routine)

\nin \ti{long} option includes these additional values

\nin $I_1 = \sigma_x + \sigma_y + \sigma_z$ 

\nin $I_2 = - ( \sigma_{xy}^2+\sigma_{yz}^2+\sigma_{xz}^2 )+ \sigma_x\sigma_y+
 \sigma_y\sigma_z+ \sigma_x\sigma_z$
 
 \nin $I_3 = \sigma_x \left ( \sigma_y\sigma_z - \sigma_{yz}^2 \right ) - 
       \sigma_{xy} \left ( \sigma_{xy}\sigma_z - \sigma_{yz}\sigma_{xz} \right ) +
       \sigma_{xz} \left ( \sigma_{xy}\sigma_{yz} - \sigma_y\sigma_{xz} \right ) $ 
       
\nin $\sigma_1 \le \sigma_2 \le \sigma_3$ (principal values)

\nin $\ell_1, m_1, n_1$  (direction cosines for $\sigma_1$)

\nin $\ell_2, m_2, n_2$ (direction cosines for $\sigma_2$)

\nin $\ell_3, m_3, n_3$ (direction cosines for $\sigma_3$)
\vspace{0.1in}\hrule \vspace{0.1in}
\normalsize


\nin Examples of output commands are
\small
\begin{verbatim}
  output wide eformat precision displacements for nodes 1-300 by 2
  output stresses elements 900-1500 by 2 300-500
  output accelerations for elements 20-40 100-300 by 3
  output temperatures for nodes 3-900 by 3
\end{verbatim}
\normalsize
where values are printed in the order indicated by the \tl{integer list}\tg.

\subsection{Binary Packets File}
\nin
ASCII files of printed results as described in the above section are formatted for
direct examination by analysts using a text processor, for sending to a 
printer, for e-mailing to a collaborator, etc. However, the files are not well
suited for use by custom post-processing programs written in a programming
language (C, C++, Fortran) or a scripting language (Python, Perl). Users frequently
develop such programs to gather information, for example, to generate input files for 
plotting software.

To provide a much cleaner and directly accessible representation 
of the results for these purposes, WARP3D
can generate a sequential, binary file of results in
the form of \ti{packets}. The binary file is written via Fortran I/O as an 
unformatted file type.\footnote{Fortran unformatted I/O has a logical record 
structure. Each read/write statement processes at least one logical record
of potentially variable length which may contain one or more physical records of the file. Fortran I/O
hides the physical structuring of the file. To maintain the logical record structure,
Fortran I/O routines insert additional information in the file. Unformatted files in 
C and C++ do not have this concept of logical records. Consequently, post-processing 
codes in C/C++ must recognize and process the additional structuring information 
present in the Fortran style, unformatted files. }
A packet contains a specific type of result, \eg nodal
reactions, written over several logical records in the file. The first (header) record
of each packet has an identical format which contains the packet type (an
integer), the load step number, the Newton iteration number and the 
number of subsequent
logical records in the file for that packet. This approach enables
post-processing programs to skip easily over packets of no interest or not yet
supported by the post-processing program (\eg use the 
statement \ti{read(packets\_file)} with no listed variables to skip all information in the next logical 
record).  This approach also enables the WARP3D
developers to implement new types of packets without breaking 
already available programs for post-processing written by users.

By default, various WARP3D processors \ti{do not} write results into a packets file. The user
must first authorize packets processing and provide the name of the file to
contain the packets. The crack growth processors, for example, then insert
packets into the file at each load step without further action by the user. The \ti{output} command,
given after a \ti{compute displacement ...} command, 
provides the \ti{packet} option to request insertion of
result packets into the file, \eg displacements at a list of nodes.

The command to enable-disable the generation of packets and the name of the
packets file is part of the \ti{nonlinear solution parameters} described in Section
2.10. The command has the form
\begin{align*}
& \hv{\ul{binary}\ \ul{pack}ets }
\begin{Bmatrix}
\hv{\ul{on}} \\ \hv{\ul{off}}
\end{Bmatrix}
\hv{(\ul{file} <file\ name:string>)}
\end{align*}

\nin where \ti{file name} must be present with the \ti{on} option. If the specified file exists,
the file is opened in \ti{append} mode.
WARP3D retains the packet file name and the \ti{on}, \ti{off}
option as part of the analysis restart database. An example is 
\small
\begin{verbatim}
  nonlinear analysis parameters
     solution type direct sparse 
          .
     binary packets on file "packets_pipe_crack"
          .
\end{verbatim}
\normalsize

The \ti{output} command also provides the \ti{packets} option to have output processors insert
packets into the file. Commands to request output of packets may be intermixed
with output commands to generate printed results. The command to request packets
generation has the form
\begin{align*}
& \hv{\ul{output}\ \ul{packet}s \ <quantity>\ (\ul{for})  }
\begin{Bmatrix}
\hv{\ul{node}s} \\ \hv{\ul{elem}ents} 
\end{Bmatrix} \ \hv{<integer\ list>}
\end{align*}

\nin where \tl{quantity}\tg is \ti{one} of the following types of results: 
\ti{\ul{displ}acements},
\ti{\ul{velo}cities}, 
\ti{\ul{accel}erations}, \ti{\ul{temp}eratures}, \ti{\ul{strain}s}, 
\ti{\ul{stress}es}, \ti{\ul{react}ions}. 

\nin Notes on use of the \ti{output packets} command:

\small
\squishlist
\item When a list of elements is specified for output of displacements,
velocities, accelerations or reactions, packets are inserted for the nodes of
each element in the list (not the merged set of nodes for all elements in the
list). 
\item Only lists of elements are permitted for output of packets containing
strains and stresses. 
\item Packets for temperatures are available only for lists of nodes.
Two values are
provided for each node: (a) total temperature (initial + all increments),
(b) initial (reference) temperature defined with the \ti{initial conditions} command in
the model input.
\item When the \tl{integer list}\tg of elements or nodes is omitted, the output
processor inserts packets for all elements or nodes of the model.
\item \ti{Reactions} denote external forces required for equilibrium at constrained
nodal dof. At unconstrained nodal dof, they denote the remaining force imbalance
due to nonlinear response and/or linear equation solving. These forces include
the effects of inertia loading on the model. Separate algebraic sums of the $X, Y,
 Z$ components of these forces are inserted in packets following the nodal
results to assist in the checking of reactions. The \ti{totals only} option
given in the command before or after the word \ti{packets}
suppresses inclusion of individual node reactions in the packets; only the summed values
appear in the reaction packets.
\item All strain-stress quantities refer to the global Cartesian coordinate
system for the model. 
\item All displacements, velocities, accelerations, reactions are in the global
$X, Y, Z$ coordinate system even when a local coordinate system is specified at
nodes.
\item The location/number of strain points for the packet is specified via the element
logical properties: \ti{gausspts}, \ti{nodpts} or \ti{center\_output} (see element
descriptions in Chapter 3). Node point values are
extrapolated from the interior integration point values. The center-point values are numerical
averages of integration point values. The default output location is \ti{gausspts}.
\item The number of strain or stress values for each element is specified in the element
properties. The element logical property \ti{long} requests an extended
set of strain-stress results at the output points. The additional quantities
include principal values, direction cosines for principal directions, state variables provided
by the material models, etc. The \ti{short} output option in the element properties
is the default. 
\item Output to packets for domain integral results is invoked by including a
command in the domain definition (see Chapter 4). It is still necessary, however,
 to enable the generation of packets using the \ti{binary packets on ...}
 command described above.
\squishend
\normalsize

\nin Tables in the previous section summarize the element strain and stress
packet quantities (they are same as printed values). Examples of commands for packet
generation are
\small
\begin{verbatim}
  output packets displacements for nodes 1-300 by 2
  output packets stresses elements 900-1500 by 2 300-500
  output packets accelerations for elements 20-40 100-300 by 3
\end{verbatim}
\normalsize
\nin
Appendix F describes each type of result packet currently implemented. The
appendix also includes a complete Fortran program that demonstrates an approach to
read and process the binary packets file. Most users should find this program a
very good starting point to develop their own customized post-processor program.

\subsection{Patran Compatible Result Files}
\nin
The Patran software has extensive capabilities to post-process
finite element results. Patran reads and displays results computed
by various finite element codes and communicated  through nodal and element
result files. The simple structure of Patran results files makes
them quite popular to store/share computed results with other
modeling and post-processing software. WARP3D provides the
capability to write such Patran compatible results files to support
post-processing tasks. The command to request 
Patran compatible result files has the form
\begin{align*}
& \hv{\ul{output}\ \ul{patran}   }
\begin{Bmatrix}
\hv{\ul{binary}} \\ \hv{\ul{format}ted} 
\end{Bmatrix} 
\begin{Bmatrix}
\hv{\ul{nodal}} \\ \hv{\ul{element}} 
\end{Bmatrix} \ \hv{<quantity>}
\end{align*}
\nin
where \tl{quantity}\tg is \ti{one} of the following types of results: 
\ti{\ul{displ}acements},
\ti{\ul{velo}cities}, 
\ti{\ul{accel}erations}, \ti{\ul{react}ions}, \ti{\ul{temp}eratures}, \ti{\ul{strain}s}, 
\ti{\ul{stress}es}. 

Both \ti{binary}\footnote[1]{Patran does
not properly read \ti{binary} files on Windows at this time.} and \ti{formatted} results are
sequential files created with Fortran open statements and written with
Fortran write statements (the binary file is type \ti{unformatted} in Fortran).  Binary
files often have a substantially smaller size but may not be portable between
Windows and Linux/OS X. Formatted (ASCII) files provide the portability but with
larger sizes and further can be examined easily with standard text editors, e-mailed, etc.

WARP3D generates the Patran compatible results files with assigned names that begin with four
letters followed by the 5 digit load step number. Nodal result files begin with
the letters \ti{wn}; element result files with the letters \ti{we}. The \ti{n} and \ti{e} letter is
followed by the letter \ti{b} for binary files or the letter \ti{f} for formatted files.
The fourth letter in the file name denotes the physical quantity: \ti{d} -
displacements, \ti{v} - velocities, \ti{a} - accelerations, \ti{t} - temperatures,
\ti{r} - reaction forces, \ti{e} - strains and \ti{s} - stresses. Element 
results files are available only
for strains and stresses. For example, the file \ti{webs00005} contains element
stress results for step 5 in a binary file.

The Patran result files have the structure of a two-dimensional array.
Each row has the data for a single node or element. The array columns set the
values for each node  or element.  For nodal files of displacements, velocities,
accelerations and reactions, each node has 3 data columns for the global
$X, Y, Z$ components. 

Temperature results are available only in a \ti{nodal} results file. Two values are
provided for each node. Data column 1 has the total temperature (initial + all increments)
$T= T_0 + \sum \Delta T$.
Data column 2 has the total temperature - the
initial (reference) temperature $T-T_0$.

Figure 2.10 summarizes the data column assignments for Patran strain-stress
results files. The first six strain-stress values that appear at each model node
 are the numerical average for the contribution
of each element at the node. The strain-stress invariants, principal values and
directions are computed from these averaged nodal values. The effective strain,
Mises equivalent stress, energy density and the three material model state
variables are first extrapolated from integration points to the nodes and then
averaged. 

Element result files for strains and stresses adopt the column assignments
listed in Fig. 2.10. A single set of values given for
each element is obtained by simple averaging of integration point values within each
element. The first six strain-stress values that appear for the element are the
numerical average for the contribution of each integration point. The strain-stress
invariants, principal values and direction cosines are computed from these averaged
values. The effective strain and Mises equivalent stress are computed from the
averaged values of the components. The energy density and the three material
model state variables are the average of integration point values.

Appendix A provides skeleton Fortran programs to read the binary and formatted
forms of the nodal results files. WARP3D generated (ASCII) versions of these
files are compatible with the current version of Patran and all
versions of Patran for the past decade. The Patran results file formats are
relatively simple and enjoy widespread adoption as a common format
for sharing finite element results.

\begin{table}[htb]
\centering
\setlength{\extrarowheight}{5.0pt}
\small
\begin{tabular}[htb] { | p{1.2 in} | p{1.2 in} ||  p{1.2 in} | p{1.2 in} | }
\hline
\quad Data Column & \quad Strain Value & \quad Data Column & \quad Stress Values  \\
\hline 
\q 1 & \q  $\veps_x$ & \q  1 & \q $\sigma_x$ \\
\q 2 & \q  $\veps_y$ & \q  2 & \q $\sigma_y$ \\
\q 3 & \q  $\veps_z$ & \q  3 & \q $\sigma_z$ \\
\q 4 & \q  $\gamma_{xy}$ & \q  4 & \q $\sigma_{xy}$ \\
\q 5 & \q  $\gamma_{yz}$ & \q  5 & \q $\sigma_{yz}$ \\
\q 6 & \q  $\gamma_{xz}$ & \q  6 & \q $\sigma_{xz}$ \\
\q 7 & \q  $\veps_{eff}$ & \q  7 & \q $U_0$ \\
\q 8 & \q  $I_1$ & \q  8 & \q $\sigma_{vm}$ \\
\q 9 & \q  $I_2$ & \q  9 & \q $c_1$ \\
\q 10 & \q  $I_3$ & \q  10 & \q $c_2$ \\
\q 11 & \q  $\veps_1$ & \q  11 & \q $c_3$ \\
\q 12 & \q  $\veps_2$ & \q  12 & \q $I_1$ \\
\q 13 & \q  $\veps_3$ & \q  13 & \q $I_2$ \\
\q 14 & \q  $\ell_1$ & \q  14 & \q $I_3$ \\
\q 15 & \q  $m_1$ & \q  15 & \q $\sigma_1$ \\
\q 16 & \q  $n_1$ & \q  16 & \q $\sigma_2$ \\
\q 17 & \q  $\ell_2$ & \q  17 & \q $\sigma_3$ \\
\q 18 & \q  $m_2$ & \q  18 & \q $\ell_1$ \\
\q 19 & \q  $n_2$ & \q  19 & \q $m_1$ \\
\q 20 & \q  $\ell_3$ & \q  20 & \q $n_1$ \\
\q 21 & \q  $m_3$ & \q  21 & \q $\ell_2$ \\
\q 22 & \q  $n_3$ & \q  22 & \q $m_2$ \\
 &  & \q  23 & \q $n_2$ \\
&  & \q  24 & \q $\ell_3$ \\
&  & \q  25 & \q $m_3$ \\
&  & \q  26 & \q $n_3$ \\
\hline
\end{tabular}
\caption{\small Fig. 2.10 -- Column numbers and strain-stress values for Patran compatible 
and \ti{flat} result files (ASCII, binary and stream)}
\normalsize
\end{table}

\nin \ul{Notes on Patran result files:}
\small
\squishlist
\item strains
and stresses at nodes are obtained using a two step process: (1) extrapolation
of integration point values to element nodes and then (2) numerical averaging of
values for all elements connected to a node. Each element connected to the node has the
identical weight factor in computing the average value for a nodal result.
\item invariants, principal values, direction cosines, the effective strain 
($\veps_{eff}$), mises effective stress ($\sigma_{vm}$) and the work density ($U_0$) are 
computed using the averaged nodal results for the strain and stress
components. This leads to a likely inconsistency in nonlinear solutions. During the 
solution, stresses
at element integration points derive from nonlinear constitutive models
acting on strains computed at the integration points. Such stresses extrapolated to
the nodes and then averaged for all elements connected to a node likely
will not match the values that would have been computed by first extrapolating/averaging 
strains and then updating stresses at model nodes using the nonlinear
constitutive model. Example: consider the simple mises plasticity material model with 
perfectly plastic flow. After the extrapolation/averaging process, $\sigma_{vm}$
at some nodes may exceed the yield stress.
\item material model state/history
variables ($c_1$, $c_2$, $c_3$) are the extrapolated and 
then averaged nodal values. Depending on the material model, the extrapolation/averaging
of $c_1$, $c_2$, $c_3$ may have no meaningful interpretation.
\item it is not possible, at present, to specify a list of nodes or elements to appear in
the Patran results files. Results are written for all nodes or elements in the model.
\item interface elements.  Zero values are written in the  element results files.
 For nodal results files, interface elements make no contribution to averaged values.
 Nodes with only interface elements attached have all result values of zero.
\squishend
\normalsize

Patran opens and reads the results files which also include
the number of nodes or elements and the number of data columns.
However, there is no inherent connection between the number and ordering 
of data columns and various physical quantities, \ie Patran is unaware
that column 12 of the strains file contains the second principal
strain value. 

To provide this connection, Patran asks the user for the name of a \ti{template} file
when it opens a results file for post-processing. The (ASCII) template file
associates each data column with a symbolic name (\eg sigma\_xx) and informs
Patran of the type of value (scalar, vector, tensor) in each column. Patran then
displays the symbolic names rather than
data column numbers in menus for convenience in
post-processing tasks.

We provide a set of \ti{template} files to facilitate use of Patran for
post-processing. These are included in the
WARP3D distribution in the \ti{patran\_templates} directory.
They should be copied into the appropriate Patran directory
on your computer, usually the \ti{res\_templates} directory of the Patran
installation. 
%=============================================
\subsection{Flat Text \& Stream (Binary) Result Files}
\nin
These \ti{flat} file types are included to simplify use of Python, C, C++, Fortran
for post-processing
and importing  results into spreadsheets (the text versions).
Python is a popular, interpretive and open-source language for
engineering computations. The extensive mathematical (\eg \ti{numpy}) and 
graphics (\eg \ti{mathplotlib}) packages readily support  development of 
customized post-processing capabilities.
Compiled languages C, C++, Fortran and other mathematical software
(Matlab and Mathematica) may easily read these text and stream files.

The command to request these \ti{flat} files 
has the form
\begin{align*}
& \hv{\ul{output}\ \ul{flat}   }
\begin{Bmatrix}
\hv{\ul{text}} \\ \hv{\ul{stream}} 
\end{Bmatrix} 
(\hv{\ul{comp}ressed})
\begin{Bmatrix}
\hv{\ul{nodal}} \\ \hv{\ul{element}} 
\end{Bmatrix} \ \hv{<quantity>}
\end{align*}

\nin
where \tl{quantity}\tg is only \ti{one} of the following types of results: 
\ti{\ul{displ}acements},
\ti{\ul{velo}cities}, 
\ti{\ul{accel}erations}, \ti{\ul{react}ions}, \ti{\ul{temp}eratures}, \ti{\ul{strain}s}, 
\ti{\ul{stress}es}, \ti{\ul{states}}. Material  \ti{states} output is available only for the
\ti{element} option. No list of nodes or elements is allowed -- results 
are written for all nodes/element into the files.

WARP3D generates the flat results files with assigned names that: (1) begin with three
letters, (2)  followed by the 5 digit load step number, (3) then \ti{\_stream}
or \ti{\_text}.  The text files may be compressed with \ti{gzip} during storage and
will have the suffix \ti{.gz}.

Nodal result files begin with
the letters \ti{wn}; element result files with the letters \ti{we}. The \ti{n} 
or \ti{e} letter is
followed by the third letter in the file name that denotes the physical quantity: \ti{d} -
displacements, \ti{v} - velocities, \ti{a} - accelerations, \ti{t} - temperatures,
\ti{r} - reaction forces, \ti{e} - strains and \ti{s} - stresses. For example, 
the file \ti{wes00005\_stream} contains element
stress results for step 5 in the \ti{stream}  binary file. Similarly, 
\ti{wne001243\_text.gz} contains (averaged) nodal strain
 results for step 1243 in the text form following compression. Files of material states
 values for elements are named, for example, \ti{wem00005\_stream} and 
 \ti{wem00005\_text}, and with compression \ti{wem00005\_text.gz}.

 Element results files are available only for strains, stresses and states.

The flat result files have the structure of a two-dimensional array.
Each row has the data for a single node or element. The array columns set the
values for each node  or element.  For nodal files of displacements, velocities,
accelerations and reactions, each node has 3 data columns for the global
$X, Y, Z$ components. 

Temperature results are available only in a \ti{nodal} results file. Two values are
provided for each node. Data column 1 has the total temperature (initial + all increments).
Data column 2 has the
initial (reference) temperature defined with the \ti{initial conditions} command in
the model input.

Figure 2.10 summarizes the data column assignments for flat (and Patran) strain-stress
results files. The first six strain-stress values that appear at each model node
 are the numerical average for the contribution
of each element at the node. The strain-stress invariants, principal values and
directions are computed from these averaged nodal values. The effective strain,
Mises equivalent stress, energy density and the three material model state
variables are first extrapolated from integration points to the nodes and then
averaged. 

Element result files for strains and stresses adopt the column assignments
listed in Fig. 2.10. A single set of values given for
each element is obtained by simple averaging of integration point values within each
element. The first six strain-stress values that appear for the element are the
numerical average for the contribution of each integration point. The strain-stress
invariants, principal values and direction cosines are computed from these averaged
values. The effective strain and Mises equivalent stress are computed from the
averaged values of the components. The energy density and the three material
model state variables are the average of integration point values.

Element result files for material states require a different scheme since the number and
type of state variables are different for the various builtin material models and for a
(possibly) user supplied umat. Consider a finite element model which has multiple materials defined. 
They use, for example, the builtin material models \ti{bilinear}, \ti{mises}, 
and \ti{cp} (the crystal plasticity model). 
WARP3D then writes three header files named  \ti{states\_header\_bilinear},
\ti{states\_header\_mises\_gurson}, and  \ti{states\_header\_crystal\_plasticity}.
These are simple text files which define the number and identifying labels
for each state variable output for that material model. At each load(time) step, three
files of element states are written, one for each material model.
The result files for load (time) step 5
would then be \ti{wem00005\_text\_bilinear}, \ti{wem00005\_text\_mises\_gurson},
and  \ti{wem00005\_text\_crystal\_plasticty}.  Stream files have \ti{text}
replaced by \ti{stream}. Values are written in each file for
\ti{all} elements in the model -- not just those assigned a material with that material model.
In \ti{wem00005\_text\_bilinear}, for example, all elements assigned the 
mises/gurson or crystal plasticity material have all zero values for the bilinear state
variables. This approach maintains simplicity and consistency across various material models at
the expense of (potentially) larger  results files.

Most builtin material models provide some type of states output. See the 
evolving descriptions in Chapter 3 for the
various material models. Until these are all updated, the \ti{states\_header\_...} files
produced during output generation provide the most up-to-date description of the currently available
states variables for reading and post-processing. For example, the 
\ti{states\_header\_mises\_gurson} file as of this writing has the following content. Lines 
beginning with \# lines are comment lines. There are 6 state variables  defined 
with the short (max 8 character) label
followed by a longer description.

\small
\begin{verbatim}
#
#  header file for state variable output
#  WARP3D material: mises_gurson        
#   
#  Notes on Mises material model:
#    sig_flow - current equivalent flow stress
#    porosity - 0.0
#    state: -1 actively linear elastic
#            1 actively plastic
#            value is average over integration points
#    q - not defined for mises
#  
#  Notes on Gurson material model:
#    sig_flow - dense matrix equivalent stress
#    state - same as above for mises
#    q - mises equiv stress for continuum
#   
#
#  8 character state labels and longer descriptors
#  material model number, number of state variables 
#
     6
     1  epspls    Eq. plastic strain                                          
     2  sig_flow  See states_header file                                      
     3  H'        Plastic modulus                                             
     4  porosity  Porosity (f)                                                
     5  state     See states_header file                                      
     6  q         See states header file                                      
\end{verbatim}
\normalsize


\nin Appendix K provides skeleton Python programs to read the text and stream
files. 

\nin \ul{Notes on \ti{flat} result files:}
\small
\squishlist
\item strains
and stresses at nodes are obtained using a two step process: (1) extrapolation
of integration point values to element nodes and then (2) numerical averaging of
values for all elements connected to a node. Each element connected to the node has the
identical weight factor in computing the average value for a nodal result.
\item invariants, principal values, direction cosines, the effective strain 
($\veps_{eff}$), mises effective stress ($\sigma_{vm}$) and the work density ($U_0$) are 
computed using the averaged nodal results for the strain and stress
components. This leads to a likely inconsistency in nonlinear solutions. During the 
solution, stresses
at element integration points derive from nonlinear constitutive models
acting on strains computed at the integration points. Such stresses extrapolated to
the nodes and then averaged for all elements connected to a node likely
will not match the values that would have been computed by first extrapolating/averaging 
strains and then updating stresses at model nodes using the nonlinear
constitutive model. Example: consider the simple mises plasticity material model with 
perfectly plastic flow. After the extrapolation/averaging process, $\sigma_{vm}$
at some nodes may exceed the yield stress.
\item material model state/history
variables ($c_1$, $c_2$, $c_3$) are the extrapolated and 
then averaged nodal values. Depending on the material model, the extrapolation/averaging
of $c_1$, $c_2$, $c_3$ may have no meaningful interpretation.
\item results are written in the files for all nodes or elements in the model.
\item interface elements.  Zero values are written in the  element strain-stress
results files. Values are written for the element states files.
 For nodal results files, interface elements make no contribution to averaged values.
 Nodes with only interface elements attached have all result values of zero.
\squishend
\normalsize

%=======================================
\subsection{Patran Compatible Neutral File for Model}
\nin The Patran \ti{neutral} file is a widely adopted format
to describe the various parts of a finite element model (coordinates, constraints,
etc.). This ASCII file contains \ti{packets} of information, \eg the coordinates
of a node are stored in a packet. There are 20+ types of such
packets with model information. The simple ASCII format of the file insures
portability and ease of use. 

The \ti{patwarp} program (Appendix C) provided with the WARP3D distribution
reads Patran neutral files and generates the corresponding
WARP3D input file using the commands described in this chapter.

Occasionally a WARP3D input file is generated by some other means and a
Patran neutral file for the model is needed. The command described here 
generates the neutral file containing just the essential information for the
model.

The command to request generation of a Patran neutral file has the form
\begin{align*}
& \hv{\ul{output}\ \ul{patran} \ \ul{neutral}\ (<name\ of\ file:label\ or\ string>)  }
\end{align*}

\nin If a name for the neutral file is omitted, the default name is \ti{structure\_name.neutral}
where \ti{structure\_name} is the label supplied in the \ti{structure} command of the
WARP3D input file (\eg \ti{beam.neutral}). When the specified filename already exists, the
current time (hr:min:sec) is appended to the filename (\eg beam.neutral\_12:00:01).

The following information about the model is included in the neutral file:
\small
\squishlist
\item neutral file header records required by Patran
\item number of nodes and elements
\item coordinates of all model nodes
\item element types and incidences for all model elements (see additional info below)
\item constraints (zero and non-zero) imposed on model nodes
\squishend 
\normalsize

\noindent At present, no loading information is written to the neutral file.

\noindent The following procedures are used to write types and incidences
\small
\squishlist
\item The 8 and 20-node hex elements are
written as Patran type = 8
\item  9, 12 and 15-node hex transition elements are written as Patran type = 8 
but with zeroes in the incidence lists for the missing mid-side nodes
\item Tetrahedral elements are written as Patran type = 5
\item Interface elements \ti{inter\_8} are written as Patran type = 8
\item Interface elements \ti{trint6}  are written as Patran type = 7 (wedge)
\item Interface elements \ti{trint12}   are written as Patran type = 7 (wedge) with
15 nodes. Patran does not have a 12-node wedge element. Patran \ti{wedge15} nodes 10, 11 and 12
are written as zero
\item The Patran configuration code (CONFIG) for the
element is assigned the material number given to the element during WARP3D input. The first material 
defined in the input file is material 1, the second defined material is 2, etc.
\squishend 
\normalsize

%=======================================
\subsection{Flat Model Description Files}
\nin The Patran \ti{neutral} file described above features a highly structured system of
packets, which although very flexible, is quite slow to parse for large models using
only standard features of Python.  A key advantage is the capability to add new types of packets without
breaking existing programs.

This section describes an alternate file type for the model description
having a simple \ti{flat} format without packets and which is readable in Python using
very efficient techniques. The file may also be imported directly into 
spreadsheets. This flat file may be written in \ti{text} format with ASCII data or
binary \ti{stream} format. The \ti{text} version may be compressed into \ti{.gz} form.
The \ti{stream} format is readily processed in C, C++ programs -- the file contains only data
without the usual record marks of Fortran \ti{unformatted} files.

The command to request a \ti{flat} file for the model description  
has the form
\begin{align*}
& \hv{\ul{output}\ \ul{model}\ (\ul{flat})   }
\begin{Bmatrix}
\hv{\ul{warp3d}} \\ \hv{\ul{patran}} 
\end{Bmatrix} 
(\hv{\ul{conv}ention})
\begin{Bmatrix}
\hv{\ul{text}} \\ \hv{\ul{stream}} 
\end{Bmatrix} 
(\hv{\ul{comp}ressed)\ \ul{file}\ <name>}
\end{align*}

\nin
where the keyword \ti{file} is required and the    $<$name$>$ may be either
a $<$label$>$ or a $<$string$>$. An extension \ti{.text} or \ti{.str} is appended by WARP3D
to the file name. The compressed text file thus becomes $<$name$>$.text.gz. Compression 
is not supported for (1) \ti{stream} files, (2) the Windows version of WARP3D. 

WARP3D and Patran element types, \eg the integer number for the 20-node hex element,
and the ordering of nodes in the element 
connectivity list are not always identical. Thus the option to select between them
is included in the command as the \ti{warp3d convention} or \ti{patran convention},
where \ti{convention} is an optional word.

The file format is shown here for the \ti{.text} version. The \ti{stream}  version omits
comment lines at the beginning.
\small
\begin{verbatim}
   #
   #  Structure: .... name from input file....
   #
   #  Created: .... date ......
   #
   #  Convention: ...warp3d.... or ....patran   
   #
       <number nodes>   <number elements>
   <x> <y> <z> 
   <x> <y> <z>
         .
         .
    <element type>   <material no.>      <27 integers with incidences>
         .
         .
\end{verbatim}
\normalsize
\nin No node numbers or element numbers are included. Both node and element data
are in sequential order.    
A least one blank separates all data values. 

The $<$material no.$>$  is as follows. Suppose the model definition has three
materials defined named \ti{mat\_1}, \ti{mat\_2}, \ti{mat\_3} (and in that
physical order in the input commands) The value of  $<$material no.$>$ is then 1, 2, 
or 3 depending on the material assigned to the element in the input.

Every element line has 27 entries for the incidences (nodes connected to the element).
Many of them will be zero for the simpler element types. The 8-node hex element 
\ti{l3disop} has 8
node numbers listed followed by 19 zeroes. The ordering of nodes listed in the element
connectivity list depends on the selection of \ti{warp3d} or \ti{patran}. See the previous section
for more details on Patran element types.

WARP3D element types are:
\small
\begin{verbatim}
     1     20-node hex
     2      8-node hex
     3     12-node hex transition
     4     15-node hex transition
     5      9-node hex transition
     6     10-node tetrahedron
     7     15-node wedge   (implementation in-progress)
    12      8-node interface
    13      4-node tetrahedron
    14      6-node triangular interface
    15     12-node triangular interface
    17      6-node wedge (implementation in-progress)
\end{verbatim}
\normalsize

\subsection{Energy File}
\nin Once the solution for a load step has converged, WARP3D appends newly updated values
of various energy quantities for the model. This file is named \ti{energy} and is
located in the directory in which WARP3D was initiated. On restart analyses, WARP3D opens
the energy file in append mode and adds additional results as computations continue.

\subsection{Convert Binary Packets to ASCII}
\nin The unformatted (binary) sequential file of packets is portable between Linux and OS X systems. 
Such portability does not exist between Windows  and  Linux/OS X systems. To provide some
increased portability, the \ti{convert binary packets} command is available to write an ASCII version of the
binary packets file. The command has the form

\small
\begin{align*}
& \hv{\ul{conv}ert\ (\ul{bin}ary)\  (\ul{pack}ets) }
\begin{Bmatrix}
\hv{[<packet\ number:integer>]} \\ \hv{\ul{all}} 
\end{Bmatrix} 
 \ \hv{(\ul{to})}\ \hv{(\ul{file})}\ \hv{<label\ or\ string>}
\end{align*}
\normalsize

\nin A list of specific packet types may be specified or \ti{all} packet types -- the default if no packet list
is specified. If omitted, the default name for the generated ASCII file is the name of the binary packets
file with the extension \texttt{.tpf}. An existing ASCII file with the specified name is overwritten.  An example is
\small
\begin{verbatim}
  convert binary packets 1-10 20 22 to file 'converted_file.tpf'
\end{verbatim}
\normalsize

\nin A summary of 
converted packets prints upon completion of the operation.  The following table lists packet types
converted at present with the command. Other packet types if present are ignored.

The \ti{convert binary packets} command
is intended to given immediately prior to the \ti{stop} command during the analysis. 


\begin{table}[htb]
\centering
\setlength{\extrarowheight}{5.0pt}
\small
\begin{tabular}[htb] { | p{2.2 in} | p{0.3 in} ||  p{2.2 in} | p{0.3 in} | }
\hline
\ \  Packet Type &  No. &  \ \  Packet Type &  No.  \\
\hline 
 nodal displacements &   1 & CTOA growth: non-constant release & 18 \\
nodal velocities & 2 &  CTOA: constant release & 19 \\
nodal accelerations & 3 & killed Gurson elements & 20 \\
nodal reactions & 4 & killed SMCS elements &21 \\
Gurson elements: no adaptive & 5 & SMS elements: no load reduction & 22 \\
Gurson elements: adaptive & 6 & SMS elements: auto load reduction & 23 \\
cohesive elements (simple) & 7 &accumulated loading factors & 24 \\
CTOA growth: constant front & 8  & [currently unused] & 25 \\
CTOA growth: non-constant front & 9  & [currently unused] & 26 \\
killed cohesive elements   & 10 & $I$-integral results: SIFs & 27 \\
displacement for element nodes & 11 & $I$-integral results: $T$-stresses & 28 \\
velocities for element nodes & 12 & \ \ & \ \\
accelerations for element nodes & 13 & \ \ & \ \\
reactions for element nodes & 14 & \ \ & \ \\
element stresses & 15 & \ & \ \\
element strains & 16 & \ & \ \\
$J$-integral results & 17 & \ & \ \\
\hline
\end{tabular}
\caption{\small Fig. 2.11 -- Supported packet types with the \ti{convert binary packets} command.}
\normalsize
\end{table}



\end{document}

 


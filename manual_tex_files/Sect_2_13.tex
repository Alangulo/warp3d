%
\documentclass[11pt]{report}
\usepackage{geometry} 
\geometry{letterpaper}

%---------------------------------------------
\setlength{\textheight}{630pt}
\setlength{\textwidth}{450pt}
\setlength{\oddsidemargin}{14pt}
\setlength{\parskip}{1ex plus 0.5ex minus 0.2ex}


%----------------------------------------
\usepackage{amsmath}
\usepackage{layout}
\usepackage{color}
\usepackage{hyphenat}

%----------------------------------------------
\usepackage{fancyhdr} \pagestyle{fancy}
\setlength\headheight{15pt}
\lhead{User's Guide - \textit{WARP3D}}
\rhead{\textit{Analysis Restart}}
\fancyfoot[L] {\textit{Chapter {\thechapter}}}
\fancyfoot[C] {\thesection-\thepage}
\fancyfoot[R] {\textit{Analysis Restart}}

%---------------------------------------------------
\usepackage{graphicx}
\usepackage[labelformat=empty]{caption}
\numberwithin{equation}{section}

%---------------------------------------------
%     --- make section headers in helvetica ---
%
\frenchspacing
\usepackage{sectsty} 
\usepackage{xspace}
\allsectionsfont{\sffamily} 
\sectionfont{\large}
\usepackage[small,compact]{titlesec} % reduce white space around sections
%
%----------------------------------------------

%---------  local commands ---------------------

\newcommand{\tb} {\textbf}
\newcommand{\df} {\dotfill}
\newcommand{\nin} {\noindent}
\newcommand{\bmf } {\boldsymbol }  %bold math symbol
\newcommand{\bsf } [1]{\textrm{\textit{#1}}\xspace}
\newcommand{\ul} {\underline}
\newcommand{\hv} {\mathsf}   %helvetica text inside an equation
\newcommand{\eg}{\emph{e.g.},\xspace}
\newcommand{\ti}{\emph}
\newcommand{\noi}{\noindent}

%-------------------------------------
\newcounter{sectrefs}
\setcounter{sectrefs}{0}
\setcounter{chapter}{2}
\setcounter{section}{12}

%--------------------------------------
%--------------------------------------
%---------------------------------------

\begin{document}

\section{Analysis Restart}
%\layout
\noi {\bf{\ti{Create a Restart File}}}

\noi To maintain the highest possible performance, WARP3D allocates all data structures 
in memory during execution and does not use databases on disk to temporarily hold (swap) data 
arrays. At completion of the solution for a load step, the user may request 
creation of a binary (sequential) file of all data required to resume execution 
at that point in the solution. Only the results and status of the most recently 
completed load step are stored in the restart database.
The default form of the save command is simply:
\begin{align*}
& \hv{{\ul{save}}\ }
\end{align*}
\noi The data file created with this command has the name: @\_db where @ denotes 
the first 8 characters of the $<$structure id$>$ (see Section 2.1).

An explicit name for the restart file may be specified with the command
\begin{align*}
& \hv{\ul{save}\ (\ul{to})\ \ul{file})\ <file\ name: label\ or\ string>}
\end{align*}
\noi where a $<$string$>$ must be used if the file name starts 
with/or contains special characters,
\eg the restart file named \ti{ct\_23.wrp} must be enclosed within 
single or double quotes. 

Examples of the \ti{save} command are
\small
\begin{verbatim}
       save 
       save to file bar_step_450 
       save to file "ct_230.wrp"
\end{verbatim}
\normalsize

\noi Restart file sizes increase with the model size and solution characteristics. 
Restart files for large models may easily reach hundreds of MB to 
multiple GB in size.

In a typical analysis, the solution is advanced some number of load steps then a 
restart file generation is requested. Execution can be continued
which makes the restart file provide a solution check-point. 
WARP3D can be also be terminated after writing the restart file and
then executed later to 
output results for the load step in a restart file and/or to
continue the solution for additional load/time steps. The explicit naming 
feature for restart files enables creation of a series of unique restart files at various 
points in the analysis.


\noi {\bf{\ti{Access a Restart File}}}

\noi To restart execution of WARP3D, the first (non-comment) command must be
\begin{align*}
& \hv{\ul{retrieve}\ \ul{structure}\ <structure\ id: label>}\\
& \hv{\ul{restart}\ \ul{structure}\ <structure\ id: label>}\\
\end{align*}
\noi or using an explicit name for the restart file
\begin{align*}
& \hv{ \ul{retrieve}\ (\ul{from})\ \ul{file}\ <file\ name:label\ or\ string> } \\ 
& \hv{ \ul{restart}\ (\ul{from})\ \ul{file}\ <file\ name:label\ or\ string> } \\ 
\end{align*}
\noi where a $<$string$>$ must be used if the file name starts with/or contains 
special characters. The keywords \ti{retrieve} and \ti{restart} are 
synonyms in this command.

Once the restart file is opened and read into memory, WARP3D 
displays a message indicating the load step number $n$ for the restart file 
and the model solution time completed in the analysis. 
Commands to request output, to analyze additional load steps, to define
new load steps, define new loads, solution parameters, etc. 
may then be given as usual.

\end{document}

 


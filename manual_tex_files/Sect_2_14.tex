%
\documentclass[11pt]{report}
\usepackage{geometry} 
\geometry{letterpaper}

%---------------------------------------------
\setlength{\textheight}{630pt}
\setlength{\textwidth}{450pt}
\setlength{\oddsidemargin}{14pt}
\setlength{\parskip}{1ex plus 0.5ex minus 0.2ex}


%----------------------------------------
\usepackage{amsmath}
\usepackage{layout}
\usepackage{color}
\usepackage{hyphenat}

%----------------------------------------------
\usepackage{fancyhdr} \pagestyle{fancy}
\setlength\headheight{15pt}
\lhead{User's Guide - \textit{WARP3D}}
\rhead{\textit{Table Command}}
\fancyfoot[L] {\textit{Chapter {\thechapter}}}
\fancyfoot[C] {\thesection-\thepage}
\fancyfoot[R] {\textit{Table Command}}

%---------------------------------------------------
\usepackage{graphicx}
\usepackage[labelformat=empty]{caption}
\numberwithin{equation}{section}

%---------------------------------------------
%     --- make section headers in helvetica ---
%
\frenchspacing
\usepackage{sectsty} 
\usepackage{xspace}
\allsectionsfont{\sffamily} 
\sectionfont{\large}
\usepackage[small,compact]{titlesec} % reduce white space around sections
%
%----------------------------------------------

%---------  local commands ---------------------

\newcommand{\tb} {\textbf}
\newcommand{\df} {\dotfill}
\newcommand{\nin} {\noindent}
\newcommand{\bmf } {\boldsymbol }  %bold math symbol
\newcommand{\bsf } [1]{\textrm{\textit{#1}}\xspace}
\newcommand{\ul} {\underline}
\newcommand{\hv} {\mathsf}   %helvetica text inside an equation
\newcommand{\eg}{\emph{e.g.},\xspace}
\newcommand{\ti}{\emph}
\newcommand{\noi}{\noindent}



%
%
%        optional definition for bullet lists which
%        reduces white space.
%
\newcommand{\squishlist}{
 \begin{list}{$\bullet$}
  { \setlength{\itemsep}{0pt}
     \setlength{\parsep}{3pt}
     \setlength{\topsep}{3pt}
     \setlength{\partopsep}{0pt}
     \setlength{\leftmargin}{1.5em}
     \setlength{\labelwidth}{1em}
     \setlength{\labelsep}{0.5em} } }

\newcommand{\squishlisttwo}{
 \begin{list}{$\bullet$}
  { \setlength{\itemsep}{0pt}
     \setlength{\parsep}{0pt}
    \setlength{\topsep}{0pt}
    \setlength{\partopsep}{0pt}
    \setlength{\leftmargin}{2em}
    \setlength{\labelwidth}{1.5em}
    \setlength{\labelsep}{0.5em} } }

\newcommand{\squishend}{
  \end{list}  }
%

%-------------------------------------
\newcounter{sectrefs}
\setcounter{sectrefs}{0}
\setcounter{chapter}{2}
\setcounter{section}{13}

%--------------------------------------
%--------------------------------------
%---------------------------------------

\begin{document}

\section{Table Command}

\noi The \ti{table} command provides the capability to store
significant amounts of numerical data in a convenient
2-D array format and to
define various aspects of the model. This new capability
is designed to be extended simply and rapidly as new areas for
the application of tables evolve in the code. At present, the \ti{table} command
supports the simplified definition of aerodynamic pressure loadings on element
surfaces for unsteady, high-speed flight following the concepts of third order
``piston theory". 
\subsection{General Form}
\noi A table defines a 2-D array of numerical values. All tables
have the following components:
\squishlist
\item a user-specified name for identification (up to 24 characters)
\item a specified number of data rows that will be input
\item the ``type" (or purpose) of table, \eg \ti{piston}
\item the identifying label (name) for each column of data in the table; this feature
enables users to order columns of a given type based on their preference
(the \ti{type} of table implicitly sets the required number of data 
values for each row)
\item a sequence of input lines with the numerical values for each row of the table
\squishend

The command to initiate a \ti{table} definition has the form:
\begin{align*}
& \hv{\ul{table}\ <table\ id:label>\ \ul{rows}\ <integer>\ \ul{type}\ <label> }
\end{align*}
\noi and is followed by a command to define the names and ordering of
data columns in the table
\begin{align*}
& \hv{\ul{columns}\ [ <column\ id:label> (,) ] }
\end{align*}
\noi where the list may (as required) span across multiple input lines continued
with commas. The label \ul{type} sets the number of columns for each row and names
of each column given in the following sections.
There is no pre-defined limit on the number of 
definable data rows in a table.

\noi \ul{Note}: an existing table is over-written with a new definition.

All subsequent data lines then have the general form
\begin{align*}
& \hv{[ <column\ value: numr> (,) ] }
\end{align*}
\noi An example table definition is 
\small
\begin{verbatim}
  table example rows 30 type app_3
    columns A_1 B_1  C_1
      0.0 -3.0 6.2
      1.0 4.33 -9.11
      -20.0 42.0 -100.004
       .
       .
       .
      -45.6 100.1 61.4
\end{verbatim}
\normalsize
\

\noi \ul{Note}: The preceding example does not define an actual table in WARP3D.

\noi When a table contains many rows, it may prove convenient
to place the data rows into a separate file and then use the 
WARP3D \ti{*input ...} utility command as in
\small
\begin{verbatim}
  table example rows 3000 type app_3
    columns A_1 B_1  C_1
     *input from file 'big_data_file'
\end{verbatim}
\normalsize

The save and restart capabilities in WARP3D fully support the table features.

\subsection{Table Type: \ti{piston}}

\noindent Aircraft structural elements generate unsteady aerodynamic pressure loads 
related to the fluid velocity and the local deformation. Piston theory provides
a simple method to calculate approximate aerodynamic pressures at relatively large
flight Mach numbers on nearly plane surfaces at a low inclination to the free stream. 
WARP3D incorporates a third order version of piston theory as described 
in [\ref{R:LH1956}, \ref{R:AZ1956}] to
define the instantaneous pressure loading on surfaces of solid elements exposed
to the flow stream. The pressure is determined by

\begin{equation}\label{eqn:3rdpist}
\dfrac{p}{p_3} = 1 + \gamma \left[ M_3 \bar{V}_p
+ \dfrac{\gamma+1}{4} M_3^2 \bar{V}_p^2 
+ \dfrac{\gamma+1}{12} M_3^3 \bar{V}_p^3 
\right] \,
\end{equation}

\noindent where $p$ sets the local pressure on the aircraft surface, 
$p_3$ represents the undisturbed flow pressure at the panel's leading edge, $\gamma$ denotes 
the isentropic exponent (equal to the specific heat ratio of air), 
and $M_3$ provides the Mach number. The term $\bar{V}_p$ indicates
the normalized "piston" velocity, defined as having a 
direction normal to the approach flow at the panel leading edge:

\begin{equation}\label{eqn:normVp}
 \bar V_p = \left( \dfrac{1}{U_3} \dfrac{\partial w}{\partial t}  + \dfrac{\partial w}{\partial s} \right) \, 
\end{equation}

\noindent where $U_3$ denotes the velocity (magnitude) at the panel's leading edge. 
The term $w$ indicates the displacement perpendicular  to the aircraft's 
surface. Eq (\ref{eqn:normVp}) includes partial derivatives of $w$ 
with respect to time and the direction of fluid flow, $s$. 

Piston loads are re-computed at the beginning of every load (time) step
in the solution to reflect the current displacement derivatives with respect to
time and flow direction in $\bar{V}_p$ and also possible changes in 
parameters required in
Eq (\ref{eqn:3rdpist}). The surface pressures from the piston loading 
do not change during equilibrium iterations over a load (time) step. 
Piston pressure values ($p$) are based on the geometry
and displacements at the parametric center of the element surface. 

Piston loading histories are defined in a table using the \ti{type} command 
\ti{piston}. The 8 data columns to define a piston loading are:
\squishlist
\item \ti{time} model time (secs). Time must start at zero and increase
monotonically
\item \ti{mach} mach number ($M_3$)
\item \ti{flow\_pressure}  ($p_3$)
\item \ti{velocity} flow velocity ($U_3$)
\item \ti{isentropic} ($\gamma$)
\item \ti{x\_direction} global $X$-component of flow direction ($s_x$)
\item \ti{y\_direction} global $Y$-component of flow direction ($s_y$)
\item \ti{z\_direction} global $Z$-component of flow direction ($s_z$)
\squishend
\ul{Note:} ($x,y,z$) components will be normalized to define a 
unit direction vector.

Piston properties for model solution times between tabular values will be
computed via linear interpolation. At model solution times beyond the last table row,
piston properties remain fixed at the values specified for the last row.

An example table definition for piston loading is 
\small
\begin{verbatim}
 table simulation_1 rows 5 type piston
  columns time flowpressure mach velocity isentropic,
               xdirection ydirection zdirection
  0.0 0.006548 1.82093 22888.0 1.4 1.0 0.0 0.0
  0.5 0.006453 1.83688 23058.0 1.4 1.0 0.0 0.0
  1.0 0.006358 1.85281 23227.0 1.4 1.0 0.0 0.0
  2.0 0.006358 1.85281 23227.0 1.4 1.0 0.0 0.0
  3.0 0.000000 1.85281 23227.0 1.4 1.0 0.0 0.0
\end{verbatim}
\normalsize

Tables of type piston simplify definitions of the aerodynamic
loading histories applied to the faces of solid elements in the finite element model 
(see Chapter 3 for loadings on solids elements). An example is
\small
\begin{verbatim}
loading model_set_1  
 element loads
      16 face 4 pressure -1.0
      5 16 19 20 face 1 piston simulation_1
      3-10 body by -1.3
\end{verbatim}
\normalsize
\noi The associated table (here, \verb+simulation_1+) must be set 
prior to defining an element loading of type piston. 

%*****************************************************
\subsection {References}
%*****************************************************
\small
\noindent[\refstepcounter{sectrefs}\label {R:LH1956}\thesectrefs]~M. J. Lighthill.
Oscillating airfoils at high mach numbers. \ti{Journal of Aeronautical Sciences}, 
Vol. 20, No. 6, June 1953, pp. 402--406. 
\medskip

\noindent[\refstepcounter{sectrefs}\label {R:AZ1956}\thesectrefs]~H. Ashley and G. 
Zartarian.Piston Theory -- A New Aerodynamic Tool for the Aeroelastician.
\ti{Journal of Aeronautical Sciences}, Vol. 23, No. 12, December 1956, pp. 1109--1118.

\end{document}

 


%!TEX TS-program=pdflatexmk
\documentclass[11pt]{report}
\usepackage{geometry} 
\geometry{letterpaper}

%---------------------------------------------
\setlength{\textheight}{630pt}
\setlength{\textwidth}{450pt}
\setlength{\oddsidemargin}{14pt}
\setlength{\parskip}{1ex plus 0.5ex minus 0.2ex}


%----------------------------------------
\usepackage{amsmath}
\usepackage{layout}
\usepackage{color}
\usepackage{array}

%----------------------------------------------
\usepackage{fancyhdr} \pagestyle{fancy}
\setlength\headheight{15pt}
\lhead{\small{User's Guide - \ti{WARP3D}}}
\rhead{\small{\ti{Element Incidences}}}
\fancyfoot[L] {\small{\ti{Chapter {\thechapter}}\ \   (Updated: 12-15-2011)}}
\fancyfoot[C] {\small{\thesection-\thepage}}
\fancyfoot[R] {\small{\ti{Model Definition}}}

%---------------------------------------------------
\usepackage{graphicx}
\usepackage[labelformat=empty]{caption}
\numberwithin{equation}{section}

%---------------------------------------------
%     --- make section headers in helvetica ---
%
\usepackage{sectsty} 
\usepackage{xspace}
\allsectionsfont{\sffamily} 
\sectionfont{\large}
\usepackage[small,compact]{titlesec} % reduce white space around sections
%---------------------------------------------->
%
%
%   which fonts system for text and equations. with all commented,
%   the default LaTex CM fonts are used
%
%
\frenchspacing
%\usepackage{pxfonts}  % Palatino text 
%\usepackage{mathpazo} % Palatino text
%\usepackage{txfonts}


%---------  local commands ---------------------


\newcommand{\bmf } {\boldsymbol }  %bold math symbol
\newcommand{\bsf } [1]{\textrm{\ti{#1}}\xspace}
\newcommand{\ul} {\underline}
\newcommand{\hv} {\mathsf}   %helvetica text inside an equation
\newcommand{\eg}{\emph{e.g.},\xspace}
\newcommand{\ie}{\emph{i.e.},\xspace}
\newcommand{\vs}{\emph{vs.}\xspace}
\newcommand{\ti}{\emph}
\newcommand{\veps}{\varepsilon}
\newcommand{\ol}{\overline}
\newenvironment{offsetpar}[1]
{\begin{list}{}%
         {\setlength{\leftmargin}{#1}}%
         \item[]%
}
{\end{list}}

%
%
%        optional definition for bullet lists which
%        reduces white space.
%
\newcommand{\squishlist}{
 \begin{list}{$\bullet$}
  { \setlength{\itemsep}{0pt}
     \setlength{\parsep}{3pt}
     \setlength{\topsep}{3pt}
     \setlength{\partopsep}{0pt}
     \setlength{\leftmargin}{1.5em}
     \setlength{\labelwidth}{1em}
     \setlength{\labelsep}{0.5em} } }

\newcommand{\squishlisttwo}{
 \begin{list}{$\bullet$}
  { \setlength{\itemsep}{0pt}
     \setlength{\parsep}{0pt}
    \setlength{\topsep}{0pt}
    \setlength{\partopsep}{0pt}
    \setlength{\leftmargin}{2em}
    \setlength{\labelwidth}{1.5em}
    \setlength{\labelsep}{0.5em} } }

\newcommand{\squishend}{
  \end{list}  }
%


%-------------------------------------
\newcounter{sectrefs}
\setcounter{sectrefs}{0}
\setcounter{figure}{0}
\setcounter{chapter}{2}
\setcounter{section}{4}
\renewcommand{\thefigure}{\thesection.\arabic{figure}}

%
%--------------------------------------
%
%
%
%              start document 
%              ==========
%
%

\begin{document}

%
%------------------------------------------------------------------------------
\section{Element Incidences (Connectivity)}
%------------------------------------------------------------------------------
Each node of an element in the model must be ``mapped" onto the corresponding 
global node. Element incidences establish this correspondence. During model 
definition, the command \ti{incidences} initiates the translation of 
element incidence data. Any number of \ti{incidences} commands may 
be given prior to a \ti{compute} request. The existing incidences 
for elements are simply overwritten by any newly specified values. 
The input syntax is
\begin{align*}
&\hv{\ul{incid}ences} \\
&\qquad \hv{<element\ number:integer>\ \left[ <global\ node\ i:integer\ list> (,)\right]}
\end{align*}
\noindent where $<$global node i$>$ denotes the number of the global 
node to which the element node i is attached. 
Note that the list of global node numbers may be specified as an $<$integer list$>$.

\noindent An example of the incidences command is
\small \begin{verbatim}
   incidences 
      1 13-20 
      2 5 40 65 83 92 120 44 98 
      3	140-144 178 162 183
\end{verbatim}\normalsize

\noindent The number of entries in the $<$integer list$>$ must equal the number of nodes 
on the element (8 for \ti{l3disop}, 12 for \ti{ts12isop}, etc.). Error messages are issued 
by the input processor if the number of nodes is less than required, 
if a node number exceeds the number of structure nodes, etc. 
A warning message is issued if the same node appears more 
than once in the integer list even though collapsed elements are supported.

The ordering of nodes for each element is shown in Chapter 3 where the 
element library is described.

\noindent \ul{Note}: the incidence data must precede the \ti{blocking} information 
described in the next section. This ordering requirement enables 
consistency checking of the blocking data.

\end{document}



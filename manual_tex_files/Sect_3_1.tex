%!TEX TS-program=pdflatexmk
\documentclass[11pt]{report}
\usepackage{geometry} 
\geometry{letterpaper}

%---------------------------------------------
\setlength{\textheight}{630pt}
\setlength{\textwidth}{450pt}
\setlength{\oddsidemargin}{14pt}
\setlength{\parskip}{1ex plus 0.5ex minus 0.2ex}


%----------------------------------------
\usepackage{amsmath}
\usepackage{layout}
\usepackage{color}
\usepackage{array}

%----------------------------------------------
\usepackage{fancyhdr} \pagestyle{fancy}
\setlength\headheight{15pt}
\lhead{\small{User's Guide - \ti{WARP3D}}}
\rhead{\small{\ti{Hex Elements}}}
\fancyfoot[L] {\small{\ti{Chapter {\thechapter}}\ \   (Updated: 9-4-2012)}}
\fancyfoot[C] {\small{\thesection-\thepage}}
\fancyfoot[R] {\small{\ti{Elements and Material Models}}}

%---------------------------------------------------
\usepackage{graphicx}
\usepackage[labelformat=empty]{caption}
\numberwithin{equation}{section}

%---------------------------------------------
%     --- make section headers in helvetica ---
%
\usepackage{sectsty} 
\usepackage{xspace}
\allsectionsfont{\sffamily} 
\sectionfont{\large}
\usepackage[small,compact]{titlesec} % reduce white space around sections
%---------------------------------------------->
%
%
%   which fonts system for text and equations. with all commented,
%   the default LaTex CM fonts are used
%
%
\frenchspacing
%\usepackage{pxfonts}  % Palatino text 
%\usepackage{mathpazo} % Palatino text
%\usepackage{txfonts}


%---------  local commands ---------------------


\newcommand{\bmf } {\boldsymbol }  %bold math symbol
\newcommand{\bsf } [1]{\textrm{\ti{#1}}\xspace}
\newcommand{\ul} {\underline}
\newcommand{\hv} {\mathsf}   %helvetica text inside an equation
\newcommand{\eg}{\emph{e.g.},\xspace}
\newcommand{\ie}{\emph{i.e.},\xspace}
\newcommand{\vs}{\emph{vs.}\xspace}
\newcommand{\ti}{\emph}
\newcommand{\veps}{\varepsilon}
\newcommand{\ol}{\overline}
\newenvironment{offsetpar}[1]
{\begin{list}{}%
         {\setlength{\leftmargin}{#1}}%
         \item[]%
}
{\end{list}}

%
%
%        optional definition for bullet lists which
%        reduces white space.
%
\newcommand{\squishlist}{
 \begin{list}{$\bullet$}
  { \setlength{\itemsep}{0pt}
     \setlength{\parsep}{3pt}
     \setlength{\topsep}{3pt}
     \setlength{\partopsep}{0pt}
     \setlength{\leftmargin}{1.5em}
     \setlength{\labelwidth}{1em}
     \setlength{\labelsep}{0.5em} } }

\newcommand{\squishlisttwo}{
 \begin{list}{$\bullet$}
  { \setlength{\itemsep}{0pt}
     \setlength{\parsep}{0pt}
    \setlength{\topsep}{0pt}
    \setlength{\partopsep}{0pt}
    \setlength{\leftmargin}{2em}
    \setlength{\labelwidth}{1.5em}
    \setlength{\labelsep}{0.5em} } }

\newcommand{\squishend}{
  \end{list}  }
%


%-------------------------------------
\newcounter{sectrefs}
\setcounter{sectrefs}{0}
\setcounter{figure}{0}
\setcounter{chapter}{3}
\setcounter{section}{0}
\renewcommand{\thefigure}{\thesection.\arabic{figure}}

%
%--------------------------------------
%
%
%
%              start document 
%              ==========
%
%

\begin{document}


\LARGE
\hfill
\textbf{Chapter \thechapter}
\rule[0.15in]{450pt}{0.5mm}
\LARGE
\begin{flushright}
 \textbf{
{\fontfamily{phv}\selectfont Elements and Material Models}}
\end{flushright}
\normalsize

%
%------------------------------------------------------------------------------
\section{Isoparametric Hex Elements}
%------------------------------------------------------------------------------

These isoparametric, hexahedral  ``brick" elements with linear
and quadratic displacement fields provide the fundamental 
meshing capability for solids in WARP3D. The 8-node 
element (\ti{l3disop}) employs a conventional tri-linear displacement field. With the
$\bmf{\ol B}$ modifications 
of Hughes [\ref{R:TH1979}], the element exhibits minimal volumetric locking under 
fully incompressible material response and exhibits slightly improved 
bending response. This element performs well under finite deformations 
encountered near severe discontinuities and crack fronts. A simple 
stabilization scheme may be invoked should hour-glassing modes appear 
(infrequently experienced in this elements). Unfortunately, 
the element exhibits shear locking when subjected to very strong bending fields.

The 20-node element provides a quadratic displacement field with the ability 
to model crack front singularities in a focused mesh with element shapes collapsed 
into wedges. With a reduced order of Gauss quadrature, this element  
resolves accurately the strong bending fields without shear locking; 
moreover, the reduced 
integration order also eliminates volumetric locking under fully plastic deformation. 
Thin shell structures are modeled accurately with just one element through the 
thickness when combined with the reduced integration -- unless the analysis 
requires precise resolution of through thickness yielding. In such cases 
multiple elements defined through the thickness becomes necessary to locate Gauss points 
near the outer surfaces.

The 9, 12 and 15-node elements have selected quadratic edges 
that enable transitions between the 8 and 20-node elements while 
maintaining full displacement compatibility. The \ti{patwarp} program 
converts user-defined 8-node 
elements into these transition elements based on shared faces/edges 
between 8 and 20-node elements in the model. In the Patran neutral
file (produced by Patran or other mesh generators), simply connect
8-node and 20-node elements without concerns for displacement compatibility.

Transitions from coarse to fine meshes of the same element type or 
different element types may also be accomplished using the tie-mesh 
approach with multipoint constraints (Section 2.7).

The element formulations support geometrically nonlinear analysis (large 
displacements, rotations, finite strains), materially nonlinear analysis and 
combined geometric/material nonlinear analysis. 

For dynamic analyses, the diagonal (lumped) mass matrix derives from 
scaled terms of the consistent mass matrix.

All element computations take place in the global coordinate system for 
the model. Strains and stresses output by the model reference the 
global coordinate axes.

For modeling initially sharp crack fronts, these elements are frequently 
degenerated or collapsed into a wedge shape. While this modeling 
technique causes no problems for a small-strain analysis, difficulties 
in Newton convergence of the global solution may be experienced 
when the collapsed elements have the geometric nonlinear formulation. 
The remedy involves modeling the crack front as a very small 
tube (\ie a keyhole) or to model the crack tip as an initially 
blunt notch with a root radius comparatively small compared to the 
crack length or remaining ligament length.


%
\begin{figure}[tb]
\begin{center}
\includegraphics[scale=0.6,angle=0]{Fig_Section_3_1_1.pdf} 
\caption{{\small Fig. \thefigure\ Local node ordering for the isoparametric 
hex elements. Isoparametric coordinates for the element corner nodes are listed.}
\label{fig:hex_1}}
%
\end{center}
\end{figure}
%
%
\begin{figure}[htb]
\begin{center}
\includegraphics[scale=0.6,angle=0]{Fig_Section_3_1_2.pdf} 
\caption{{\small Fig. \thefigure. Local node ordering for the isoparametric 
hex transition element \ti{ts9isop} element. Isoparametric 
coordinates for the element corner 
nodes are listed.} \label{fig:hex_2}}
%
\end{center}
\end{figure}
%
%------------------------------------------------------------------------------
\subsection{Node and Integration Point Ordering}
%------------------------------------------------------------------------------
Figures \ref{fig:hex_1} and \ref{fig:hex_2} show the ordering of nodes 
for the elements and the orientation of 
parametric axes $( \xi , \eta , \zeta )$. The 9, 12, 15 and 20-node elements retain the same 
numbering for the 8 corner nodes defined for the \ti{l3disop} element. Note: the node 
ordering for the 20-node element matches that used in Abaqus.

These elements have mass, stiffness and internal forces computed using quadrature. 
Figure \ref{fig:hex_int_pts} tabulates the locations of integration points in 
parametric coordinates. 
The 8-node element is always evaluated using a conventional (2,2,2) Gauss quadrature. 
The same (2,2,2) is defined as the default order for the 9, 12, 15 and 20-node elements. 
For these elements, two additional integration rules are provided as options: a 
9-point rule and a 14-point rule. The 9 and 14 point rules are especially useful 
for the 20-node brick to suppress hourglass modes. Fracture mechanics models 
constructed with 20-node elements tend to develop hourglass modes in those 
elements located on the crack plane just behind the front. Hourglass modes 
can also appear in those elements with multiple ��free� faces, especially those 
subjected to applied loading. The 9 point and 14 point rules offer a remedy 
for the hour glass modes but at the cost of some increased 
computation time and memory storage.

\begin{figure}[htb]
\begin{center}
\includegraphics[scale=0.6,angle=0]{Fig_Section_3_1_3.pdf} 
\caption{{\small Fig. \thefigure. Location of integration points in 
isoparametric coordinates.}\label{fig:hex_int_pts}}
%
\end{center}
\end{figure}
%

To eliminate potential errors due to strongly varying element shapes, the mass 
matrix and equivalent forces for applied body forces are evaluated with 
the 14-point rule for the 9, 12, 15 and 20-node elements. Similarly, the 
equivalent loads for applied face tractions are evaluated using a (3, 3)
Gauss rule. These procedures are applied independent of the integration 
order specified by the user. The user specified integration order is applied 
for stiffness and internal force computations and for strain-stress updating.

Element results are frequently output at the center point which corresponds 
to parametric location $( \xi =0, \eta=0 , \zeta=0 )$.

Isoparametric elements provide a powerful capability to model the geometry 
of irregularly shaped bodies. The parent element in parametric coordinates 
is mapped into the global Cartesian space using (current) coordinates of the 
nodes and the linear interpolation functions. The element behavior remains 
adequate unless the mapped shape becomes unreasonable (either the initial, 
undeformed shape or the current shape for geometric nonlinear analyses). 
Corner angles on each face must be 
$>0^\textrm{o}$ and $<180^\textrm{o}$. The best element 
response is obtained for angles within the 
range $90^\textrm{o} \pm30^\textrm{o}$. Very large aspect 
ratios should be avoided if possible. The best element behavior derives 
from a cubical shape; however, rectangular prism shapes with aspect 
ratios of 10-20 are commonly used without undue loss of accuracy, 
especially if the strain field varies gently in the longer direction.

Element routines check for badly distorted elements by examining the 
determinant of the coordinate Jacobian at the integration points (using 
the current nodal coordinates for geometric nonlinear analysis). Zero 
or negative values indicate a severely distorted element. Messages 
identifying these problems are printed with information about the element.

For edges of elements having mid-side nodes, the nodes must be 
located initially within a narrow range from the geometric center 
of the element edge -- unless mid-side nodes are intentionally moved to the
quarter-point at a crack front for linear-elastic analyses.
%
%------------------------------------------------------------------------------
\subsection{Element Properties}
%------------------------------------------------------------------------------
Table 3.1 summarizes the user-assignable values that control element behavior. 
Element properties are defined by the name of the property, a $<$label$>$, 
followed by a value. Logical properties are set .true. simply by the appearance 
of the property name. The default behavior for the \ti{l3disop} element is: 
small-strain formulation, (2,2,2) Gauss integration, $\bmf{\ol B}$
formulation, and output of 
a short list of strains-stresses at the Gauss points. For the \ti{ts9isop}, \ti{ts12isop}, 
\ti{ts15isop} and \ti{q3disop} elements, the default behavior is: small-strain 
formulation, (2,2,2) Gauss integration, and output of a short list of 
strains-stresses at the Gauss points. 

%
%=================================================
\begin{table}[htb]	
\centering
{
\setlength{\extrarowheight}{2.5pt}
\begin{tabular}{ | l | c |  c | c | }
\hline
Element property & Keyword & Mode & Default Value \\
\hline \hline
Geometrically \ti{linear} formulation &linear &logical  &True  \\ \hline
Geometrically \ti{nonlinear} formulation &nonlinear &logical  &False  \\ \hline
Material associated with element &material &label  &none  \\ \hline
Order of integration  & order & string  & ``2x2x2''$^\dag$  \\ \hline
Use $\bmf{\ol B}$ formulation  & bbar & logical  & True$^*$  \\ \hline
Do not use $\bmf{\ol B}$ formulation  &  no\_bbar & logical  & False$^*$  \\ \hline
Output $( \sigma_{ij},\, \veps_{ij})$ at integration points  & gausspts & logical  & True  \\ \hline
Output $( \sigma_{ij},\, \veps_{ij})$ at element nodes  & nodpts & logical  & False \\ \hline
Output $( \sigma_{ij},\, \veps_{ij})$ at element center  & center\_output & logical  & False \\ \hline
Output minimal set of  $( \sigma_{ij},\, \veps_{ij})$ values  & short & logical  & True \\ \hline
Output full set of $( \sigma_{ij},\, \veps_{ij})$ values  & long & logical  & False \\ \hline
\end{tabular}}

\small {\dag 9pt\_rule and 14pt\_rule are available for 9, 12, 15, and 20-node elements }

\small {$^*\bmf{\ol B}$ is not applicable (or needed) for 9, 12, 15 and 20-node elements}
\normalsize
	
%
\caption{Table \thesection.1 
Properties for the \ti{hex} isoparametric elements.}
\label{table:elem_props}
\end{table}

%
%------------------------------------------------------------------------------
\subsection{Output Options}
%------------------------------------------------------------------------------
Printed strain-stress results may be obtained at the integration points (default), 
the element nodes or at the parametric center-point of the element (0,0,0). 
Figures in Section 2.12 define each of the strain-stress values output by the element.
When the (2,2,2) Gauss integration rule is specified, 
nodal values of $( \sigma_{ij},\, \veps_{ij})$ are computed 
by extrapolation of Gauss point values using linear, Lagrangian polynomials. 

When other integration orders are specified, the nodal values are defined 
simply as the mean values of the integration point values since no clear 
extrapolation procedure exists (values output at all nodes of an element are identical). 
Values of invariants, principal values and directions are computed from these 
extrapolated nodal values. Nodal values of effective strain, Mises stress, energy 
density and state variables dependent on the associated material model are also
the corresponding mean values. 

The center-point values of $( \sigma_{ij},\, \veps_{ij})$ are the simple numerical average of 
Gauss point values. Values of invariants, principal values and directions 
are computed from these averaged, center-point values. Values of 
effective strain, Mises stress, energy density and state variables dependent 
on the associated material model are simply averaged.
The short option requests printing of a reduced set of output values. The 
invariants, principal values and direction cosines are omitted. 
This is the default output option.

\ul{Note}:  in analyses with nonlinear material behavior, the stress values output at
element nodes are \ti{not} those obtained by processing extrapolated or
averaged nodal strains through
the nonlinear material model routines. This can lead, for example, to extrapolated or
averaged stresses at a node that do satisfy a yield criterion. 
%
%------------------------------------------------------------------------------
\subsection{Mass Formulation}
%------------------------------------------------------------------------------
The element (diagonal) mass matrix is evaluated once at the start of 
computations for the first load step. Entries of the lumped mass are 
proportional to the diagonal entries of the element consistent mass. The 
proportionality factor preserves the total mass of the element, \eg the 
sum of the diagonal terms for the $\ddot v_i$ accelerations equals the element mass. 
This procedure always generates positive values for the lumped mass and 
leads to optimal convergence rates with mesh refinement (see [\ref{R:ZT2005}])

The mass matrix diagonal term for each of the three accelerations at 
node $i$ is then
%

\begin{equation}\label{eqn:1}
m_{ii}^e = \alpha \int_{V_e} \rho N_i^2\, dV_e
\end{equation}

%
\noindent  where $\rho$ denotes the 
mass density of the undeformed material. $N_i$ denotes the usual 
interpolating function for the element node $i$. The scaling factor $\alpha$ 
is given by

\begin{equation}\label{eqn:2}
\alpha =  \int_{V_e} \rho\, dV_e \bigg / \left [ \sum_k^{nen}
\int_{V_e} \rho N_k^2\, dV_e
\right ]
\end{equation}

%
\noindent where $nen$ denotes the number of element nodes. A full 
integration order is employed to evaluate these integrals.
%
%------------------------------------------------------------------------------
\subsection{Element Loads}
%------------------------------------------------------------------------------
Loads available for these elements include body forces, face tractions, 
face pressures and uniform temperature changes. In addition, imposed 
nodal temperatures may define a non-uniform temperature 
field over the element. Nodal and element temperature loads 
may be active simultaneously and can be mixed with other types 
of element loadings. A sequence of element load definitions has the form
%

\begin{align*}
&\hv{\ul{element}\ (\ul{loads})} \\
&\hv{\ \ <elements: list>\ <type\ of\ loading\ and\ parameters>}\\
&\hv{\ \ <elements: list>\ <type\ of\ loading\ and\ parameters>}\\
&\qquad\qquad{\bmf{\cdot}}\\
&\qquad\qquad{\bmf{\cdot}}
\end{align*} 
%

\noindent where the $<$type of element loading$>$ is a body force, a face traction,
a face (normal) pressure, an aerodynamic pressure (piston loading),
or a constant temperature over the element volume.
%

\noindent \bf{Body Forces}\rm

\noindent Body forces are specified by the intensity (units of $F/L^3$) and the 
direction along one of the coordinate axes. The body force 
intensity is constant over the element. The body force loads 
are defined by the command
%

\begin{align*}
&\hv{\ul{body}\  (\ul{force}s)\ } \left [
\begin{Bmatrix}
\hv{\ul{bx}} \\ \hv{\ul{by}} \\ \hv{\ul{bz}}      
\end{Bmatrix}
(=)\ \hv{<force\ intensity:number>(,)} \right ]
\end{align*} 
%

\noindent \bf{Face Tractions}\rm


\noindent Tractions applied to the faces of elements may have a direction 
along one of the global coordinate axes or a direction normal to 
the specified face. Figure \ref{fig:hex_faces} defines the face numbers. The 
commands define the loaded face of the element, the loading 
intensity (units of $F/L^2$), and the loading direction. When the traction 
is aligned with one of the coordinate axes, the command has the form
%

\begin{align*}
&\hv{\ul{face}\ (\ul{abaq}us)\ <face\ number:integer>\ (\ul{tract}ions)\ } \left [
\begin{Bmatrix}
\hv{\ul{tx}} \\ \hv{\ul{ty}} \\ \hv{\ul{tz}}      
\end{Bmatrix}
(=)\ \hv{<intensity:number>(,)} \right ]
\end{align*} 

%
\noindent The \ti{abaqus} option applies for hex elements and loadings intended for
WARP3D faces (4,5). These faces are flipped from the Abaqus 
numbering (see Fig. \ref{fig:hex_faces}). With this option, the input translators
change the specified face number 5 to 4 and 4 to 5 before continuing as a convenience
for users having Abaqus input files for the model.

For a normal (pressure) loading, use a command of the form
%

\begin{align*}
&\hv{\ul{face}\ (\ul{abaq}us) <face\ number:integer>\ \ul{press}ure\ (=)\ <intensity:number> }
\end{align*} 

%
\noindent where a positive value for the intensity denotes a load directed into 
the face, \ie a positive pressure loads the face in compression.
%

\begin{figure}[htb]
\begin{center}
\includegraphics[scale=0.7,angle=0]{Fig_Section_3_1_4.pdf} 
\caption{{\small Fig. \thefigure. Face numbers for applying tractions 
to the isoparametric elements.}\label{fig:hex_faces}}
%
\end{center}
\end{figure}

\noindent \bf{Piston Model Pressures}\rm 

\noindent 
Aircraft structural elements generate unsteady aerodynamic pressure loads 
related to the fluid velocity and the local deformation. Piston theory provides
a simple method to calculate approximate aerodynamic pressures at relatively large
flight Mach numbers on nearly plane surfaces at a low inclination to the free stream. 
WARP3D incorporates a third order version of piston theory as described 
in [\ref{R:LH1956}, \ref{R:AZ1956}] to
define the instantaneous pressure loading on surfaces of solid elements exposed
to the flow stream. The pressure is determined by

\begin{equation}\label{eqn:3rdpist}
\dfrac{p}{p_3} = 1 + \gamma \left[ M_3 \bar{V}_p
+ \dfrac{\gamma+1}{4} M_3^2 \bar{V}_p^2 
+ \dfrac{\gamma+1}{12} M_3^3 \bar{V}_p^3 
\right] \,
\end{equation}

\noindent where $p$ sets the local pressure on the aircraft surface, 
$p_3$ represents the undisturbed flow pressure at the panel's leading edge, $\gamma$ denotes 
the isentropic exponent (equal to the specific heat ratio of air), 
and $M_3$ provides the Mach number. The term $\bar{V}_p$ indicates
the normalized piston velocity, defined as having a 
direction normal to the approach flow at the panel leading edge:

\begin{equation}\label{eqn:normVp}
 \bar V_p = \left( \dfrac{1}{U_3} \dfrac{\partial w}{\partial t}  + \dfrac{\partial w}{\partial s} \right) \, 
\end{equation}

\noindent where $U_3$ denotes the velocity (magnitude) at the panel's leading edge. 
The term $w$ indicates the displacement perpendicular  to the aircraft's 
surface. Eq (\ref{eqn:normVp}) includes partial derivatives of $w$ 
with respect to time and the direction of fluid flow, $s$. 

Piston loads are re-computed at the beginning of every load (time) step
in the solution to reflect the current displacement derivatives with respect to
time and flow direction in the
above equation and also possible changes in parameters required in
Eq (\ref{eqn:3rdpist}). The surface pressures from the piston loading 
do not change during equilibrium iterations over a load (time) step. 
Piston pressure values ($p$) are based on the geometry
and displacements at the parametric center of the element surface. 

The following command sets piston loads on element surfaces: 

\begin{align*}
&\hv{\ul{face}\ (\ul{abaq}us) <face\ number:integer>\ \ul{pist}on\ <table:string> }
\end{align*} 

\noindent A table of type \ti{piston} must be defined prior to 
defining the piston loading command. The piston table provides a 
loading history based on flow pressure, Mach number, velocity, 
isentropic exponent, and the fluid flow direction with time. 

\color{black} \noindent \bf{Uniform Temperature}\rm

\noindent A uniform temperature change over the element 
may be imposed through element loads. The command has the form
%

\begin{align*}
&\hv{\ul{temper}ature\ (=)\ <value:number>\ (,) }
\end{align*} 

%
\noindent \bf{Multiple Loads Defined on Same Element}\rm

\noindent Element loads are additive -- if the same element number and loading 
direction appear in two different loading commands the sum of 
two loads is applied to the model. An example sequence 
to define a loading condition and a set of element loads is

\small \begin{verbatim}
       element loads
         1-40 620-800 by 2 face 6 pressure -2.1 temperature 32.4
         140 face 3 tractions tx -0.5 ty 14.34 tz 42.6
         3256-4000 body forces bz 12.3 bx -32.4
         20 body force bx -3
\end{verbatim}\normalsize

\noindent In the above example, element 20 has a normal 
pressure on face 6, a temperature increment,
and a body force in the $x$ direction. 
Specifications for different loading types for a list of elements 
may be combined onto a single line if desired.

\noindent \bf{Large Displacement Effects on Loads}\rm

\noindent When the analysis includes geometric nonlinear effects 
(large displacements), equivalent loads for the incrementally 
applied surface tractions and pressures are re-computed at the beginning 
of each load step using the current (deformed) geometry of the elements.
%

%------------------------------------------------------------------------------
\subsection{Strains-Stresses for Large Displacement Formulation}
%------------------------------------------------------------------------------
The \ti{nonlinear} property requests a geometrically nonlinear element formulation. 
Stress values output by the element are then the Cauchy (true) stresses. The 
Cauchy stress defines tractions over internal surfaces in the \ti{deformed} configuration 
with stress components, $\sigma_{ij}$, aligned with the global 
coordinate axes for the model. The symmetric Cauchy stress tensor is defined by
the equilibrium conditions 
%

\begin{equation}\label{eqn:3}
\int_S \bmf{n \cdot \sigma }\, dS = 
\int_V \left ( \frac{\partial}{\partial \bmf{x}} \right ) \bmf{\cdot \,\sigma}\, dV
\end{equation}

\noindent where $\bmf{n}$ defines a unit outward normal to the deformed surface $S$, and 
$V$ denotes the deformed volume of the body.

The increment of strain that advances the solution from load step $n$ to $n+1$ is 
given by
%

\begin{equation}\label{eqn:4}
\Delta \bmf{\veps}= \bmf{ B} \left( \bmf{x}_{n+\frac{1}{2}} \right ) \Delta \bmf{u}
\end{equation}

\noindent where the $\bmf{B}$ (linear-strain) displacement matrix is evaluated 
at the mid-step 
deformed configuration of the element. Then $\Delta \bmf{\veps}$ corresponds to a 
finite increment of the rate of 
deformation tensor, $\bmf{D}_{n+1/2}\cdot \Delta t$, over the step. For the 8-node,
linear displacement element, $\bmf{B}$ in the above is replaced by $\bmf{\overline B}$.
The corresponding increment of \ti{unrotated} strain, $\Delta \bmf{d}$,
is given by (see Chapter 1 discussion on finite-strain plastictiy)

\begin{equation}\label{eqn:50}
\Delta \bmf{d} = \bmf{R}^T_{n+\frac{1}{2}}\, 
\bmf{D}_{n+\frac{1}{2}}\, \bmf{R}_{n+\frac{1}{2}} \, \Delta t
\end{equation}

\noindent where $\bmf{R}$ is the rotation component of the polar decomposition
of the deformation gradient $\bmf{F=RU}$. Increments ${\Delta \bmf{d}}$
over load steps all refer to the orthogonal coordinate system $\bmf{B_u}$
on the right side of the polar decomposition figure in finite-strain discussions in Chapter 1.
These increments may be summed and then rotated to the final (deformed)
configuration for output purposes via the current ($n=k$)  $\bmf{R}$  

\begin{equation}\label{eqn:51}
\bmf{\bar D} = \bmf{R} \left ( {\sum_{n=1}^{n=k}\Delta \bmf{d}} \right )\bmf{R}^T
\end{equation}

%
\noindent where the symmetric tensor denoted $\bmf{\bar D}$ then defines an 
approximate logarithmic strain measure for the current 
deformed configuration expressed with components in the global coordinate axes. 
Shear terms in the output follow the usual engineering definition, \ie 
$\Delta \gamma_{xy}= 2 \times \Delta \veps_{xy}$.

An example illustrates interpretation of the strain output. Using hex elements,
define a simple axial bar 
with initial length $L_0$ and a deformed length denoted $L$ at each stage of deformation.
The bar lies initially along the global $X$-axis and is loaded to significant 
elastic-plastic elongation
using a number of load steps. The output strain $\veps_{xx}$ for this
elongation loading corresponds to the usual
logarithmic (true) strain provided a sufficient number of load steps are used 
to approximate the defining integral

\begin{equation}\label{eqn:52}
\veps_{xx} =  \sum \Delta \bmf{d}_{11} =  \sum \frac{\Delta u}{L} \approx \int \frac{dL}{L}
\end{equation}

Now define additional load steps with displacement constraints that impose 
a $+90$-degree rigid rotation of the bar around the $Z$-axis. The rotation process defined
by Eq (\ref{eqn:51}) provides output strain values with $\veps_{yy}$ now 
having the $\veps_{xx}$
value prior to the rotation. Similarly, the output $\veps_{xx}$ has the
$\veps_{yy}$ value prior to rotation. The output values of Cauchy stress follow
a similar result from the imposed rigid-body rotation (\ie $\sigma_{yy}$ is
the (true) axial stress).

%------------------------------------------------------------------------------

\subsection{The $\bmf{\overline B}$ Formulation (8-node element)}
%------------------------------------------------------------------------------
Many methods to alleviate volumetric locking under nearly incompressible material
behavior in fully integrated, linear elements have been 
proposed in the literature. The so-called $\bmf{\overline B}$ framework 
(Hughes [\ref{R:TH1979}]) implemented for the \ti{l3disop} element in WARP3D is outlined below.

The strains are divided into deviatoric and dilatational parts in the following manner.
%

\begin{equation}\label{eqn:5}
\veps_{ij} = \veps_{ij}^{dev} +  \veps_{ij}^{dil}\quad \quad 
 \veps_{ij}^{dil} = \frac{1}{3} \delta_{ij} \sum_{k=1}^{k=3} \veps_{kk} \quad\quad
 \veps_{ij}^{dev} = \veps_{ij} - \veps_{ij}^{dil}
\end{equation}
%

\noindent The strain-displacement matrix, $\bmf{B}$, is divided divided into  
dilatational and deviatoric parts in the same manner as
%

\begin{equation}\label{eqn:6}
\bmf{B} = \left [\bmf{B}_1\ \bmf{B}_2\ \ldots\ \bmf{B}_n \right ]
\end{equation}
%
where,
%

\begin{equation}\label{eqn:7}
   \bmf{B}_i =
       \left [
          \begin{matrix}
             B_1 & 0 & 0 \\
             0 & B_2 & 0 \\
             0 & 0 & B_3 \\
             B_2 & B_1 & 0 \\
             0 & B_3 & B_2 \\
             B_3 & 0 & B_1 
         \end{matrix}
        \right ]
        \qquad
  \bmf{B}_i^{dil} = \frac{1}{3}
       \left [
          \begin{matrix}
             B_1 & B_2 & B_3 \\
             B_1 & B_2 & B_3 \\
             B_1 & B_2 & B_3 \\
             0 & 0 & 0  \\
             0 & 0 & 0 \\
             0 & 0 & 0  
         \end{matrix}
        \right ]
        \qquad
 \bmf{B}_i^{dev} = \bmf{B}_i - \bmf{B}_i^{dil}       
\end{equation}
%

\noindent with the subscript $i$ omitted for clarity on each term inside the $[\ ]$ and

%
\begin{equation}\label{eqn:8}
B_{i1}=\frac{\partial N_i}{\partial x} \qquad
B_{i2}=\frac{\partial N_i}{\partial y} \qquad
B_{i3}=\frac{\partial N_i}{\partial z}
\end{equation}
%

The dilatational contribution to the stiffness causes locking for near 
incompressible conditions and is replaced the dilatational 
part of the strain-displacement matrix with a modified dilatational 
part, $\bmf{\overline B}^{dil}$. The strain-displacement matrix is 
replaced by $\bmf{\overline B}$ defined as:

\begin{equation}\label{eqn:9}
 \bmf{\ol B}_i = \bmf{B}_i^{dev} + \bmf{\ol B}_i^{dil}       
  \qquad
   \textrm{where } \bmf{\ol B}_i^{dil} = \frac{1}{3}
       \left [
          \begin{matrix}
             \ol B_1 & \ol B_2 & \ol B_3 \\
             \ol B_1 & \ol B_2 & \ol B_3 \\
             \ol B_1 & \ol B_2 & \ol B_3 \\
             0 & 0 & 0  \\
             0 & 0 & 0 \\
             0 & 0 & 0  
         \end{matrix}
        \right ]
        \qquad
\end{equation}
%
\noindent where again the subscript $i$ on each term in the $[\ ]$ has been omitted. 
The $\bmf{\ol B}$ matrix can then be written out explicitly in 
the following form (with subscript $i$ omitted inside $[\ ]$)
%
\begin{equation}\label{eqn:9}
   \bmf{\ol B}_i =\frac{1}{3}
       \left [
          \begin{matrix}
             2 B_1 + \ol B_1 & \ol B_2 - B_2 & \ol B_3 - B_3 \\
             \ol B_1 - B_1 & 2 B_2 + \ol B_2 & \ol B_3 - B_3 \\
             \ol B_1 - B_1 & \ol B_2 - B_2 & 2 B_3 + \ol B_3 \\
             3 B_2 & 3 B_1 & 0  \\
             0 & 3 B_3 & 3 B_2 \\
             3 B_3 & 0 & 3 B_1 
         \end{matrix}
        \right ]
\end{equation}

\noindent \bf{Mean Dilatation}\rm

\noindent Several  options for defining $\ol{\bmf{B}}_i^{dil}$ have 
been proposed in the literature. Here we use the ``mean dilatation" approach suggested 
by Nagtegaal, et al. [\ref{R:NPR1974}]. A volume averaged 
(mean) $\widetilde {\bmf{B}}_i$ matrix is computed 
over the element as

%
\begin{equation}\label{eqn:10}
   \bmf{\widetilde B}_i =\frac{1}{V_e} \int_{V_e}  \bmf{ B}_i \, d V_e 
\end{equation}

\noindent with $\ol{\bmf{B}}_i^{dil}$ at each Gauss point taken as the 
dilatational component 
of  $\widetilde {\bmf{B}}_i$ as in Eq. (3.10). To save computations, only 
the three terms needed from  $\widetilde {\bmf{B}}_i$ to compute
$\ol{\bmf{B}}_i^{dil}$ are actually evaluated
%

\begin{equation}\label{eqn:11}
\ol B_{i1} = \frac{\partial \ol N_i}{\partial x} = \frac{1}{V_e} \int_{V_e} \frac{\partial  N_i}{\partial x}  \, d V_e 
\end{equation}

%
\begin{equation}\label{eqn:12}
\ol B_{i2} = \frac{\partial \ol N_i}{\partial y} = \frac{1}{V_e} \int_{V_e} \frac{\partial  N_i}{\partial y}  \, d V_e 
\end{equation}

%
\begin{equation}\label{eqn:13}
\ol B_{i3} = \frac{\partial \ol N_i}{\partial z} = \frac{1}{V_e} \int_{V_e} \frac{\partial  N_i}{\partial z}  \, d V_e 
\end{equation}

\noindent using the standard (2,2,2) Gauss quadrature. 

This formulation provides an element with the same dilatational strain 
and mean stress at each of the (2,2,2)  Gauss points. When plane strain 
constraints are imposed on the $\ol{\bmf{B}}$ element, the $\veps_z$
is not restricted to 0 at
 each Gauss point, but is only restricted to 0 over the element as a 
 whole, \ie for \ti{center\_output} the $\veps_z$ value is zero.
 
\noindent \bf{Large Displacement Form}\rm

\noindent When large displacement effects are present, 
the current coordinates of the element nodes are adopted to 
form $\ol{\bmf{B}}$ to compute virtual strains for internal force computation as in 

%
\begin{equation}\label{eqn:14}
\bmf{IF}_e = 
 \int_{V_e} \bmf{\ol B}^T \left ( \bmf{x}_{n+1}\right ) \bmf{\sigma}_{n+1}\, d V_e
\end{equation}
%

\noindent where $\bmf{\sigma}$ denotes the Cauchy stresses and 
$V_e$ the current (deformed) element volume at $n+1$.

\noindent \bf{Hourglass Stabilization}\rm

\noindent The $\ol{\bmf{B}}$ modification which enforces 
constant pressure throughout the element can 
introduce spurious hourglass modes. A classic example 
which illustrates this element behavior involves 
finite compression of a plane-strain block, where the 
ends are restrained from lateral expansion. The 
differences between deformed shapes with and without 
the $\ol{\bmf{B}}$ modification are quite surprising.

A simple stabilization procedure suggested by Nakamura and
Parks [\ref{R:NP1988}]
often helps to suppress this behavior. A specified fraction of the 
usual $\bmf{B}_i^{dil}$  replaces a similar fraction 
of $\ol{\bmf{B}}_i^{dil}$ as

%
\begin{equation}\label{eqn:15}
   \bmf{\ol B}_i =\bmf{B}_i^{dev} + \bmf{\ol B}_i^{dil}  \ \  +
   \ \  \varepsilon \left [ \bmf{B}_i^{dil} -\bmf{\ol B}_i^{dil} \right ]
\end{equation}

\noindent When $\veps=0.0$, the full $\ol{\bmf{B}}$ form of 
the element is obtained; when  $\veps=1.0$, the 
conventional  $\bmf{B}$ matrix for the 8-node element results. 
No extra computational costs incur for $\veps > 0$.

Users specify the value of $\veps$ in the nonlinear solution parameters for the analysis 
with the command (see Section 2.10 on Solution Parameters) 
\ti{bbar stabilization factor }$<$numr$>$.
The default value of $\veps$ is 0.0.

\noindent \bf{Mean Dilatation for Deformation Gradients, $\bmf{\ol F}$}\rm

\noindent Some material models require access to the deformation 
gradients at $n$ and $n+1$ for direct computation of strain-like
terms (\eg the UMAT material model and various crystal
plasticity models). A mean-dilatation form for the deformation gradient
$\bmf{\ol F}$ at an integration point is computed using the modification

%
\begin{equation}\label{eqn:Fbar}
   \bmf{\ol F} =\bmf{F} \left ( \frac{\bar J}{J} \right )^{\frac{1}{3}}
\end{equation}

\noindent where $J = det(\bmf F)$ and the $\bar J$ is the average value of
$J$ over the deformed  element volume computed by

%
\begin{equation}\label{eqn:Jbar}
   \bar J =\frac{V}{V_0}= \frac{1}{V_0} \int_{V_0} J \,dV_0~.
\end{equation}

\noindent and $V_0$ is the undeformed element volume. Note that use of  $\bmf{\ol F}$
does not affect $\bmf R$ computed in the polar decomposition.
%------------------------------------------------------------------------------
\subsection{Collapsed Elements for Fracture Analyses}
%------------------------------------------------------------------------------
\noindent For linear-elastic crack analyses, it often proves convenient to model a 
sharp crack tip that does not open under loading. To support this modeling 
option, simply collapse one face of a hex element to a line by repeating node 
numbers in the incidence list. In Fig. \ref{fig:hex_1}, for example, this option can be 
invoked by assigning the same structure node number for element nodes 4 
and 8 of the \ti{l3disop} element and a different structure node number for elements 
nodes 1 and 5.

During input processing, the input translators issue a warning message for each 
element with repeated nodes in an element incidence list. When this 
modeling approach is intentional, just ignore the warning messages.

For nonlinear analyses, crack front elements often have a face collapsed to a
line but with unique node numbers retained for all nodes. This scheme 
allows blunting deformations of the crack front under increased loading. 
The crack front elements thus have a topologically unique definition but 
geometrically coincident nodes (at least before loading). This modeling 
approach creates no difficulties in the element computations.

%------------------------------------------------------------------------------
\subsection{Temperature Gradients in Linear Displacement Elements}
%------------------------------------------------------------------------------
\noindent When users specify temperatures at nodes of the model, non-zero temperature 
gradients may occur across elements. Elements with linear displacement 
approximations behave very poorly under such loadings (\ie they exhibit 
shear locking). To alleviate this behavior, the element processors that 
compute strains first average the temperatures on the element nodes 
(imposed through the structure level nodal temperature loadings) to construct a 
uniform temperature change over the element. This value is added to 
any specified uniform temperature change imposed through the element 
uniform temperature loads (described above). 
%------------------------------------------------------------------------------

\subsection{Material Property Gradients in Linear Displacement Elements}
%------------------------------------------------------------------------------
\noindent Materials with temperature dependent elastic properties or functionally 
graded materials (elastic properties specified at model nodes) can generate strong 
gradients of strain and stress within elements subjected to otherwise simple loadings. 
Elements with linear displacement fields may develop severe locking under such 
conditions. To help alleviate this effect, a single set of material properties is 
constructed and used in the computations for all the Gauss points of each element 
with linear displacement fields. A model constructed with these elements then 
accommodates the gradient effects through the usual jumps in strain-stress values across 
element boundaries. 

Simple averaging of FGM properties specified for the model 
nodes incident on the element or averaging of the temperatures for the element 
nodes before element level computations accomplishes this smoothing effect.
This averaging process is performed for all solid (hex) elements 
but the \ti{q3disop} element.

%------------------------------------------------------------------------------
\subsection{Examples}
%------------------------------------------------------------------------------
\noindent The following examples illustrate the specification 
of hex elements in a model.
\small \begin{verbatim}
       structure cct
       c
       material a533b 
         properties ....
       c
       number of nodes 25642 22092
       c
       elements
         14000-22092 type l3disop  nonlinear material a533b order 2x2x2,
                             long bbar center_output
         2000-4000 type q3disop  linear material a533b order 2x2x2,
                             long nodpts
       c
                                            
\end{verbatim}\normalsize
%

%*****************************************************
\subsection {References}
%*****************************************************
\small

\noindent[\refstepcounter{sectrefs}\label {R:TH1979}\thesectrefs]~T. R.
Hughes. Generalization of selective integration procedures to anisotropic and nonlinear media.
\ti{International Journal for Numerical Methods in Engineering},
Vol 15, No 9, pages 1413-�1418, 1980. DOI: 10.1002/nme.1620150914

\medskip

\noindent[\refstepcounter{sectrefs}\label {R:ZT2005}\thesectrefs]~O. C. Zienkiewicz
and  R. L. Taylor. Finite Element Method For Solid And Structural Mechanics. 6th Edition.
Butterworth-Heinemann. 2005.

\medskip

\noindent[\refstepcounter{sectrefs}\label {R:NPR1974}\thesectrefs]~J.C. Nagtegaal, D.M. Parks, 
J.R. Rice. On numerically accurate finite element solutions in the fully plastic range.
\ti{Computer Methods in Applied Mechanics and Engineering}, Vol 4, No 2, pp. 153--177, 1974.
DOI 10.1016/0045-7825(74)90032-2.

\medskip
\noindent[\refstepcounter{sectrefs}\label {R:LH1956}\thesectrefs]~M. J. Lighthill.
Oscillating airfoils at high mach numbers. \ti{Journal of Aeronautical Sciences}, 
Vol. 20, No. 6, June 1953, pp. 402--406. 
\medskip

\noindent[\refstepcounter{sectrefs}\label {R:AZ1956}\thesectrefs]~H. Ashley and G. 
Zartarian.Piston Theory -- A New Aerodynamic Tool for the Aeroelastician.
\ti{Journal of Aeronautical Sciences}, Vol. 23, No. 12, December 1956, pp. 1109--1118.

\noindent[\refstepcounter{sectrefs}\label {R:NP1988}\thesectrefs]~T. Nakamura and D. M. Parks.
Three-Dimensional Stress Field Near the Crack Front of a Thin Elastic Plate. \ti{Journal
of Applied Mechanics}, Vol. 55, No 4,  pp. 805--813, 1988.

\end{document}




\documentclass[11pt]{report}
\usepackage{geometry}
\geometry{letterpaper}

%-----------------------------------

%---------------------------------------------
\setlength{\textheight}{630pt}
\setlength{\textwidth}{450pt}
\setlength{\oddsidemargin}{14pt}
\setlength{\parskip}{1ex plus 0.5ex minus 0.2ex}


%----------------------------------------
\usepackage{amsmath}
\usepackage{layout}
\usepackage{color}
\usepackage{array}
\usepackage{bm}
\usepackage{graphicx}
\usepackage{longtable}
\usepackage[us,12hr]{datetime}

%
\usepackage[comma,numbers,sort,compress]{natbib} % numbered, sequential refs 
\renewcommand{\bibsection}{\subsection{\bibname}} %  makes bib a numbered subsection 
\renewcommand{\bibname}{References} % use References not Bibliography for section name
%----------------------------------------------
\usepackage{fancyhdr} \pagestyle{fancy}
\setlength\headheight{15pt}
\lhead{\small{User's Guide - \textit{WARP3D}}}
\rhead{\small{Material: \textit{creep}}}
%\fancyfoot[L] { \small{\  Updated:  \today\ at \currenttime  }
\fancyfoot[L] {\small{\  Updated:  \today\ at \currenttime}}
\fancyfoot[C] {\small{\thesection-\thepage}}
\fancyfoot[R] {\small{\textit{Elements and Material Models}}}

%---------------------------------------------------
\usepackage{graphicx}
% \usepackage[labelformat=empty]{caption}
\numberwithin{equation}{section}

%---------------------------------------------
%     --- make section headers in helvetica ---
%
\usepackage{sectsty} 
\usepackage{xspace}
\allsectionsfont{\sffamily} 
\sectionfont{\large}
\usepackage[small,compact]{titlesec} % reduce white space around sections
%---------------------------------------------->
%
%
%   which fonts system for text and equations. with all commented,
%   the default LaTex CM fonts are used
%
%
\frenchspacing
%\usepackage{pxfonts}  % Palatino text 
%\usepackage{mathpazo} % Palatino text
%\usepackage{txfonts}
\usepackage[stable]{footmisc}

%---------  local commands ---------------------

\newcommand{\ttt} {\texttt}  %typewriter text
\newcommand{\tb} {\textbf}
\newcommand{\nf} {\normalfont}
\newcommand{\df} {\dotfill}
\newcommand{\nin} {\noindent}
\newcommand{\bmf } {\boldsymbol }  %bold math symbol
\newcommand{\bsf } [1]{\textrm{\textit{#1}}\xspace}
\newcommand{\ul} {\underline}
\newcommand{\hv} {\mathsf}   %helvetica text inside an equation
\newcommand{\eg}{\emph{e.g.},\xspace}
\newcommand{\ie}{\emph{i.e.},\xspace}
\newcommand{\ti}{\emph}
\newcommand{\bardelta}{\bar \delta}
\newcommand{\barDelta}{\bar \Delta}
\newcommand{\veps}{\varepsilon}
\newcommand{\noi}{\noindent}

\newenvironment{offsetpar}[1]%
{\begin{list}{}%
         {\setlength{\leftmargin}{#1}}%
         \item[]%
}
{\end{list}}

%
%
%        optional definition for bullet lists which
%        reduces white space.
%
\newcommand{\squishlist}{
 \begin{list}{$\bullet$}
  { \setlength{\itemsep}{0pt}
     \setlength{\parsep}{3pt}
     \setlength{\topsep}{3pt}
     \setlength{\partopsep}{0pt}
     \setlength{\leftmargin}{1.5em}
     \setlength{\labelwidth}{1em}
     \setlength{\labelsep}{0.5em} } }

\newcommand{\squishlisttwo}{
 \begin{list}{$\bullet$}
  { \setlength{\itemsep}{0pt}
     \setlength{\parsep}{0pt}
    \setlength{\topsep}{0pt}
    \setlength{\partopsep}{0pt}
    \setlength{\leftmargin}{2em}
    \setlength{\labelwidth}{1.5em}
    \setlength{\labelsep}{0.5em} } }

\newcommand{\squishend}{
  \end{list}  }
%


%-------------------------------------
\newcounter{sectrefs}
\setcounter{sectrefs}{0}
\setcounter{chapter}{3}
\setcounter{section}{12}
\setcounter{figure}{0}
\renewcommand{\thefigure}{\thesection.\arabic{figure}}
%
%--------------------------------------
%
%
%		Editing
\newcommand{\medit}{\color{blue}}
\newcommand{\eedit}{\color{black}}
\definecolor{gold}{rgb}{0.83, 0.69, 0.22}
%
%
%              start document 
%              ==========
%
%
\begin{document}

\section{Material Model Type: \textit{creep}}
This initial constitutive model for creep adopts the simple Norton relationship
generally  applicable in the secondary creep regime. The total strain rate comprises the
elastic, creep  and thermal rates

\begin{equation} \label{eq:strain_rate}
      \dot{\veps}_{ij} =\dot{\veps}_{ij}^e + \dot{\veps}_{ij}^C + \dot{\veps}_{ij}^\theta
 \end{equation}
 
\noi where each contribution to the rate is given by


\begin{equation} \label{eq:epsdot-elastic}
      \dot{\veps}_{ij}^e =\frac{1+\nu}{E}\dot \sigma_{ij} - \frac{\nu}{E}\dot \sigma_{kk}\delta_{ij}\ ,
\end{equation}

\begin{equation} \label{eq:epsdot-creep}
      \dot{\veps}_{ij}^C = \frac{3}{2}B \sigma_e^n \frac{S_{ij}}{\sigma_e}\ ,
\end{equation}

\begin{equation} \label{eq:epsdot-thermal}
      \dot{\veps}_{ij}^\theta = \alpha_{ij}\dot \theta\ .
\end{equation}


\noi Here a superposed dot denotes the normal time derivative, $E$ is Young's modulus, 
$\nu$ is Poisson's ratio, $S_{ij}$ are the deviatoric components
of the stress $\sigma_{ij}$. $B$ and $n$ are the creep coefficient and exponent 
material properties to define the Norton power model. General anisotropic
thermal expansion coefficients, $\alpha_{ij}$, are supported with the isotropic option also available.
All material properties are taken to remain temperature invariant in the current implementation. 

The equivalent (mises) stress, $\sigma_e$, and the equivalent creep strain rate, 
$\dot \veps_e^C$, are defined by


\begin{equation} \label{eq:mises}
      \sigma_e = \sqrt{   \textstyle\frac{3}{2}   S_{ij}:S_{ij}} \ ,
\end{equation}

\begin{equation} \label{eq:epsdot-equiv}
      \dot \veps_e^C = \sqrt{ \textstyle\frac{2}{3} \dot \veps_{ij}^C:\dot \veps_{ij}^C} \ ,
\end{equation}


\noi where the $3/2$ and $2/3$ factors   provide 
$\dot \veps_e^C = \dot \veps^C = B \sigma_e^n= B \sigma^n$ under simple uniaxial tension loading.
The above shows that creep strains are isochoric in correspondence with $J_2$ plasticity theory.

For finite strain simulations, strain rates $\dot{\bm{\veps}}$ are the (symmetric) components of
the unrotated deformation rate, $\textbf{d}$, with stresses, $\bm{\sigma}$, corresponding to the Cauchy
stresses also expressed on the unrotated configuration. These follow the
Green-Naghdi framework adopted in WARP3D (see details in Chapter 1).

The model predicts the classic behavior characteristic of secondary creep
deformation. At constant loading, the creep strain rate remains 
constant as given by Eq. (\ref{eq:epsdot-creep}) and defines the total strain rate since
$\dot\veps_{ij}^e=0$ (in the absence of temperature changes). At constant displacement
loading(\ie fixed grips), the stresses $\sigma_{ij}$ relax as the elastic strain 
continuously decreases to
accommodate the increasing creep strain.



%*****************************************************
\subsection {Stress Update and Material Tangent}
%*****************************************************
The stress update process adopts a fully implicit, backward Euler integration of the
rate equations represented by Eqs. (\ref{eq:strain_rate})-(\ref{eq:epsdot-equiv}). This provides
unconditional stability with respect to specified time step sizes -- accuracy of the integration
process will require investigation for specific models of interest.

The exact, material consistent tangent has the form
 
\begin{equation}
\mathbf D_c = \left ( \frac{\partial\Delta \bm{\sigma }}{\partial\Delta \bm{\varepsilon}} \right )_{n+1}= 
\left ( \frac{\partial\bm{\sigma}}{\partial\bm{\varepsilon}}\right )_{n+1}  \label{eq:defDc}
\end{equation}

\noi is symmetric, and is computed on the unrotated configuration for geometric nonlinear analyses
(see Chapter 1 for details on transformation of this matrix to spatial 
coordinates from the unrotated system) .

%*****************************************************
\subsection {Model Properties}
%*****************************************************
Table \thesection.1 lists the properties defined for the 
creep material model. At present, the material properties are temperature invariant.
Models may employ isotropic or anisotropic thermal expansion 
coefficients. The mass density and thermal expansion coefficients have default
values of 0.0.

\begin{table}[htbp]
\small
\centering
\setlength{\extrarowheight}{3pt}

\begin{tabular}{|l|>{\centering}m{3cm}|c|c|}
\hline 
\textbf{Property} & \textbf{Keyword} & \textbf{Options/type}&\textbf{Units}\tabularnewline
\hline \hline
Young's modulus & E &real&$F/L^2$\tabularnewline \hline
Poisson's ratio, $\nu$ &nu &real&--\tabularnewline \hline
Creep exponent, $n$ &n\_power & real&--\tabularnewline \hline
Creep coefficient & B & real&$\textrm{MPa}^{-n}\cdot time^{-1}$\tabularnewline \hline
Mass density	& rho & 	real & $\textrm{mass}/L^3$ \\ \hline
Thermal expansion (isotropic)	& alpha & 	real	& $1/T$ \\ \hline
Thermal expansion (anisotropic) & alphax, alphay, alphaz, alphaxy, alphaxz, alphayz & real & $1/T$ \\ \hline
\end{tabular}
\caption{Properties for the \ti{creep} material model.
\label{tab:creep_props}}
\normalsize
\end{table}


%*******************************
\subsection{Model Output: Printed, Flat, Patran Files of Strains-Stresses}
%*******************************
The model provides the usual strains-stresses in the global coordinate system
as described in Chapter 2, section on Output.
Element values may be at integration points, a single set of values obtained by 
averaging integration point values or values at element nodes obtained by
extrapolation from integration point values. 

The strain-stress values described in this section may be obtained as 
printed output, flat result files (text or stream versions) and as
Patran compatible results files (text or binary).  Chapter 2 section on Output provides 
details on output commands, flat and Patran formats, naming
conventions, etc.

For geometrically
nonlinear analyses, stresses follow the conventional definition of
Cauchy stresses ($\bmf{\sigma}$) expressed in the global (model) coordinate system.
Output strain values ($\veps_{xx}, \veps_{yy}, \dots $ $\gamma_{xz}$)
are given by the summed increments of 
\ti{unrotated} strain, $\Delta \mathbf{d}$,
rotated to the fixed-global axes just prior to output
using $\mathbf{R}$ from the polar decomposition,
$\mathbf{F}=\mathbf{RU}$ (these values match closely those output by Abaqus 
which uses summed, incrementally rotated values over the loading history). 
Section 1.8 provides a more extensive discussion on the interpretation of strain 
values for geometrically nonlinear solutions. All output values of shear strains
follow the engineering definition ($\gamma_{xy}, \gamma_{yz}, \gamma_{xz}$).

The strain output quantity labelled \ttt{eff\_eps} is then computed using

\small
\begin{equation} \label{eq:eps-eff-output}
\hv{eff\_eps} =\frac{\sqrt{2}}{3} \sqrt{ (\veps_{xx} - \veps_{yy})^2+
(\veps_{yy} - \veps_{zz})^2 +(\veps_{xx }- \veps_{zz})^2
+ 1.5 ( \gamma_{xy}^2 +  \gamma_{yz}^2 + \gamma_{xz}^2)}\ .
\end{equation}
\normalsize

\noi The stress output quantities labeled \ttt{energy, mises, c1, c2, c3} are computed using


\begin{equation} \label{eq:work-output}
      \hv{energy} =U =\int \bm{\sigma}: ( \dot{\bm{\veps}}-\dot{\bm{\veps}}_{\theta})\,dt\ ,
\end{equation}

\begin{equation} \label{eq:mises-output}
\hv{mises} = \frac{1}{\sqrt{2}} \sqrt{(\sigma_{xx} - \sigma_{yy})^2 +
 (\sigma_{yy} - \sigma_{zz})^2 + (\sigma_{zz} - \sigma_{xx})^2 + 
 6(\sigma_{xy}^2 + \sigma_{yz}^2 + \sigma_{xz}^2)} \ ,
\end{equation}


\begin{equation} \label{eq:crp-eps}
    \hv{c_1} =  \veps_e^C=\int\sqrt{{\textstyle\frac{2}{3}}\dot{\bm{\veps}}^C:\dot{\bm{\veps}}^C}\,dt\ ,
\end{equation}

\begin{equation} \label{eq:output-crp-eps-rate}
  \hv{c_2} = \dot  \veps_e^C=\frac{\sqrt{{\textstyle\frac{2}{3}}\Delta{\bm{\veps}}^C:\Delta{\bm{\veps}}^C}}{t_{n+1}-t_n}\ ,
\end{equation}

\begin{equation} \label{eq:work-output}
      \hv{c_3} =U_{el} =\int \bm{\sigma}:  \dot{\bm{\veps}}_{el}\,dt\ ,
\end{equation}

\normalsize

\noi The integrals above are computed incrementally over the solution
history from $t_n \rightarrow t_{n+1}$ using a trapezoidal rule.

To obtain element (center) averaged values for stress output: (1) integration point values for
$\sigma_{ij}$, \ttt{energy, c1, c2, c3} are simply averaged, (2) the quantity \ttt{mises} is
computed from the averaged $\sigma_{ij}$. For center strain output: (1) integration point values for
$\veps_{ij}$ are simply averaged, (2) the quantity \ttt{eps\_eff} is
computed from the averaged $\veps_{ij}$.  The same procedures are followed to obtain
average stresses/strains output to flat and Patran \ti{element} results files.

In a similar approach to obtain element nodal values for stress output: (1) integration point values for
$\sigma_{ij}$, \ttt{energy, c1, c2, c3} are extrapolated to the nodes, (2) the quantity \ttt{mises} is
computed from the nodal values $\sigma_{ij}$. For nodal  strain output: (1) integration point values for
($\veps_{xx}, \veps_{yy}, \dots $ $\gamma_{xz}$)
 are extrapolated to the nodes, (2) the quantity \ttt{eps\_eff} is
computed from the nodal values $\veps_{ij}$. Each element type has a specific scheme
to extrapolate integration point values to element nodes (consult sections this chapter that
describe each element type). 


%*******************************
\subsection{Model Output: State Variables -- Flat Files}
%*******************************
Computations with the creep model create a number of possible material state variables and other
such quantities that may be of interest to the user of this model.
Access to these variables is made available through the \ti{flat} file system and Patran compatible
files as described
in the Output section of Chapter 2. Flat files for state variables are described here;
Patran files in the next section.

A set of averaged (state) values of the creep model for
each element is written to 
the text and/or stream files. Element averaged values are computed as described
above for printed output. The usual flat files of element strains, stresses for elements associated
with the creep model may also be useful in post processing.

Flat files with creep state variables have names of the form \small{\ttt{ wem\#\#\#\#\#\_text\_creep}}
\normalsize and  \small{\ttt{ wem\#\#\#\#\#\_stream\_creep}}, \normalsize 
where \verb|#####| is the load(time) step number.
Details of the simple file structure are given in Chapter 2 (Output).

The content of each column in a creep states file is described by the associated (ascii) text file named  
\small{\ttt{states\_header\_creep}}\normalsize. This file is written by
WARP3D each time a request for states output is given and there are elements in the
mesh associated with the creep material model.  The current creep header file contains

\small
\begin{verbatim}
#
#  header file for state variable output
#  WARP3D material: creep  
#
#  8 character state labels and longer descriptors
#  material model number, number of state variables 
#
    3
     1  crp-eps    equivalent creep strain
     2  crp-rate   rate of equivalent creep strain
     3  Uelas      elastic energy density
\end{verbatim}
\normalsize 
 
\noi At present, 3 state variables are written for each element at each load(time) step requested
in the usual \ti{output} commands. The 3 values are well described by the short text following the 
integer column number and the label text. The label text is used for example by the \ti{warp3d2exii}
post-processing program in the generation of a \ttt{.exo} file for use in ParaView software. ParaView
displays the label text associated with each data column.  

%*******************************
\subsection{Examples}
%*******************************
In this example, the solution units are
MPa, mm, N, hr.

\small
\begin{verbatim}
 material  GR91 creep e 150000 nu 0.285 n_power 5.0 B 1.0e-16,
                                           rho 7.7e-06 alpha 10.1e-06 

 structure vessel
  number  of nodes 1325832  elements 500000
  elements
    1-100000, 300001-500000 type l3disop nonlinear material GR91  bbar ...
       ...
\end{verbatim}
\small

A test problem for the creep material model is included in the WARP3D 
verification directory in the sub-directory test80. The component is a notched-round
bar subjected to linearly increasing, remote axial displacements for 15 time increments.
The simulation continues for another 35 time increments with fixed remote displacement
-- during this period the stresses relax. The finite element mesh represents a 
small angular slice of the axisymmetric
geometry, loading, and boundary conditions with non-global constraints to enforce zero
normal displacements to the $\theta=constant$ plane.

%*******************************
\subsection{Potential Numerical Issues}
%*******************************
The combination of creep exponent, $n$, and coefficient, $B$, can create 
challenges in the numerical computations even with all double precision
storage and operations. Numerical testing shows that loss of accuracy 
can begin with B smaller than $\approx 10^{-25}$. The creep exponent
is  $\approx 5$ which seldom creates issues. A change in time units to hours
(or days) and corresponding increases in $B$ often alleviates 
such numerical problems. A common set of units for simulation of
long-term, secondary creep is MPa, mm, hr with forces then in N.

\end{document}

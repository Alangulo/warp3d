

\documentclass[11pt]{report}
\usepackage{geometry} 
\geometry{letterpaper}

%---------------------------------------------
\setlength{\textheight}{630pt}
\setlength{\textwidth}{450pt}
\setlength{\oddsidemargin}{14pt}
\setlength{\parskip}{1ex plus 0.5ex minus 0.2ex}


%----------------------------------------
\usepackage{amsmath}
\usepackage{layout}
\usepackage{color}
\usepackage{array}
\usepackage{bm}
\usepackage{graphicx}
\usepackage{longtable}
\usepackage[us,12hr]{datetime}

%----------------------------------------------
\usepackage{fancyhdr} \pagestyle{fancy}
\setlength\headheight{15pt}
\lhead{\small{User's Guide - \textit{WARP3D}}}
\rhead{\small{\textit{Tet Elements}}}
\fancyfoot[L] {\small{\  Updated:  \today\ at \currenttime}}
\fancyfoot[C] {\small{\thesection-\thepage}} 
\fancyfoot[R] {\small{\textit{Elements and Material Models}}}

%---------------------------------------------------
\usepackage{graphicx}
\usepackage[labelformat=empty]{caption}
\numberwithin{equation}{section}
\usepackage{bm}

%---------------------------------------------
%     --- make section headers in helvetica ---
%
\usepackage{sectsty} 
\usepackage{xspace}
\allsectionsfont{\sffamily}  
\sectionfont{\large}
\usepackage[small,compact]{titlesec} % reduce white space around sections
%---------------------------------------------->
%
%
%   which fonts system for text and equations. with all commented,
%   the default LaTex CM fonts are used
%
%
\frenchspacing
%\usepackage{pxfonts}  % Palatino text 
%\usepackage{mathpazo} % Palatino text
%\usepackage{txfonts}
\usepackage[stable]{footmisc}


\newcommand{\ttt} {\texttt}  %typewriter text
\newcommand{\tb} {\textbf}
\newcommand{\nf} {\normalsize}
\newcommand{\bmf } {\boldsymbol }  %bold math symbol
\newcommand{\bsf } [1]{\textrm{\textit{#1}}\xspace}
\newcommand{\ul} {\underline}
\newcommand{\hv} {\mathsf}   %helvetica text inside an equation
\newcommand{\ti}{\emph}

\newcommand{\df} {\dotfill}
\newcommand{\eg}{\emph{e.g.},\xspace}
\newcommand{\ie}{\emph{i.e.},\xspace}
\newcommand{\etal}{\ti{et al.}\xspace}
\newcommand{\vareps}{\varepsilon}
\newcommand{\veps}{\varepsilon}
\newcommand{\vepsilon}{\varepsilon}

\newcommand{\nid}{\noindent}
\newcommand{\noi}{\noindent}

\newcommand{\tfour}{\ti{tet4}\xspace}
\newcommand{\tten}{\ti{tet10}\xspace}




\newenvironment{offsetpar}[1]%
{\begin{list}{}%
         {\setlength{\leftmargin}{#1}}%
         \item[]%
}
{\end{list}}

%
%
%        optional definition for bullet lists which
%        reduces white space.
%
\newcommand{\squishlist}{
 \begin{list}{$\bullet$}
  { \setlength{\itemsep}{0pt}
     \setlength{\parsep}{3pt}
     \setlength{\topsep}{3pt}
     \setlength{\partopsep}{0pt}
     \setlength{\leftmargin}{1.5em}
     \setlength{\labelwidth}{1em}
     \setlength{\labelsep}{0.5em} } }

\newcommand{\squishlisttwo}{
 \begin{list}{$\bullet$}
  { \setlength{\itemsep}{0pt}
     \setlength{\parsep}{0pt}
    \setlength{\topsep}{0pt}
    \setlength{\partopsep}{0pt}
    \setlength{\leftmargin}{2em}
    \setlength{\labelwidth}{1.5em}
    \setlength{\labelsep}{0.5em} } }

\newcommand{\squishend}{
  \end{list}  }
%
\newcounter{Lcount}
\newcommand{\squishnum}{
\begin{list}{\arabic{Lcount}. }
{ \usecounter{Lcount}
\setlength{\itemsep}{0pt}
\setlength{\parsep}{3pt}
\setlength{\topsep}{3pt}
\setlength{\partopsep}{0pt}
\setlength{\leftmargin}{1in}
\setlength{\labelwidth}{1em}
\setlength{\labelsep}{0.5em} } }

\makeatletter
\renewcommand*\env@matrix[1][\arraystretch]{%
  \edef\arraystretch{#1}%
  \hskip -\arraycolsep
  \let\@ifnextchar\new@ifnextchar
  \array{*\c@MaxMatrixCols c}}
\makeatother


%-------------------------------------
\newcounter{sectrefs}
\setcounter{sectrefs}{0}
\setcounter{chapter}{3}
\setcounter{section}{1}
\setcounter{figure}{0}
\renewcommand{\thefigure}{\thesection.\arabic{figure}}
%
%--------------------------------------

%--------------------------------------
%
%
%
%              start document 
%              ==========
%
%
\begin{document}



\section{Solid Elements:\textit{ tet4}, \textit{tet10}}

With the development of sophisticated, automatic mesh generators for complex, 3-D
geometries, tetrahedral elements have become essential for general finite
element modeling of solids. WARP3D provides a family of linear and quadratic
tetrahedral elements based on an isoparametric formulation for solid modeling.
The 4-node element (\tfour) employs a conventional tri-linear displacement field -- 
recommended only for pedagogical purposes.
This constant strain element is analogous to a constant strain triangle element
in 2-D. The 10-node element (\tten) provides a quadratic displacement field with
the ability to better represent bending deformations without severe shear
locking. Under fully incompressible plastic deformations, \tten exhibits minimal
volumetric locking (even when the element edges are curved). A brief
evaluation of the \ti{tet} element performance relative to \ti{hex}
elements for axial, bending and torsion loadings is given in [\ref{R:CK1992}].

The element formulations support geometrically nonlinear analysis (large
displacements, rigid-rotations, finite strains), materially nonlinear analysis and
combined geometric-material nonlinear analysis. 

For dynamic analyses, the diagonal (lumped) mass matrix derives from the scaled
terms of the consistent mass matrix.

To support analyses with functionally graded materials (FGMs), the \ti{tet} elements
utilize material properties defined at the model nodes when specified in the
input (see Section 2.2).

All element computations take place in the global coordinate system for the
model. Strains and stresses output by the model reference the global coordinate
axes.

For modeling initially sharp crack fronts in linear-elastic fracture
 mechanics, the mid--side nodes of the \tten element are often
 moved to the quarter-point position along the edge to better model the strain singularity.
%*****************************************************
\subsection {Node and Integration Point Ordering} 
%*****************************************************
Figure \ref{fig:tet1}  shows the ordering of nodes for the elements and the orientation of the
parametric, tetrahedral coordinate axes $s_1, s_2, s_3, s_4$. \ul{Note}: the node ordering for these
elements follows the same order in Abaqus for elements C3D4, C3D10. Patran also
uses this same ordering.

All computed quantities for these elements are evaluated using numerical
quadrature. Figure  \ref{fig:tet2} tabulates the locations of integration points in the
parametric triangular coordinates. The default order of integration for the \tfour
element is the simple one point rule (\ttt{1pt\_rule}) with coordinates $s_1, s_2, s_3, s_4=0.25$.
For the 10-node element, the stiffness and internal forces are computed with the 4-point
(\ttt{4pt\_rule}) rule by default. Strain-stress values are computed, updated and output at these
4 integration points as well. A 5-point rule (\ttt{5pt\_rule}) is also provided as indicated in
Fig. \ref{fig:tet2}. The element mass matrix and equivalent nodal forces for applied body
loads are computed with a 15-point rule (\tten) and a 4-point rule (\tfour).
Equivalent nodal forces tractions applied on the element faces are computed with
a 7-point rule.

Simple patch tests performed on a mesh containing initially distorted \tten
elements reveals better performance for the 5-point integration rule, \ie the
number of correct significant digits in the computed displacements, strains,
stresses increases significantly.

Element results are frequently output at the the center point -- such values are simple 
averages of the integration point values.

Isoparametric elements provide a powerful capability to model the geometry of
irregularly shaped bodies. The parent element in parametric coordinates is
mapped into the global Cartesian space using (current) coordinates of the nodes
and the interpolation functions. The element behavior remains adequate unless
the mapped shape becomes unreasonable (either the initial, undeformed shape or
the current shape if geometric nonlinear analysis). Large aspect ratios should
be avoided if possible. The best element behavior derives from a equilateral
shape.

Element routines check for badly distorted elements by examining the determinant
of the coordinate Jacobian at the integration points (using the current nodal
coordinates for geometric nonlinear analysis). Zero or negative values indicate
a severely distorted element. Messages identifying these problems are printed
with information about the element.

For edges of the 10-node element, mid-side nodes must be located initially
within a narrow range from the geometric center of the element edge, unless the
user intentionally places them at the quarter-point location to model
singularities at crack fronts.


%
\begin{figure}
\begin{center}
\includegraphics[trim=1.5in 3.0in 0.5in 1.2in, clip=true,scale=0.8,angle=0]{Figure_3_2_1.pdf} 
\caption{{\small Fig. \thefigure\ Local node ordering for the isoparametric tetrahedral elements. 
Tetrahedral coordinates for the element nodes are listed. Node ordering is the same as used in Abaqus.}
\label{fig:tet1}}
%
\end{center}
\end{figure}
%


%
\begin{figure}
\begin{center}
\includegraphics[trim=0.0in 4.0in 0.5in 0.5in, clip=true,scale=0.8,angle=0]{Figure_3_2_2.pdf} 

\caption{\small{Fig. \thefigure\ Location of integration points for tetrahedral elements. Nearest integration 
point for each corner node used to set strain-stress output values at element nodes.}
\label{fig:tet2}}
%
\end{center}
\end{figure}
%

%*****************************************************
\subsection {Element Properties}
%*****************************************************

Table \ref{table:tet_props} summarizes the user-assignable values that 
control element behavior.
Element properties are defined by the name of the property, a $<$label$>$,
followed by a value. Logical properties are set \ttt{true} simply by the appearance
of the property name. The default behavior for the \tfour element is this:
small-strain formulation, 1-point integration, and output of a short list of
strains-stresses at the integration points. For the \tten element, the default
behavior is this: small-strain formulation, 4-point integration, and output of a
short list of strains-stresses at the integration points. 


\renewcommand{\arraystretch}{1.4}
%=================================================
\begin{table}[htb] 
\small
 \centering
% 	\begin{tabular}{ | l | c |  c | c | }
 	\begin{tabular}{ |p{2.5in} | p{1.0in} | p{0.8in} |p{0.9in} | }
 	\hline
 	Element Property & Keyword & Mode & Default Value \\
 	\hline \hline
Geometrically \textit{linear} formulation &\ linear &\ logical\ &\ True\ \\ \hline 
Geometrically \textit{nonlinear} formulation &\ nonlinear &\ logical\ &\ False\ \\ \hline
Material associated with element & \ material	& \ label	& \ none \\ \hline
Order of integration &\ order & \ string	& \ see notes* \\ \hline
Output ($\sigma_{ij},\veps_{ij}$) at integration points & \ gausspts & \ logical	& \ True \\ \hline
Output ($\sigma_{ij},\veps_{ij}$) at element nodes & \ nodpts & \ logical	& \ False \\ \hline
Output ($\sigma_{ij},\veps_{ij}$) at element center & \ center\_output & \ logical & \ False \\ \hline
Output minimal set of ($\sigma_{ij},\veps_{ij}$) values & \ short & \ logical & \ True \\ \hline
Output full set of ($\sigma_{ij},\veps_{ij}$) values & \ long & \ logical & \ False \\ \hline
 	\end{tabular}
*4pt\_rule is the default for \ti{tet10}, 1pt\_rule is default for \tfour. A 5pt\_rule is also available 
and may be preferable (see discussion above). 	
  \caption{\small Table \thesection.1 
Properties for the \ti{tet} elements.
\nf}
 \label{table:tet_props}
\end{table}
%=====================================================


%*****************************************************
\subsection {Output Options}
%*****************************************************
Printed strain-stress results may be obtained at the integration points
(default), one set of values averaged over all integration points, or at the element
nodes. Section 2.12  defines each of the strain-stress values output by the
element.

To generate nodal values of $\sigma_{ij}, \veps_{ij}$  for output, 
the code adopts the following
procedures. For the \tfour element, all nodal values are identical as this is a
constant-strain element. For the \tten element, each corner node has a nearby
integration point for both the 4 and 5 point rules. The four corner nodes are
assigned values computed for the nearest integration point
(see Fig. \ref{fig:tet2}).  Element mid-side node
values then are obtained by averaging values of the adjacent corner nodes. Values of
invariants, principal values and directions are computed from these nodal values.
Values of effective strain, Mises stress, energy density and state variables
dependent on the material model are taken as the integration point values for
the four corner nodes and the averaged values at mid-side nodes. 

The center-point values of $\sigma_{ij}, \veps_{ij}$  are the simple 
numerical average of integration
point values. Values of invariants, principal values and directions are computed
from these averaged, center-point values. Values of effective strain, Mises
stress, energy density and state variables dependent on the material model are
simply averaged.

The \ttt{short} option requests printing of a reduced set of output values (see Section 2.12). 
The invariants, principal values and direction cosines are omitted. This is the
default output option.

%*****************************************************
\subsection {Mass Formulation}
%*****************************************************
The element (diagonal) mass matrix is evaluated once at the start of
computations for the first load (time) step. Entries of the lumped mass are
proportional to the diagonal entries of the element consistent mass. The
proportionality factor preserves the total mass of the element, \eg the sum of
the diagonal terms for the  accelerations equals the element mass. This
procedure always generates positive values for the lumped mass and leads to
optimal convergence rates with mesh refinement. 

The mass matrix diagonal term for each of the three accelerations at node $i$ is then

\begin{equation} \label{eq:mass}
 m_{ii}^e = \alpha \int_{V_e} \rho \,N_i^2\, d V_e
 \end{equation}
\noi where $\rho$ denotes the mass
density of the undeformed material.  $N_i$ denotes the usual interpolating function
for the element node $i$. The scaling factor $\alpha$, is given by

\begin{equation} \label{eq:mass-alpha}
 \alpha = \frac{ \int_{V_e} \rho \, d V_e }{  \sum_{k=1}^{nen} \int_{V_e} \rho \,N_k^2\, d V_e }
 \end{equation}
 
\noi where $nen$ denotes the number of element nodes. As noted previously, a 15-point
full integration order is employed to evaluate these integrals (\tten) and a 4-point rule (\tfour).

 
%*****************************************************
\subsection {Element Loads}
%*****************************************************

Loads available for these elements include body forces, face tractions, face
pressures and uniform temperature changes. Imposed nodal temperatures can
generate a non-uniform temperature field over the element. Node and element
temperature loads may be active simultaneously and can be mixed with other types
of element loadings. A sequence of element load definitions has the form

\small
\begin{align*}
&\hv{\ul{element}\ (\ul{loads})} \\
&\hv{\ \ <elements: list>\ <type\ of\ loading\ and\ parameters>}\\
&\hv{\ \ <elements: list>\ <type\ of\ loading\ and\ parameters>}\\
&\qquad\qquad{\bmf{\cdot}}\\
&\qquad\qquad{\bmf{\cdot}}
\end{align*} 

\nf
%

\noi where the $<$type of loading$>$ is a body force, a face traction, a faced (normal)
pressure, and aerodynamic pressure (piston loading), or a constant temperature
over the element volume.



%
\begin{figure}
\begin{center}
\includegraphics[trim=0.0in 5.8in 0.5in 1.2in, clip=true,scale=0.8,angle=0]{Figure_3_2_3.pdf} 
\caption{{\small Fig. \thefigure\ Face numbering scheme for application of element 
loads. \ul{Note}: face numbers follow the same convention used in Abaqus.}
\label{fig:tet3}}
%
\end{center}
\end{figure}
%
\nf
%*****************************************************
\noi \textbf{Body Forces}
%*****************************************************

\noi Body forces are specified by the intensity (units of $F/L^3$) and the direction
along one of the (global) coordinate axes. The body force intensity is constant over the
element. The body force loads are defined by the command

\small
\begin{align*}
&\hv{\ul{body}\  (\ul{force}s)\ } \left [
\begin{Bmatrix}
\hv{\ul{bx}} \\ \hv{\ul{by}} \\ \hv{\ul{bz}}      
\end{Bmatrix}
(=)\ \hv{<force\ intensity:number>(,)} \right ]
\end{align*} 
\nf
%

%*****************************************************
\noi \textbf{Face Tractions}
%*****************************************************

\noi Tractions applied to the faces of elements may have a direction along one of the
global coordinate axes or a direction normal to the specified face. Figure \ref{fig:tet3}
defines the face numbers. The commands define the loaded face of the element,
the loading intensity (units of $F/L^2$), and the loading direction. When the
traction is aligned with one of the coordinate axes, the command has the form

\small
\begin{align*}
&\hv{\ul{face}\ <face\ number:integer>\ (\ul{tract}ions)\ } \left [
\begin{Bmatrix}
\hv{\ul{tx}} \\ \hv{\ul{ty}} \\ \hv{\ul{tz}}      
\end{Bmatrix}
(=)\ \hv{<intensity:number>(,)} \right ]
\end{align*} 
\nf

\noi For a normal (pressure) loading, use a command of the form

\small
\begin{align*}
&\hv{\ul{face}\ <face\ number:integer>\ \ul{press}ure\ (=)\ <intensity:number> }
\end{align*} 
\nf

\noi where a positive value for the intensity denotes a load directed into the face,
\ie a positive applied pressure loads the face in compression.


%*****************************************************
\noi \textbf{Piston Model Pressures}
%*****************************************************

\noi Aircraft structural elements generate unsteady aerodynamic pressure loads 
related to the fluid velocity and the local deformation. Piston theory provides
a simple method to calculate approximate aerodynamic pressures at relatively large
flight Mach numbers on nearly plane surfaces at a low inclination to the free stream. 
WARP3D incorporates a third order version of piston theory as described 
in [\ref{R:LH1956}, \ref{R:AZ1956}] to define the instantaneous pressure loading 
on surfaces of solid elements exposed
to the flow stream. The pressure is determined by


\begin{equation} \label{eq:piston_a}
\frac{p}{p_3} = 1 + \gamma \left [ M_3 \overline V_p + \frac{\gamma +1}{4}
M_3^2 \overline V_p^2 + \frac{\gamma +1}{12} M_3^3 \overline V_p^3
\right ]
 \end{equation}

\noindent where $p$ sets the local pressure on the aircraft surface, 
$p_3$ represents the undisturbed flow pressure at the panel's leading edge, $\gamma$ denotes 
the isentropic exponent (equal to the specific heat ratio of air), 
and $M_3$ provides the Mach number. The term $\overline{V}_p$ indicates
the normalized piston velocity, defined as having a 
direction normal to the approach flow at the panel leading edge:

\begin{equation} \label{eq:piston_b}
\overline V_p = \left ( \frac{1}{U_3}  \frac{\partial w}{\partial t} + \frac{\partial w}{\partial s} \right ) 
 \end{equation}

\noindent where $U_3$ denotes the velocity magnitude at the panel's leading edge. 
The term $w$ indicates the displacement perpendicular  to the aircraft's 
surface. Eq (\ref{eq:piston_b}) includes partial derivatives of $w$ 
with respect to time and the direction of flow, $s$. 

Piston loads are re-computed at the beginning of every load (time) step
in the solution to reflect the current displacement derivatives with respect to
time and flow direction in the
above equation and also possible changes in parameters required in
Eq (\ref{eqn:3rdpist}). The surface pressures from the piston loading 
do not change during equilibrium iterations over a load (time) step. 
Piston pressure values ($p$) are based on the geometry
and displacements at the parametric center of the element surface. 

The following command sets piston loads on element surfaces: 

\small
\begin{align*}
&\hv{\ul{face}\ <face\ number:integer>\ \ul{pist}on\ <table:string> }
\end{align*} 
\nf

\noindent A table of type \ti{piston} must be defined prior to 
defining the piston loading command (Section 2.14). The piston table provides a 
loading history based on flow pressure, Mach number, velocity, 
isentropic exponent, and the fluid flow direction with time. 

%*****************************************************
\noi \textbf{Uniform Temperature}
%*****************************************************

\noi A uniform temperature change over the complete element may be imposed through
element loads. The command has the form

\small
\begin{align*}
&\hv{\ul{temper}ature\ (=)\ <value:number>\ (,) }
\end{align*} 
\nf

%*****************************************************
\noi \textbf{Multiple Loads Defined on Same Element}
%*****************************************************

\noi Element loads are additive -- if the same element number and loading 
direction appear in two different loading commands the sum of 
two loads is applied to the model. An example sequence 
to define a loading condition and a set of element loads is

\small \begin{verbatim}
       element loads
         1-40 620-800 by 2 face 3 pressure -2.1 temperature 32.4
         140 face 1 tractions tx -0.5 ty 14.34 tz 42.6
         3256-4000 body forces bz 12.3 bx -32.4
         20 body force bx -3
\end{verbatim}\nf

\noindent In the above example, element 20 has a normal 
pressure on face 3, a temperature increment,
and a body force in the $X$ direction. 
Specifications for different loading types for a list of elements 
may be combined onto a single line if desired.

%*****************************************************
\noi \textbf{Large Displacement Effects on Loads}
%*****************************************************

\noi When the analysis includes geometric nonlinear effects 
(large displacements), equivalent loads for the incrementally 
applied surface tractions and pressures are re-computed at the beginning 
of each load step using the current (deformed) geometry of the elements.

%*****************************************************
\subsection {Strains-Stresses for Finite Strain/Rotation}
%*****************************************************

The \ti{nonlinear} property requests a geometrically nonlinear element formulation. 
See Section 3.1.6 for a complete discussion -- the \ti{tet} 
and \ti{hex} formulations are identical.

%*****************************************************
\subsection {Temperature Gradients in \tfour}
%*****************************************************

When users specify temperatures at nodes of the model, non-zero temperature
gradients may exist across elements. Elements with linear displacement
approximations (\tfour) behave very poorly under such loadings (\ie they exhibit
shear locking). To alleviate this behavior, the element processors that compute
strains first average the temperatures imposed on the element nodes (through the
structure level nodal temperature loadings) to construct a uniform temperature
change over the element. This value is added to any specified uniform
temperature change imposed through the element uniform temperature loads
(described above). 


%*****************************************************
\subsection {Material Property Gradients in \tfour}
%*****************************************************

\noi Materials with temperature dependent elastic properties or functionally graded
materials (elastic properties specified at model nodes) can generate strong
gradients of strain and stress within elements subjected to homogeneous
deformation. Elements with linear displacement fields may develop severe locking
under such conditions. To help alleviate this effect, a single set of material
properties is constructed and used in the computations for all the integration
points of each element with linear displacement fields. A model constructed with
these elements then accommodates the gradient effects through jumps in
strain-stress values across element boundaries. Simple averaging of FGM
properties specified for the model nodes incident on the element or averaging of
the temperatures for the element nodes before element level computations
accomplishes this smoothing effect.

This averaging process is performed for the \tfour element.
%*****************************************************
\subsection {Limitations on Current Use}
%*****************************************************

\noi The tetrahedral elements remain relatively new in WARP3D and as such are not yet
fully integrated into all parts of the code. The following limitations are noted
at this time: 
\small
\squishlisttwo
\item \tfour and \tten elements are NOT supported in $J$-integral computations 

\item \tfour and \tten elements are NOT supported by the \ti{patwarp} program to perform
domain decomposition for MPI-based parallel execution (\ti{patwarp} does support the
\ti{tet} elements in the model and loading definition)

\squishend
\nf

%*****************************************************
\subsection {Examples}
%*****************************************************

\noi The following examples illustrate the specification 
of \ti{tet} elements in a model.
\small \begin{verbatim}
       structure cct
       c
       material a533b 
         properties ....
       c
       number of nodes 25642 22092
       c
       elements
         14000-22092 type tet10 nonlinear material a533b order 2x2x2,
                             long bbar center_output
         2000-4000 type tet10 linear material a533b order 2x2x2,
                             long nodpts
       c
                                            
\end{verbatim}\normalsize


%*****************************************************
\subsection {References}
%*****************************************************
\small

\medskip
\noi [\refstepcounter{sectrefs}\label {R:CK1992}\thesectrefs]~A.O. Cifuentes and 
A. Kalbag. A performance study of tetrahedral and hexahedral elements in 
3-D finite element structural analysis. \ti{Finite Elements in Analysis 
and Design}, Vol. 12, 1992, pp. 313-318.


\medskip
\noindent[\refstepcounter{sectrefs}\label {R:LH1956}\thesectrefs]~M. J. Lighthill.
Oscillating airfoils at high mach numbers. \ti{Journal of Aeronautical Sciences}, 
Vol. 20, No. 6, June 1953, pp. 402--406. 
\medskip

\noindent[\refstepcounter{sectrefs}\label {R:AZ1956}\thesectrefs]~H. Ashley and G. 
Zartarian.Piston Theory -- A New Aerodynamic Tool for the Aeroelastician.
\ti{Journal of Aeronautical Sciences}, Vol. 23, No. 12, December 1956, pp. 1109--1118.

\end{document}

%! program = pdflatex

\documentclass[11pt]{report}
\usepackage{geometry} 
\geometry{letterpaper}
%
%
%   margins and inter-paragraph spacing
%
%---------------------------------------------
\setlength{\textheight}{630pt}
\setlength{\textwidth}{450pt}
\setlength{\oddsidemargin}{14pt}
\setlength{\parskip}{1ex plus 0.5ex minus 0.2ex}


%----------------------------------------
\usepackage{amsmath}
\usepackage{layout}
\usepackage{color}
\usepackage{hyphenat}
\usepackage{listings}

%----------------------------------------------
%
%          --- header and footer contents ---
%
\usepackage{fancyhdr} \pagestyle{fancy}
\setlength\headheight{15pt}
\lhead{\small{User's Guide - \textit{WARP3D}}}
\rhead{\small{\textit{Crack Growth}}}
\fancyfoot[L] {\small{\textit{Chapter {\thechapter}}}}
\fancyfoot[C] {\small{\thesection-\thepage}}
\fancyfoot[R] {\small{\textit{Crack Growth}}}

%---------------------------------------------------
\usepackage{graphicx}
\usepackage[labelformat=empty]{caption}
\numberwithin{equation}{section}

%---------------------------------------------
%
%     --- make section headers in helvetica ---
%
\frenchspacing
\usepackage{sectsty} 
\usepackage{xspace}
\allsectionsfont{\sffamily} 
\sectionfont{\large}
\usepackage[small,compact]{titlesec} % reduce white space around sections

%
%
%   which fonts system for text and equations. with all commented,
%   the default LaTex CM fonts are used
%
%
%\usepackage{pxfonts}  % Palatino text 
%\usepackage{mathpazo} % Palatino text
%\usepackage{txfonts}

%----------------------------------------------

%   ---  local commands ---

\newcommand{\bmf } {\boldsymbol }
\newcommand{\bsf } [1]{\textrm{\textit{#1}}\xspace}
\newcommand{\HRule}{\rule{\linewidth}{0.5mm}}
\newcommand{\patwarp}{\textit{patwarp\xspace}}
\newcommand{\eg}{\textit{e.g.,\xspace}}
\newcommand{\ie}{\textit{i.e.,\xspace}}
\newcommand{\ul} {\underline}
\newcommand{\hv} {\mathsf}   %helvetica text inside an equation

%
%
%     --- set chapter number ---
%
%-------------------------------------
\setcounter{chapter}{5}
\newcounter{sectrefs}
%
%
%
%              start document 
%              ==========
%
%
\begin{document}

\LARGE
\hfill
\textbf{Chapter \thechapter}
\rule[0.15in]{450pt}{0.5mm}
\LARGE
\begin{flushright}
 \textbf{
{\fontfamily{phv}\selectfont Crack Growth Procedures}}
\end{flushright}
\normalsize


\section{Introduction}
\noindent 
This chapter describes commands to invoke all three crack growth procedures and
additional details of their implementation in WARP3D. The three procedures
provided in WARP3D to model crack extension are summarized below. 

\subsubsection{Extinction of Solid Elements}

WARP3D provides three procedures are provided to include the effects of crack
extension. In the first type of crack growth, termed \textit{element\_extinction},
solid elements in the model are deleted when a critical condition (damage) is reached
under increased loading. The element stiffness is set to
zero and the remaining forces exerted by the element on adjacent nodes are relaxed to zero
over a user-specified number of load steps or using a simple traction-separation
model. In this procedure the element is not topologically deleted from the model
but it no longer contributes any resistance to loading. In other codes, this
technique of element extinction is often referred to as an element ``death"
option.

\subsubsection{Node Release}

In the second type of crack growth, termed \textit{node\_release}, an increment of crack
extension along a symmetry plane is achieved by the traditional node release
procedure. When conditions for growth are achieved, the displacement constraint
holding the crack closed at that point on the front is replaced by the
corresponding reaction force, which is then relaxed to zero. The force release
process occurs over a user-specified number of steps or using a simple
traction-separation model. Elements behind the growing crack remain in the model
and most often undergo inelastic unloading and then re-yielding as the crack
front continues to extend. The node release procedures support growth along
multiple crack fronts on the symmetry plane and readily model non-uniform growth
along the front, \eg tunneling. Alternatively, the user can request uniform
growth over the thickness at each crack front; the user selects the nodes on
each front which govern growth. Currently, the conditions for growth are
specified by critical crack-tip opening angles (CTOA) for initiation and for
continuation of growth. The growth processor examines each possible CTOA value
using element edges incident on an active front node and grows the crack when
any of those angles exceeds the specified critical value. The user can also
request crack extensions during the analysis irrespective of the crack growth
criterion.

At the present time, only the \textit{l3disop} elements are supported for crack growth
using the node release procedures. Other element types may be used to construct
the finite element model but only \textit{l3disop} elements may be involved in the crack
growth processing.

\subsubsection{Cohesive Elements}

The third type of crack growth procedure termed \textit{cohesive} supports 
\textit{interface}
elements with the material model type \textit{cohesive} (see Chapter 3). 
With this approach, crack extension
evolves as a natural outcome of the computations. Once the cohesive tractions
reduce to a small fraction of their peak values (and displacements of 
the interface elements grow 
large), the element extinction procedures effectively remove the
element from the model and initiate the process to relax the remaining (small)
nodal forces to zero over a fixed number of subsequent load increments.  This
final process generally occurs simultaneously over several interface elements
and concurrently with continued, global loading of the model. This completes the
process to create new, traction-free surfaces in the model.

An extinct 
interface element remains topologically in the model but makes no contribution to
the response (it has no stiffness and contributes no nodal forces). 
For crack extension along symmetry planes, the crack growth processors impose
new displacement constraints as needed on nodes of extinct interface elements
to prevent singular solutions.


\end{document}


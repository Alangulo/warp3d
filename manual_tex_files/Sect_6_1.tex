%!TEX TS-program=pdflatexmk
\documentclass[11pt]{report}
\usepackage{geometry} 
\geometry{letterpaper}

%---------------------------------------------
\setlength{\textheight}{630pt}
\setlength{\textwidth}{450pt}
\setlength{\oddsidemargin}{14pt}
\setlength{\parskip}{1ex plus 0.5ex minus 0.2ex}


%----------------------------------------
\usepackage{amsmath}
\usepackage{layout}
\usepackage{color}
\usepackage{array}

%----------------------------------------------
\usepackage[us,12hr]{datetime}
\usepackage{fancyhdr} \pagestyle{fancy}
\setlength\headheight{15pt}
\lhead{\small{User's Guide - \ti{WARP3D}}}
\rhead{\small{\ti{Contact Procedures}}}
\fancyfoot[L] {\small{\  Updated:  \today\ at \currenttime}}
\fancyfoot[C] {\small{\thesection-\thepage}}
\fancyfoot[R] {\small{\ti{Overview}}}

%---------------------------------------------------
\usepackage{graphicx} 
\usepackage[labelformat=empty]{caption}
\numberwithin{equation}{section}

%---------------------------------------------
%     --- make section headers in helvetica ---
%
\usepackage{sectsty} 
\usepackage{xspace}
\allsectionsfont{\sffamily} 
\sectionfont{\large}
\usepackage[small,compact]{titlesec} % reduce white space around sections
%---------------------------------------------->
%
%
%   which fonts system for text and equations. with all commented,
%   the default LaTex CM fonts are used
%
%
\frenchspacing
%\usepackage{pxfonts}  % Palatino text 
%\usepackage{mathpazo} % Palatino text
%\usepackage{txfonts}


%---------  local commands ---------------------


\newcommand{\bmf } {\boldsymbol }  %bold math symbol
\newcommand{\bsf } [1]{\textrm{\ti{#1}}\xspace}
\newcommand{\ul} {\underline}
\newcommand{\hv} {\mathsf}   %helvetica text inside an equation
\newcommand{\eg}{\emph{e.g.},\xspace}
\newcommand{\ie}{\emph{i.e.},\xspace}
\newcommand{\vs}{\emph{vs.}\xspace}
\newcommand{\ti}{\emph}
\newcommand{\veps}{\varepsilon}
\newcommand{\ol}{\overline}
\newcommand{\nid}{\noindent}

\newenvironment{offsetpar}[1]
{\begin{list}{}%
         {\setlength{\leftmargin}{#1}}%
         \item[]%
}
{\end{list}}

%
%
%        optional definition for bullet lists which
%        reduces white space.
%
\newcommand{\squishlist}{
 \begin{list}{$\bullet$}
  { \setlength{\itemsep}{0pt}
     \setlength{\parsep}{3pt}
     \setlength{\topsep}{3pt}
     \setlength{\partopsep}{0pt}
     \setlength{\leftmargin}{1.5em}
     \setlength{\labelwidth}{1em}
     \setlength{\labelsep}{0.5em} } }

\newcommand{\squishlisttwo}{
 \begin{list}{$\bullet$}
  { \setlength{\itemsep}{0pt}
     \setlength{\parsep}{0pt}
    \setlength{\topsep}{0pt}
    \setlength{\partopsep}{0pt}
    \setlength{\leftmargin}{2em}
    \setlength{\labelwidth}{1.5em}
    \setlength{\labelsep}{0.5em} } }

\newcommand{\squishend}{
  \end{list}  }
%


%-------------------------------------
\newcounter{sectrefs}
\setcounter{sectrefs}{0}
\setcounter{figure}{0}
\setcounter{chapter}{6}
\setcounter{section}{0}
\renewcommand{\thefigure}{\thesection.\arabic{figure}}

%
%--------------------------------------
%
%
%
%              start document 
%              ==========
%
%

\begin{document}


\LARGE
\hfill
\textbf{Chapter \thechapter}
\rule[0.15in]{450pt}{0.5mm}
\LARGE
\begin{flushright}
 \textbf{
{\fontfamily{phv}\selectfont Contact Procedures}}
\end{flushright}
\normalsize

%
%------------------------------------------------------------------------------
\section{Introduction}
%------------------------------------------------------------------------------


Many of the most difficult problems in solid mechanics involve the contact interaction 
between deformable bodies. Key examples include metal forming processes (rolling and 
die-casting) and crash-impact problems. In fracture mechanics, problems of crack closure 
and the proper modeling of experimental conditions often necessitates treatment of 
contact. Regions of a finite element model which undergo contact have boundary 
conditions that vary with the amount of deformation. In contact, implicit analyses 
are especially challenging as the boundary conditions may change abruptly during 
a load increment. This introduces severe nonlinearity into analyses, with 
corresponding changes in model behavior and solution convergence rates.

The large volume of research on contact algorithms offers a range of complexity 
in available approaches to treat contact. The literature typically focuses 
on two aspects of contact algorithms -- contact detection and contact enforcement. 
Detection of contact in the most general form, where any part of a modeled body 
can interact with any other body and/or itself, often consumes most of the computation 
time  and introduces significant complexity in the code. Finite element 
codes often employ a number of techniques to simplify contact detection by explicitly 
identifying regions which may contact each other, or by limiting contact to simple rigid 
surfaces. 

Enforcement of contact proceeds along one of two approaches. 
Lagrange multiplier techniques eliminate penetration of contact surfaces by including 
additional constraint equations directly in the solution of the problem. 
The Lagrange multipliers, included as additional unknown scalar variables, become the 
forces required to eliminate penetration. This technique is very popular in implicit codes; 
however, the additional equations may cause the global structural stiffness matrix to 
become positive semi-definite, thus limiting the techniques available for solution 
of the corresponding linear system. The penalty method provides an alternative 
approach by introducing very stiff springs which prevent further penetration once
detection occurs. This method allows some limited 
penetration between bodies. Increasing the penalty parameter (spring stiffness) 
reduces the amount of penetration. Although the penalty method retains a positive 
definite stiffness matrix, very high penalty parameters can make the stiffness matrix 
ill-conditioned, causing convergence problems and significant round-off error 
in the final results. Efficient analysis requires care in choosing an appropriate 
penalty parameter. Other approaches to enforce contact conditions involve hybrids 
between these methods.

The contact algorithms in WARP3D implement frictionless, rigid-body contact 
using a standard penalty method. \ti{Contact between deformable bodies and self-contact 
is not supported in this implementation}. Contact occurs between nodes of a finite 
element mesh and a user-defined set of rigid contact surfaces. Currently, WARP3D 
supplies three geometries of surfaces which can be arbitrarily oriented in space: 
finite-sized rectangular planes, cylinders, and spheres. Additional surfaces may be 
added as needs warrant. The contact processors allow the assignment of velocities 
to the contact surfaces to simplify simulation of moving boundaries. Each contact 
surface requires a stiffness (penalty parameter) which limits the penetration 
between the model and the surface. Specification of an appropriate stiffness can 
be difficult; Section 6.4 provides some guidance on this topic. The algorithms also 
address issues of multiple contact, where a node penetrates more than one contact 
surface. This enables the appropriate treatment of intrinsic and extrinsic corners 
in contact analyses. The implementation of contact is fully compatible with 
all other parts of WARP3D, including crack growth, finite deformation, all material 
models, restart facilities, and parallel execution (threads only and MPI+threads).

This chapter continues with a description of the contact algorithms used in WARP3D, 
including an overview of the penalty method, a summary of the contact detection techniques, 
a description of the algorithms which handle intersecting contact surfaces, and key 
details of the parallel implementation of contact. A section on contact input describes 
the necessary commands for contact specification, as well as some restrictions on 
contact models. Section 6.4 provides advice on performing analyses with contact. 
The chapter concludes with a section of examples which illustrate three WARP3D 
contact analyses: rolling of a steel bar, crushing of a pipe, and crack closure on a 
pin-loaded C(T) specimen.


\end{document}



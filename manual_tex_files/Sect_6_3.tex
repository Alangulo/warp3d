
\documentclass[11pt]{report}
\usepackage{geometry}
\geometry{letterpaper}

%-----------------------------------

%---------------------------------------------
\setlength{\textheight}{630pt}
\setlength{\textwidth}{450pt}
\setlength{\oddsidemargin}{14pt}
\setlength{\parskip}{1ex plus 0.5ex minus 0.2ex}


%----------------------------------------
\usepackage{amsmath}
\usepackage{layout}
\usepackage{color}
\usepackage{array}
\usepackage{bm}
\usepackage{graphicx}
\usepackage{longtable}
%
\usepackage[comma,numbers,sort,compress]{natbib} % numbered, sequential refs 
\renewcommand{\bibsection}{\subsection{\bibname}} %  makes bib a numbered subsection 
\renewcommand{\bibname}{References} % use References not Bibliography for section name
%----------------------------------------------

\usepackage[us,12hr]{datetime}
\usepackage{fancyhdr} \pagestyle{fancy}
\setlength\headheight{15pt}
\lhead{\small{User's Guide - \textit{WARP3D}}}
\rhead{\small{\textit{Contact Commands}}}
\fancyfoot[L] {\small{\  Updated:  \today\ at \currenttime}}

%\fancyfoot[L] {\small{\textit{Chapter {\thechapter}}\ \   (Updated: 8-10-2015. 9:00 AM)}}
\fancyfoot[C] {\small{\thesection-\thepage}}
\fancyfoot[R] {\small{\textit{Contact Procedures}}}

%---------------------------------------------------
\usepackage{graphicx}
% \usepackage[labelformat=empty]{caption}
\numberwithin{equation}{section}

%---------------------------------------------
%     --- make section headers in helvetica ---
%
\usepackage{sectsty} 
\usepackage{xspace}
\allsectionsfont{\sffamily} 
\sectionfont{\large}
\usepackage[small,compact]{titlesec} % reduce white space around sections
%---------------------------------------------->
%
%
%   which fonts system for text and equations. with all commented,
%   the default LaTex CM fonts are used
%
%
\frenchspacing
%\usepackage{pxfonts}  % Palatino text 
%\usepackage{mathpazo} % Palatino text
%\usepackage{txfonts}
\usepackage[stable]{footmisc}

%---------  local commands ---------------------

\newcommand{\ttt} {\texttt}  %typewriter text
\newcommand{\tb} {\textbf}
\newcommand{\nf} {\normalfont}
\newcommand{\df} {\dotfill}
\newcommand{\nin} {\noindent}
\newcommand{\bmf } {\boldsymbol }  %bold math symbol
\newcommand{\bsf } [1]{\textrm{\textit{#1}}\xspace}
\newcommand{\ul} {\underline}
\newcommand{\hv} {\mathsf}   %helvetica text inside an equation
\newcommand{\eg}{\emph{e.g.},\xspace}
\newcommand{\ie}{\emph{i.e.},\xspace}
\newcommand{\ti}{\emph}
\newcommand{\bardelta}{\bar \delta}
\newcommand{\barDelta}{\bar \Delta}
\newcommand{\veps}{\varepsilon}
\newcommand{\noi}{\noindent}

\newenvironment{offsetpar}[1]%
{\begin{list}{}%
         {\setlength{\leftmargin}{#1}}%
         \item[]%
}
{\end{list}}

%
%
%        optional definition for bullet lists which
%        reduces white space.
%
\newcommand{\squishlist}{
 \begin{list}{$\bullet$}
  { \setlength{\itemsep}{0pt}
     \setlength{\parsep}{3pt}
     \setlength{\topsep}{3pt}
     \setlength{\partopsep}{0pt}
     \setlength{\leftmargin}{1.5em}
     \setlength{\labelwidth}{1em}
     \setlength{\labelsep}{0.5em} } }

\newcommand{\squishlisttwo}{
 \begin{list}{$\bullet$}
  { \setlength{\itemsep}{0pt}
     \setlength{\parsep}{0pt}
    \setlength{\topsep}{0pt}
    \setlength{\partopsep}{0pt}
    \setlength{\leftmargin}{2em}
    \setlength{\labelwidth}{1.5em}
    \setlength{\labelsep}{0.5em} } }

\newcommand{\squishend}{
  \end{list}  }
%


%-------------------------------------
\newcounter{sectrefs}
\setcounter{sectrefs}{0}
\setcounter{chapter}{6}
\setcounter{section}{2}
\setcounter{figure}{0}
\renewcommand{\thefigure}{\thesection.\arabic{figure}}
%
%--------------------------------------
%
%
%		Editing
\newcommand{\medit}{\color{blue}}
\newcommand{\eedit}{\color{black}}
\definecolor{gold}{rgb}{0.83, 0.69, 0.22}
%
%
%              start document 
%              ==========
%
%
\begin{document}

\section{Commands for Contact}
\subsection {Outline of Process}
Contact in WARP3D takes place between a deformable mesh and a set of rigid contact 
surfaces. 
Input involves a block of commands, beginning with the \ttt{contact surfaces} command, 
followed by specifications for each contact surface. The description of a contact 
surface includes information about the type of surface, the geometry, the location 
in space, and the parameters stiffness, surface velocity, etc. 
Errors encountered in the input for a contact surface cause the contact 
processors to ignore the surface. Re-definition or removal of one or more 
contact surfaces may occur between load(time) steps. 
Adequate convergence of the global Newton iterations may require a change of the 
penalty parameters at different times in the analysis (see the tips section 
for additional information). Currently, WARP3D allows 
up to 20 defined contact surfaces in an analysis.

\subsection {Initiating Contact Definition}
The command to initiate the definition of contact surfaces has the form
\begin{align*}
&\hv{\ul{cont}act\ (\ul{surf}aces)} 
\end{align*}

\noi The contact input processors assume the commands following this 
statement pertain to contact. Contact input stops once the 
processors encounter a command they do not recognize.

Contact information in an input file must be placed after the \ttt{blocking} and
\ttt{constraints} have been specified.

\subsection{Description of Contact Surfaces}
WARP3D currently supports three geometries of contact surfaces: 
rectangular surfaces, cylinders, and spheres. Description of a contact surface
 requires the specification of the type of surface, the geometry, the 
 orientation in space, and the parameters stiffness, velocity, etc.
 
The command to initiate the definition of a contact surface is:
  \begin{align*}
& \hv{\ul{surf}ace\ <surface\ number: integer> }
\begin{Bmatrix}
\hv{\ul{plane}} \\ \hv{\ul{cyl}inder}\\ \hv{\ul{sph}ere}
\end{Bmatrix}
\end{align*}
\noi  where \ttt{surface number} is a number between 1 and 20. Only one surface 
 may be assigned to a specific surface number; defining a surface with a 
 specific surface number delete any previous surface definitions 
 assigned to that number. Sequential numbering of contact surfaces is not required.
 
\noindent \bf{Information Required for all Contact Surfaces}\rm
 
\noi All contact surfaces require a stiffness (penalty parameter). The command 
 \begin{align*}
&\hv{(\ul{cont}act)\ \ul{stiff}ness < stiffness: number >}
\end{align*}

\noi specifies the stiffness for the contact surface currently under definition. The stiffness 
value must be a number greater than zero. Efficient analyses may mandate 
altering the stiffness of a surface during solution (see the tips section for 
additional information). There is no default for this value.
Contact surfaces may also move through space over time. The 
following command gives the velocity of the surface:
 \begin{align*}
&\hv{\ul{vel}ocity\ <u\mbox{-}dot: number>\ <v\mbox{-}dot: number >\ <w\mbox{-}dot: number >}
\end{align*}

\noi where the velocities have units of distance per unit time. The default value for  
velocity is zero in all directions. Note that this command requires appropriate 
setting of the time step size (see Section 2.10 on solution parameters). 

Currently, the contact processors translate, but do not rotate, contact surfaces. 
Rotation of a contact surface occurs only through user re-definition of the 
contact surface after each load step.


\noindent \bf{Rectangular Surface}\rm

\noi The rectangular contact surface is a plane located in space with a given normal. 
The normal defines the positive (outward) side of the surface. The 
rectangular surface defines a right rectangular prism extending in the 
negative normal direction a depth as specified in the contact input. All 
nodes falling within the volume of the rectangular prism are considered penetrating 
nodes, with penetration defined as the distance to the rectangular surface. 
See Figure 6.3.1 for a typical rectangular contact surface. Section 6.2.3 describes 
the algorithms which handle intersections between multiple 
rectangular surfaces and/or other contact surfaces.

%
\begin{figure}[tb]
\begin{center}
\includegraphics[trim=1.0in 6.2in 0.5in 1.2in, clip=true,scale=1.0,angle=0]{fig_6_3_1.pdf} 
\caption{{\small Definition of rectangular contact surface.}
\label{fig:z}}
%
\end{center}
\end{figure}
%

Designation of the geometry of the rectangular surface 
requires the specification of three points in space (use the following command three times): 
\begin{align*}
&\hv{\ul{point} <x:number>\ <y:number>\ <z:number>}
\end{align*}

\noi These three points define the location in global coordinates of three 
corners for the rectangle. The first point serves as the base corner
while the second and third points are the corners adjacent 
to the base point; see Figure 6.3.1. The normal of the contact plane follows 
from the three points as follows. Denote the three points as 
$\bmf{p}_1, \bmf{p}_2,$ and $\bmf{p}_3$. 
Define two vectors, $\bmf{v}_1=\bmf{p}_2 - \bmf{p}_1$ and 
$\bmf{v}_2=\bmf{p}_3 - \bmf{p}_1$. The normal is 
$\bmf{v}_3=\bmf{v}_1 \times �\bmf{v}_2$. The 
normal vector defines the positive (outward) side of the contact plane; 
all nodes found on the negative side of the plane are penetrating nodes. 
This places several restrictions on the specification of the three points. 
The vectors $\bmf{v}_1$ and $\bmf{v}_2$ must define a 90-degree 
included angle. Furthermore, 
the order in which the points are input determines the normal vector; 
flipping the definition of points 2 and 3 flips the direction of the normal. 
Users should verify that the normal vector has the correct direction.

The command to specify the depth of the rectangular contact plane is: 
\begin{align*}
&\hv{\ul{depth} <depth\ value: number >}
\end{align*}

\noi where \ti{depth value} is a number greater than zero. The default 
depth is very large,  $10^{10}$.

A typical set of commands to define a rectangular contact surface is:
\small
\begin{verbatim}
    contact surface
       surface 1 plane
          point 3 5 2
          point 5 4 3
          point 4 6 1
          depth 4
          stiffness 1.0e8
          velocity 0.1 0.1 0.1
\end{verbatim} 
\normalsize


\noindent \bf{Cylindrical Surface}\rm

\noi This is a right circular cylinder with a 
finite length. Contact occurs on the curved surface -- nodes penetrating 
the flat circular ends are moved to the nearest point on the 
cylindrical surface. Required geometrical input includes a base 
point, the direction of the center line measured from the base point, 
the length of the cylinder, and the radius. See 
Figure 6.3.2 for an sample contact cylinder.
To input the base point and direction vector, use the commands:
\begin{align*}
&\hv{\ul{point} <x:number>\ <y:number>\ <z:number>}\\
&\hv{\ul{dir}ection <delta\mbox{-}x:number>\ <delta\mbox{-}y:number>\ <delta\mbox{-}z:number>}
\end{align*}
\noi where the specified point and direction correspond to $\bmf{P}$
and $\bmf{V}$ in 
Figure 6.3.2. The direction does not need to be a unit vector; the 
contact processors automatically normalize the direction.
To input the length and radius, use the commands:
\begin{align*}
&\hv{\ul{rad}ius\ <R:number>}\\
&\hv{\ul{len}gth\ <L:number>}
\end{align*}

\noi $R$ and $L$ values must be greater than zero and have no defaults.

%
\begin{figure}[htb]
\begin{center}
\includegraphics[trim=1.0in 6.9in 0.5in 1.3in, clip=true,scale=1.0,angle=0]{fig_6_3_2.pdf} 
\caption{{\small Definition of cylindrical contact surface.}
\label{fig:z}}
%
\end{center}
\end{figure}
%

A typical set of commands to define a cylindrical contact surface is:
\small
\begin{verbatim}
   contact surface 
      surface 1 cylinder
         point 5 2 3
         direction -3 1 -1
         radius 1.5
         length 4
         stiffness 1.0e8
         velocity 0.1 0.1 0.1
\end{verbatim} 
\normalsize

\noindent \bf{Spherical Surface}\rm

\noi The spherical contact surface is a full sphere in space, and requires only the center 
point and the radius as input. The commands for the sphere are:
\begin{align*}
&\hv{\ul{point} <x:number>\ <y:number>\ <z:number>}\\
&\hv{\ul{rad}ius <R:number>}
\end{align*}

\noi where $R$ is a number greater than zero. There are no defaults for these values.
A typical set of commands to define a spherical contact surfaces is:
\small
\begin{verbatim}
    contact surface
       surface 1 sphere
          point 5 2 3
          radius 1.5
          stiffness 1.0e8
          velocity 0.0 -0.1 0.0
\end{verbatim} 
\normalsize

\subsection{Utility Commands}
The clear command provides an easy means to delete all 
contact surfaces. The syntax is:

\begin{align*}
&\hv{\ul{cont}act\ (\ul{surf}aces)} \\
&\hv{\ul{clear}}
\end{align*}

\noi Additionally, the \ttt{dump} command prints the information 
about all of the contact surfaces, including the current geometry 
and parameter values. The printed location of the base point for each 
contact surface is the current location after considering the surface 
velocity and elapsed analysis time. The syntax is:
\begin{align*}
&\hv{\ul{cont}act\ (\ul{surf}aces)} \\
&\hv{\ul{dump}}
\end{align*}

\subsection{Notes on Multiple Contacting Surfaces}
 
The implementation of frictionless, rigid-body contact in WARP3D includes 
procedures to address cases involving multiple intersecting contact surfaces; 
Section 6.2.3 details the interaction algorithms. This allows the creation and 
appropriate handling of corners and other composite rigid surfaces. However, 
the algorithms impose some restrictions and requirements on 
contact surface definitions.

%
\begin{figure}[htb]
\begin{center}
\includegraphics[trim=0.1in 5.5in 1.5in 1.4in, clip=true,scale=1.2,angle=0]{fig_6_3_3.pdf} 
\caption{{\small Overlapping of corners: (a) with no overlap, 
node only returns to surface 1; (b) overlap allows return of node to corner.}
\label{fig:z}}
%
\end{center}
\end{figure}
%

\noindent \bf{Overlapping of Contact Surfaces for Corners}\rm

\noi The proper treatment of corners requires that the contact surfaces 
be defined to form a corner which overlaps slightly. Consider Figure 6.3.3a, 
where two 
rectangular contact surfaces form a corner, but without the required
overlap. For the 
contact processors to return the penetrating node to the corner, they  
consider both planes. However, the node is actually only penetrating 
contact surface 1 in Fig. 6.3.3a. After the the evaluation of contact, 
the node moves 
close to the surface of contact surface 1, but not far enough to move 
into contact surface 2. The node does not return to the corner in this 
case. With the specified overlap as in Figure 6.3.3b,
the contact node violates 
both contact surfaces, thus the contact processors return the node 
to the corner. The amount of specified 
overlap should be greater than the remaining 
penetration after enforcement of contact.

\noindent \bf{Avoid Acute Angles in Corners}\rm

\noi The algorithms which manage intersecting contact surfaces 
operate best with corners formed at obtuse angles. Avoid 
internal or external corners with acute angles. See 
Figure 6.3.4 for examples of acceptable and unacceptable corners.

%
\begin{figure}[htb]
\begin{center}
\includegraphics[trim=0.02in 5.2in 1.1in 1.5in, clip=true,scale=1.3,angle=0]{fig_6_3_4.pdf} 
\caption{{\small Acceptable and unacceptable corner definitions: (a) corners
have obtuse angles (permissible); (b) corners have acute angles (will likely cause problems) }
\label{fig:z}}
%
\end{center}
\end{figure}
%

\noindent \bf{Potential Errors in Intersecting Rectangles with Cylinders and Spheres}\rm

\noi The algorithms which resolve penetration of intersecting contact surfaces 
assume the contact surfaces are planes. Large penetrations into intersecting 
contact surfaces which include curved surfaces, particularly curved surfaces 
with small radii, may cause errors in the return location of the node. The use 
of small load steps, and avoidance of excess intersections between rectangles 
and curved surfaces, can help alleviate this problem.

\subsection{Complete Examples}
The following example is a valid contact definition which includes an 
example of each of the types of contact surfaces.
\small
\begin{verbatim}
    contact surface
       clear
       surface 1 plane
          point 3 5 2
          point 5 4 3
          point 4 6 1
          depth 4
          stiffness 1.0e8
          velocity 0.1 0.1 0.1a
       surface 2 cylinder
           point 5 2 3
           direction -3 1 -1
           radius 1.5
           length 4
          stiffness 1.0e8
          velocity 0.05 0.0 0.05
       surface 3 sphere
           point 5 2 3
           radius 1.5
           stiffness 1.0e8
           velocity 0.0 -0.1 0.0
       dump
\end{verbatim} 
\normalsize

\end{document}

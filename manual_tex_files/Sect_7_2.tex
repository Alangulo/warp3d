%
%
\input{common_preamble.tex}
%
%----------------------------------------------
%
\usepackage{fancyhdr} \pagestyle{fancy}
\setlength\headheight{15pt}
\lhead{\small{User's Guide - \textit{WARP3D}}}
\rhead{\small{\textit{Parallel Execution}}}
\fancyfoot[L] {\small{\textit{Chapter {\thechapter}}\ \   (Updated: 5-10-2013)}}
\fancyfoot[C] {\small{\thesection-\thepage}}
\fancyfoot[R] {\small{\textit{Threads-Only Execution}}}
%
%-------------------------------------
\newcounter{sectrefs}
\setcounter{sectrefs}{0}
\setcounter{chapter}{7}
\setcounter{section}{1}
%
%
\begin{document}
%
\section{Threads-Only Execution (Windows, Linux, OS X)}
\noindent
The threads-only approach provides parallel execution in the processing
blocks of elements, in the Pardiso equation solvers and in the element-by-element
conjugate gradient solver. Input files are identical for this approach on the three
platforms. The operating system commands to start execution are the same on
Linux and OS X, but different on Windows. Refer to Section 2.10 for a more in-depth
discussion of equation solvers.

On each platform, WARP3D is an executable program that does not have a 
graphical user interface. Consequently, the process to execute WARP3D begins with 
the user entering commands into a shell program -- the Bash shell on
Linux/OS X and the command shell on Windows. The subsections here provide details for
each platform.

\subsubsection {Windows}
WARP3D supports current 64 bit versions of the Windows operating 
system (Windows 7, ...) 
executing on Intel and AMD processors. 

The command shell is generally located at
c:\textbackslash Windows\textbackslash System32\textbackslash cmd.exe.
Once the shell command window is open, enter commands to change 
to the directory where the WARP3D input file(s) for the analysis are located. 
You will also need to know where the WARP3D executable is located. 
For this discussion, we assume WARP3D has been installed with the 
executable located at c:\textbackslash 
WARP3D\textbackslash run\_windows\_64\textbackslash warp3d.exe, and that 
the input files for analysis are located in the directory d:\textbackslash analysis\_project. 
The following commands can be entered at the command shell prompt (a \% here):

\footnotesize
\begin{verbatim}
%set warp3d=�c:\WARP3D\run_windows_64\warp3d.exe�
%cd d:\analysis_project
%set MKL_NUM_THREADS=8
%set OMP_NUM_THREADS=8
% %warp3d% <input_file >output_file
\end{verbatim}
\normalsize

\nin The same commands above may be entered into a \ti{.bat} file using Notepad, 
Word, etc. (for example,  runwarp3d.bat) and then executed by simply typing 
runwarp3d.bat at the command shell prompt. To see examples of many features 
available in bat files, examine the Makewarp.bat file included under 
the src directory in the WARP3D distribution.

In these commands above, the \%set warp3d... defines a short-cut 
(shell variable) to access the WARP3D executable. The subsequent \%warp3d\% references 
the contents of the shell variable.
The key commands above are the \%set MKL.. and \%set OMP.. which define 
variables (just within the command shell) to the number of threads that 
the MKL solver will use and the number of threads used in all 
parts of WARP3D that run parallel using OpenMP (\ie OMP). These values 
can also be set from within the WARP3D input file as illustrated below. 

On the line above that invokes WARP3D, the $<$ and $>$ are I/O redirection 
operators uniform across nearly all types of command shells. If 
the $<$ (input) file is omitted, WARP3D begins interactive 
execution in the command shell window, issues a prompt 
character ($>$) and waits for the user to enter an input line.

Many people today work across Linux, OS X and Windows environments; 
they often find the non-standard command and scripting language of the Windows 
command shell quite cumbersome. Fortunately, a large portion of the 
standard Linux command and shell system has been ported to Windows, 
is very robust, and is free. This system is known as Cygwin (http://www.cygwin.com/). 
Cygwin supports all the major Linux command shells 
(bash, csh, tcsh, .... The following commands are taken from 
a simple Bash shell script to start execution of WARP3D (with Cygwin) 
on a Windows system:

\footnotesize
\begin{verbatim}
#!/bin/bash -f
#
#     run  WARP3D on Windows
#
warp_dir="d:/WARP3D_development/warp3d_project"
export MKL_NUM_THREADS=4 
export OMP_NUM_THREADS=4
$warp_dir/run_windows_64/warp3d.exe  <input_file >output_file
exit
\end{verbatim}
\normalsize

%-------------------------------------------------------------------------------------

\subsubsection{Linux and OS X}
The threads-only version of WARP3D runs  
on Linux and OS X. In this section, 
let the threads-only executable be named warp3d.omp (for OpenMP) 

Prior to starting WARP3D, the user sets the shell environment variables to indicate 
the number of threads allocated for use by the Pardiso (MKL) solver and the number of threads 
for use by those parts of WARP3D. The number of MKL threads and the number of 
OpenMP threads can be identical or different as the analyst desires. 
Most often the number of MKL and OpenMP threads are assigned the same value. The
following example illustrates this process enclosed within a simple shell
script file.

\footnotesize
\begin{verbatim}
#!/bin/bash -f
#
#     run WARP3D on Linux or OS X
#
export WARP3D_HOME=/Users/jsmith/warp3d_distribution
export MKL_NUM_THREADS=8 
export OMP_NUM_THREADS=8
$WARP3D_HOME/run_linux_em64t/warp3d.omp  <input_file >output_file 
   or
$WARP3D_HOME/run_mac_os_x/warp3d.omp  <input_file >output_file 
exit
\end{verbatim}
\normalsize
 
\nin The analysis input file need not request use of 
 a specific sparse solver; WARP3D defaults to the Pardiso (MKL) sparse direct solver 
 automatically: 

\small
\begin{verbatim}
  nonlinear analysis parameters
   solution technique sparse direct
   maximum iterations 5
   minimum iterations 1
        .
        .
\end{verbatim}
\normalsize
  
\end{document}

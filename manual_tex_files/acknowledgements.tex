%
\documentclass[10pt]{report}
\usepackage{geometry} 
\geometry{letterpaper}
%
%
%   --- margins and inter-paragraph spacing ---
%
%---------------------------------------------
\setlength{\textheight}{630pt}
\setlength{\textwidth}{450pt}
\setlength{\oddsidemargin}{14pt}
\setlength{\parskip}{1ex plus 0.5ex minus 0.2ex}
%
%----------------------------------------
\usepackage{amsmath}
\usepackage{layout}
%---------------------------------------------
%
%          --- header and footer contents ---
%
\usepackage{fancyhdr} \pagestyle{fancy}
%\setlength\headheight{15pt}
%\lhead{User's Guide - \ti{WARP3D}}
%\rhead{\ti{Revision History}}
\fancyfoot[L] {\small{\ti{Acknowledgements}}}
\fancyfoot[C] {\small{\thepage}}
\fancyfoot[R] {\small{\ti{Updated: 3-6-2014}}}
\renewcommand{\headrulewidth}{0.0 pt}

%   
\frenchspacing
%
%   ---  local commands ---
%
\newcommand{\bmf } {\boldsymbol }
\newcommand{\bsf } [1]{\textrm{\ti{#1}}\xspace}
\newcommand{\HRule}{\rule{\linewidth}{0.5mm}}
\newcommand{\patwarp}{\ti{patwarp\xspace}}
\newcommand{\eg}{\ti{e.g.},\xspace}
\newcommand{\ie}{\ti{i.e.},\xspace}
\newcommand{\ul} {\underline}
\newcommand{\hv} {\mathsf}   %helvetica text inside an equation
\newcommand{\ti}{\emph}
%
%        optional definition for bullet lists which
%        reduces white space.
%
\newcommand{\squishlist}{
 \begin{list}{$\bullet$}
  { \setlength{\itemsep}{0pt}
     \setlength{\parsep}{3pt}
     \setlength{\topsep}{3pt}
     \setlength{\partopsep}{0pt}
     \setlength{\leftmargin}{1.5em}
     \setlength{\labelwidth}{1em}
     \setlength{\labelsep}{0.5em} } }

\newcommand{\squishlisttwo}{
 \begin{list}{$\bullet$}
  { \setlength{\itemsep}{0pt}
     \setlength{\parsep}{0pt}
    \setlength{\topsep}{0pt}
    \setlength{\partopsep}{0pt}
    \setlength{\leftmargin}{2em}
    \setlength{\labelwidth}{1.5em}
    \setlength{\labelsep}{0.5em} } }

\newcommand{\squishend}{
  \end{list}  }
%
%
%   ---  page numbering ---
%
\pagenumbering{roman}
\setcounter{page}{10}
%
%
%
%              start document 
%              ==========
%
%
\begin{document}
\noindent
\LARGE
\begin{center}
 \textbf{
{\fontfamily{phv}\selectfont Acknowledgements}}
\end{center}
\normalsize
%
\begin{center}
\line(1,0){250}
\end{center}

\vspace{0.2in}

%
%
The work described in this guide has been supported by grants principally from 
the U.S. Nuclear Regulatory Commission,  the National Aeronautic and Space 
Administration (Ames and Marshall), the Naval Surface Warfare Center 
(Carderock, Maryland), the Oak Ridge national Laboratory (ORNL), 
the Computational Science \& Engineering program at the University of Illinois
and the University of Illinois through the M.T. Geoffrey Yeh Endowed Chair Fund.

The earliest research on numerical algorithms and software architecture 
of WARP3D were supported by Grant SCCA 90-82144 from the Illinois 
Department of Commerce and Grant DE-FG02-85ER25001 from the 
Department of Energy made to the Center for Supercomputer 
Research and Development at the University of Illinois.

Support for graduate students has been provided by generous fellowships 
from the Office of Naval Research, the Computational Science and 
Engineering program at the University of Illinois, the NASA Graduate 
Fellowship Program, and the U.S. Department of Energy 
Computational Science and Engineering Fellowship Program.  

The tied-contact modeling capability was developed in collaboration 
with Quest Integrity 
Group (office in Boulder, Colorado). Our ongoing collaborations 
with Dr. Greg Thorwald and Dr. Ted Anderson at QIG continue to provide very fruitful. 

Key recent research and development efforts on improved 3-D 
fracture models, MPI parallel implementation and support for 
functionally graded materials and fatigue analyses have been supported 
by: NASA Grants 2-1031, 2-1126, 2-1424 (through the Engineering for 
Complex Systems Program) and 8-1751; and the Oak Ridge National 
Laboratory (ORNL) through grant 4000095143.  We are particularly 
pleased to recognize the technical contributions of Dr. Rob Tregoning (USNRC), 
Roy Hampton (NASA-Ames), Dr. Tina Panontin (Chief Engineer, NASA-Ames), 
Doug Wells and Dr. Phillip Allen (NASA-MSFC), and Drs. Richard Bass, Sean Yin 
and Sam Sham (ORNL).

NASA-MSFC, the Oak Ridge National Laboratory and the U.S. Nuclear 
Regulatory Commission provided partial financial support for the 
sabbatical of Professor Dodds during 2009-2010. That support 
enabled significant new fracture modeling and computational 
capabilities to be included in WARP3D.

The \ti{hypre} solver now integrated in WARP3D for the 
MPI-based solution of extremely large models 
is developed by the CASC group at Lawrence Livermore National Laboratory
(\ti{https:computation.llnl.gov/casc}).

New work to improve modeling capabilities for complex manufacturing processes
is partially supported by Caterpillar, Inc. and Engineering Mechanics 
Research Corporation of Columbus 
through the manufacturing competitiveness program of DoE.

Development of the crystal plasticity material model was supported in part by the
National Defense Science and Engineering Graduate (NDSEG) Fellowship Program.

Development of the expanded interfaces to provide Abaqus-like interface for user routines 
has been implemented to support work on Department of Energy projects. 
\end{document}


